\section{Homophily: \emph{"Birds of a feather flock together"}}
\label{sec:homophily}
%\vspace{-0.2cm}
%\begin{center} \emph{Birds of a feather flock together} \end{center}
%\vspace{0.1cm}

Homophily refers to the tendency of individuals to connect to similar others: two individuals (and thus their corresponding nodes in a social network) are more likely to be connected if they share common characteristics~\cite{mcpherson2001birds,lazarsfeld1954friendship}. The characteristics often considered are inherent to the individuals: they may represent their social status, their preferences, their interest, ... A related notion is the one of {\it assortativity}, which is slightly more general since it applies to any network, and not just social networks, and refers to the tendency of nodes in networks to be connected to others that are similar in some way.

A definition of homophily has been proposed in~\cite{la2010randomization}. However, this definition, which relies on a single characteristic (as age or gender), does not allow one to assess whether latent models for link prediction capture the homophily effect or not. We thus introduce a new definition of homophily below, which directly aims at this:
%
\begin{definition}[Homophily]
	Let $\mathcal{M}$ be a link prediction model as defined above and $s$ a similarity measure between nodes. We say that \emph{$\mathcal{M}$ captures the homophily effect} iff, $\forall (i,j,i',j') \in V^4$:
%
\begin{equation}
s(i,j) > s(i',j')  \implies \pr(y_{ij}=1 \mid \mathcal{M}) > \pr(y_{i'j'}=1  \mid \mathcal{M}) \nonumber
\end{equation}
%
A model which verifies this condition is said to be \emph{homophilic} under the similarity $s$.
\end{definition}
%
As one can note, this definition directly captures the effect "if two nodes are more similar, then they are more likely to be connected The similarity function assesses to which extent two nodes share the same latent characteristics. In the case of $\mathcal{M}_e$, these characteristics are captured in the latent features $\mat{F}$ as the model implicitly tries to relate through $\mat{F}$ nodes that are connected in the observations. In the case of $\mathcal{M}_g$ however, there is no mechanism to define latent characteristics as $\mat{F}$ is solely defined from probability distributions that say nothing about possible shared characteristics of nodes. There is thus no sense in this case to talk about homophily. For this reason, we focus only on $\mathcal{M}_e$ for homophily. 

The estimated matrix $\mat{\hat{F}}$ captures some latent characteristics of the nodes, whereas the estimated matrix $\mat{\hat{\Phi}}$ captures the correlations between these latent characteristics. One can thus define, on their basis, a "natural" similarity between nodes as follows:
%
\begin{equation}
s_n(i,j) = \mat{\hat{f}}_{i} \mat{\hat{\Phi}} \mat{\hat{f}}_j^\top \nonumber
%\label{eq:natural-sim}
\end{equation}
%
It is straightforward to show that $\mathcal{M}_e$ is homophilic wrt to $s_n$, as $s_n(i,j)$ corresponds to the probability of generating a link between nodes $i$ and $j$ (Eq.   ~\ref{eq:link-me}). We thus have the following property:
%
\begin{proposition}[] $\M_e$ is homophilic wrt the natural similarities $s_n$.
\end{proposition}

The homophily defined by $s_n$ is trivial because there is a direct mapping between the nodes similarity and the links likelihood. Though, this proposition carries the idea that we can always choose a nodes similarity that would conserve the homophily effect but with a loss of interpretability of the similarity metric. An example of such conservation is any homothetic transformation of the natural similarity.

Therefore, a more interesting question would be to inspect the homophilic effect with a similarity that only depends on the latent features. It should be noticed that this other interpretation corresponds to the classical notion of homophily in social networks analysis according to which nodes are more likely to be connected if they share the same characteristics. This leads us to define a latent similarity between nodes which is decorrelated of $\Phi$ (Note that $\Phi$ encode the metric to go from the feature space, to the probability space):

\begin{align}
&s_l(i,j) = \mat{f}_{i} \mat{f}_j^\top \nonumber \\
\label{eq:latent-sim}
\end{align}


\begin{proposition}[]
	ILFM and IMMSB do not satisfy the latent homophily under the latent similarities respectively $s_l$.
\end{proposition}

\begin{proof}
We have that:
\begin{align}
&\pr(y_{ij}=1 \mid \M_e) \nonumber \\
&=  \mat{\hat{f}}_{i} \mat{\hat{\Phi}} \mat{\hat{f}}_j^\top \nonumber = \sum_{k,k'} f_{ik}\phi_{kk'}f_{jk'}   \nonumber \\
&= \sum_k f_{ik}\phi_{kk}f_{jk} + \sum_{k\neq k'} f_{ik}\phi_{kk'}f_{jk'} \nonumber
\end{align}

 Suppose now that we have a weight matrix with constant weights $\mu$. One can write:

\begin{equation}
    \pr(y_{ij}=1 \mid \mathcal{M}_e)= \mu (\sum_k f_{ik}f_{jk} + \sum_{k\neq k'} f_{ik}f_{jk'} ) \nonumber
\end{equation}

It is easy to find a counterexample where the similarity order is lost. A counter example is as follow, choose $f_i=f_j=(0,1,0)$ and $f_{i'}=(1,0,1)$ and $f_{j'}=(0,1,0)$. Thus,  we have $s(i,j)=1$ and $s(i',j')=0$, and suppose $\mu$ is equal to 1. We have  $p(y_{ij}) = 1$ but $p(y_{i'j'}) \propto 2$. We see that the similarity order does not preserve the likelihood. Thus, homophily is not satisfied. 
\end{proof} 



This proposition means that the latent features generated or learned for both model will not reflect the classical vision of homophily, according to which two individuals having similar features are more  likely to be connected. This proposition also highlights the fact that, in the general case,  the latent features of IMMSB and ILFM can not be interpreted out of the box as communities in the usual sense according to which  individuals having an identical membership have a high probability to be connected.

Nevertheless, it is worth to say this proposition doesn't means that models could not commply with the homophily effect. Rather, it says that to comply with the homophily effect one has  to relax the default iid assumptions over the weight of  $\Phi$. Typically if the models fix all the non-diagonal weight of $\Phi$ to zeros, links would only occur inside a block. In such a case, the latent features could be interpreted as communities indicators, in the classic sens of communities (high densities of links inside vs low densities outside ). This  specific case corresponds to an approach introduced to find overlapping communities within the MMSB models \cite{AMMSB}. The authors renamed the constraint MMSB to a-MMSB, standing for assortative MMSB. \textcolor{red}{I have the proof with the matrix normal as an example, but proof for a-MMSB is even simpler --- $\phi_{kk'}=0$ if $k\neq k'$.}

\subsection{Empirical Illustration}
\label{subsec:mg}

In order to illustrate our results, we provide a set of experiments where we use models to generate random networks and extract the underlying block structure. It shows that the block structure do not enforce an homophily effect. In other word, it do no enforce that links inside a block has a higher density than outside a block.

In figure \ref{fig:gen_blocks_mmsb}, we report the block structure  generated for 3 settings of IMMSB, (column 1:$\alpha=1, \gamma=2, \lambda_1=\lambda_2=0.5$ , column 2:$\alpha=0.1, \gamma=1, \lambda_1=5, \lambda_0=1$, column 3: $\alpha=1, \gamma=1, \lambda_1=\lambda_2=1$). Then for each of this settings we fix $N=100$ and generate full networks. To create the blocks structure, we choose to assign each nodes to a block by a max assignment of its latent features. It consist of selecting the block for a node which corresponds to its most representative feature.

%In figure \ref{fig:gen_blocks_ilfm}, with report similar experiment for ILFM with the 3 settings () and $N=100$. Representing the block structure for ILFM is more arbitrary, because of the hard assignment of nodes. One way to do it is to cluster the feature matrix F in a lower dimensional space to find a membership for each node. We thus perform a Kmeans clustering by setting the number of block equal to $K/3$ ($K$ being the dimension of the latent feature).

Given, the block structure, we reordered the adjacency matrices in a descending order of the size of blocks. It means that the block with the highest number of node is in the top left corner, and the lowest in the bottom right corner.

We also reported the block structure in a graph form where blocks size correspond to the number of nodes associated to it and the weighted edges between blocks reflects the strength of connections between blocks.

Note that did not report a block structure for ILFM. The reason is that such a structure is not well defined in this model because of the hard assignment of nodes and because feature are indistinguishable. The first fact is due to fact IBP draw binary feature (In our current settingds). The second is due to the fact latent feature are exchangeable in the IBP prior which provide no obvious way to assign a specific membership of any interactions. 

\begin{figure}[h]
	\centering
	
	\minipage{0.17\textwidth}
	\includegraphics[width=2.94cm, height=3cm]{img/M_g_peaks/figure_6}
	\endminipage
		\minipage{0.17\textwidth}
	\includegraphics[width=2.94cm, height=3cm]{img/M_g_power_law/figure_6}
	\endminipage
	\minipage{0.17\textwidth}
	\includegraphics[width=2.94cm, height=3cm]{img/M_g_regular/figure_6}
	\endminipage
	%\vspace{-0.3cm}
    \vspace{0.3cm}
	\minipage{0.16\textwidth}
	\includegraphics[width=3.5cm, height=4cm]{img/M_g_peaks/graph_dot}
	\endminipage
		\minipage{0.16\textwidth}
    \hspace{1cm}\includegraphics[width=3.0cm, height=4cm]{img/M_g_power_law/graph_dot} 
	\endminipage
	\minipage{0.16\textwidth}
	\includegraphics[width=3.5cm, height=4cm]{img/M_g_regular/graph_dot}
	\endminipage
	\caption{Block structure for IMMSB for 3 generated networks. The 3 columns represent 3 different settings. The first row represent the block structure draw on the adjacency matrix. The second row represent the graph of the underlying block structure.}
	\label{fig:gen_blocks_mmsb}
\end{figure}


 We now turn to the burstiness effect.
