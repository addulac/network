\section{Homophily: \emph{"Birds of a feather flock together"}}
\label{sec:homophily}
%\vspace{-0.2cm}
%\begin{center} \emph{Birds of a feather flock together} \end{center}
%\vspace{0.1cm}

Homophily refers to the tendency of individuals to connect to similar others: two individuals (and thus their corresponding nodes in a social network) are more likely to be connected if they share common characteristics~\cite{mcpherson2001birds,lazarsfeld1954friendship}. The characteristics often considered are inherent to the individuals: they may represent their social status, their preferences, their interest, ... A related notion is the one of {\it assortativity}, which is slightly more general as it applies to any network, and not just social networks, and refers to the tendency of nodes in networks to be connected to others that are similar in some way.

A definition of homophily has been proposed in~\cite{la2010randomization}. However, this definition, which relies on a single characteristic (as age or gender), does not allow one to assess whether latent models for link prediction capture the homophily effect or not. We thus introduce a new definition of homophily below, which directly aims at this:
%
\begin{definition}[Homophily]
	Let $\mathcal{M}$ be a link prediction model as defined above and $s$ a similarity measure between nodes. We say that \emph{$\mathcal{M}$ captures the homophily effect} iff, $\forall (i,j,i',j') \in V^4$:
%
\begin{equation}
s(i,j) > s(i',j')  \implies \pr(y_{ij}=1 \mid \mathcal{M}) > \pr(y_{i'j'}=1  \mid \mathcal{M}) \nonumber
\end{equation}
%
A model which verifies this condition is said to be \emph{homophilic} under the simylarity $s$.
\end{definition}
%
As one can note, this definition directly captures the effect "if two nodes are more similar, then they are more likely to be connected". The similarity function assesses to which extent two nodes share the same (latent) characteristics (or classes). In the case of $\mathcal{M}_e$, these characteristics are captured in the latent features $\mat{F}$ as the model implicitly tries to relate through $\mat{F}$ nodes that are connected in the observation. In the case of $\mathcal{M}_g$ however, there is no mechanism to define latent characteristics as $\mat{F}$ is solely defined from probability distributions that say nothing about possible shared characteristics of nodes. There is thus no sense in this case to talk about homophily. For this reason, we focus on $\mathcal{M}_e$ for homophily. 

The estimated matrix $\mat{\hat{F}}$ captures some latent characteristics of the nodes, whereas the estimated matrix $\mat{\hat{\Phi}}$ captures the correlations between these latent characteristics. One can thus define, on their basis, a "natural" similarity between nodes as follows:
%
\begin{equation}
s_n(i,j) = \mat{\hat{f}}_{i} \mat{\hat{\Phi}} \mat{\hat{f}}_j^\top \nonumber
%\label{eq:natural-sim}
\end{equation}
%
It is straightforward that $\mathcal{M}_e$ and $\M_g$ is homophilic wrt to $s_n$, as $sn(i,j)$ corresponds to the probability of generating a link between nodes $i$ and $j$ (Eq.   ~\ref{eq:link-me}). We thus have the following property:
%
\begin{proposition}[] $\M_e$ is homophilic wrt the natural similarities $s_n$.
\end{proposition}

The homophyly define by $s_n$ is trivial because there is a direct mapping between the node similarity and the links likelihood. Though, this proposition carry the idea that we can always choose a node similarity that would conserve the homophily effect but with a loss of interpretability of the similarity metric. An exemple of such conservation is any homothetic transformation of the natural similarity.

Therefore, A more interesting question would be to inspect the homophilic effect with a similarity that only depend os the latent features.Moreover is share characteristic of with the classical notion of homphily in networks. We define a latent similarity between nodes which is decorrelated of $\Phi$ (Note that $\Phi$ encode the metric to go from the feature space, to the probability space):

\begin{align}
&s_l(i,j) = \mat{f}_{i} \mat{f}_j^\top \nonumber \\
\label{eq:latent-sim}
\end{align}


\begin{proposition}[]
	ILFM is and IMMSB does not satisfy the latent homophily under the latent similarities respectively $s_l$.
\end{proposition}

\begin{proof}
We have that:
\begin{align}
&\pr(y_{ij}=1 \mid \M_e) \nonumber \\
&=  \mat{\hat{f}}_{i} \mat{\hat{\Phi}} \mat{\hat{f}}_j^\top \nonumber = \sum_{k,k'} f_{ik}\phi_{kk'}f_{jk'}   \nonumber \\
&= \sum_k f_{ik}\phi_{kk}f_{jk} + \sum_{k\neq k'} f_{ik}\phi_{kk'}f_{jk'} ) \nonumber
\end{align}

 Suposse now that we have a weight matrix with constant weights $\mu$. one can write:

\begin{equation}
\pr(y_{ij}=1 \mid \mathcal{M_e})= \mu (\sum_k f_{ik}f_{jk} + \sum_{k\neq k'} f_{ik}f_{jk'} ) \nonumber
\end{equation}

It is easy to find a counter example where the similarity order is lost.A counter example is as follow, chose $f_i=f_j=(0,1,0)$ and $f_{i'}=(1,0,1)$ and $f_{j'}=(0,1,0)$.Thus we have $s(i,j)=2$ and $s(i',j')=0$, and $p(i,j) \propto 2$ but $p(i',j') \propto 4$. We see that similarity order does not preserve the likelihood. Thus homophily is not satisfied. 
\end{proof} 



This proposition means that the latent features generated or learn for both model will not encode the classical vision of homophily, which says that if two individuals have similar feature, they will be likely to connect. This proposition also hilghight the fact the latent feature of iMMSB and ILFM can't be interpreted out of the box as communities where links occur more likely when two individuals has identical membership.

In the contrary, one can imagine that if $\Phi$ was constrained such that latent homophily is true, one can directly interpret the latent features as communities indicator. This is basically what is done is this paper \cite{AMMSB} to find overlapping communities within the MMSB models. The authors renamed the constraint MMSB to a-MMSB, standing for assortative MMSB. \textcolor{red}{I have the proof with the matrix normal as an example, but a-MMSB is even simpler --- $\phi_{kk'}=0$ if $k\neq k'$. if a possible reference, if needed.}

In figure \ref{fig:gen_blocks}, we show a clustering result form MMSB based on the blockmodels. We choose the cluster assignment for each node which represent the latent feature with highest probability. Thus it appears that MMSB capture the structural equivalence of the underlying networks.

\begin{figure}[h]
	\centering
	
	\minipage{0.17\textwidth}
	\includegraphics[width=2.94cm, height=3cm]{img/M_g_peaks/figure_6}
	\endminipage
		\minipage{0.17\textwidth}
	\includegraphics[width=2.94cm, height=3cm]{img/M_g_power_law/figure_6}
	\endminipage
	\minipage{0.17\textwidth}
	\includegraphics[width=2.94cm, height=3cm]{img/M_g_regular/figure_6}
	\endminipage
	%\vspace{-0.3cm}
    \vspace{0.3cm}
	\minipage{0.16\textwidth}
	\includegraphics[width=3.5cm, height=4cm]{img/M_g_peaks/graph_dot}
	\endminipage
		\minipage{0.16\textwidth}
	\includegraphics[width=3.5cm, height=4cm]{img/M_g_power_law/graph_dot} 
	\endminipage
	\minipage{0.16\textwidth}
	\includegraphics[width=3.5cm, height=4cm]{img/M_g_regular/graph_dot}
	\endminipage
	\caption{Tree examples of generated network in different setting of IMMSB. We reorder the adjacency matrix (white dot are links, block dot are non-link) in th upper figure, and plot their associated bockmodel structure in the lower part. The block structure is represented in the lower plots where we represent inter-block and intra-block connection by respectively edges ans self-loop.  }
	\label{fig:gen_blocks}
\end{figure}


 We now turn to the burstiness effect.
