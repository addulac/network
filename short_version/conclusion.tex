\section{Conclusion}
\label{sec:concl} 

%%% Conclusion on the paper
This work introduce a framework which aims at studying topological properties on social networks by giving definition which are consistent with the Bayesian framework. Using this framework, We show that strong relation exist between major relational latent model and fundamental properties of networks, the preferential attachment effect and the homophily effect.~\\ 

%%% Perspectives
This work offers many perspectives to improve our understanding of Bayesian relational models for complex networks. Obviously, our implementation based on MCMC inference does not allow us to scale to large networks. We will focus in the future in implementing variational inference scheme to be able to validate our theoretical results on much larger real networks. An other direction is to study the impact of hyperparameters on the structure of the generative network. Particularly a more well adapted optimization for nonparametric models could lead to a better control for capturing the preferential attachment effect. Especially by generalizing both the IBP and the HDP prior to their general counterpart, the 2-parameters IBP and the Pitman Yor Process. To conclude on perspective, it worth to mention that complex networks include other important properties of networks to study, such as the small world effect and the temporal dynamics of the network topology relaxing the exchangeability assumption.
