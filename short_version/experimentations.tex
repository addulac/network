\section{Illustration}
%\section{Experimentations}

our experimental platform that we have used for our experiments which is available online. Notes that it makes our experiments easily fully reproducible \footnote{https://github.com/dtrckd/pymake}.

To illustrate our theoretical results, we evaluate the predictive performance and the ability of the models to capture homophily and preferential attachment on artificial and real networks. For homophily, we simply compare the distributions of the natural and latent similarities on linked and non-linked pairs of nodes. For global preferential attachment, we plot (as done in \cite{???}) the degree distribution and its corresponding best fitted line in the log-log scale. In addition, we use the measure developed in \cite{clauset2009power} for assessing whether empirical data behaves according to a power law (as mentioned before, power laws are the standard bursty distributions in social networks \cite{barabasi1999emergence}). This framework combines maximum-likelihood methods with goodness-of-fit tests based on the Kolmogorov-Smirnov statistic to compute a $p$-value. If the obtained $p$-value is large (close to 1), then the data is likely to be distributed according to a power law and the associated network displays preferential attachment;  on the other hand, if it is small, the data is likely not distributed according to a power law and the associated network does not display preferential attachment.

For local preferential attachment, we follow the same approach as before to compute the $p$-value, the only difference being that the empirical data does not correspond any longer to the global adjacency matrix, but to reduced matrices for each class. The computation of the reduced adjacency matrices varies from one model to the other:
%
\begin{itemize}
    \item For IMMSB, for a given class $k$, the reduced adjacency matrix $Y^k$ is defined by: $y_{ij}^k=1$ if $y_{ij}=1, z_{i\rightarrow j}=z_{i\leftarrow j}=k$ and $0$ otherwise.
        \item For ILFM, the reduced adjacency matrix $Y^k$ is defined by:$ y_{ij}^k=1$ if $y_{ij}=1 , f_{ik}=f_{jk}=1$ and $0$ otherwise.
\end{itemize}
%

\subsection{Datasets and model parameters}

To illustrate the above developments, we consider two artificial and two real networks, the characteristics of which are summarized in Table~\ref{table:networks_measures}.

\begin{table}[h] 
	\centering
	\caption{Characteristics of artificial and real networks.}
	%\resizebox{\textwidth}{!}{  
    \begin{tabular}{lrrr}
        \hline
        \textbf{Networks} &   nodes &   edges &   density \\
        \hline
        Network1 &    1000 &    3507 &     0.007 \\
        Network2 &    1000 &   31000 &     0.062 \\
        Blogs         &    1490 &   20512 &     0.009 \\
        Manufacturing &     167 &    5950 &     0.215 \\
    \hline
    \end{tabular}
	\label{table:networks_measures}
\end{table}

The non-oriented artificial networks (Network1 and Network2) have been generated with the DANCer-Generator \cite{largeron2015}. This generator has been chosen because it allows one to build an attributed graph having a community structure as well as known properties of real-world networks such as preferential attachment and homophily. Moreover, by modifying the parameters, these properties can be weakened.

The first real network, denoted Blogs \footnote{available at: http://moreno.ss.uci.edu/data.html\#blogs}, contains front-page hyperlinks between blogs in the context of the 2004 US election. A node represents a blog and an edge represents a hyperlink between two blogs. The second one, denoted Manufacturing \footnote{available at: https://www.ii.pwr.edu.pl/~michalski/index.php?content=datasets\#manufacturing}, is an internal email communication network between employees of a mid-sized manufacturing company. Each node is associated to an employee and an oriented link represents an email sent between the two employees. One can notice that the second network is specific since it is an enterprise network in which the relationships between the employees are (professionally) constrained. This means that this network is less likely to display some of the properties that occur in unconstrianed social networks.

The adjacency matrices and global degree distributions of these networks are presented in Figure \ref{fig:corpuses}. The adjacency matrices enable us to visualize some characteristics of the networks such as their density and their clustering patterns: as one can note, the two artificial networks and Blogs have a clear community structure, corresponding to the blocks associated with the white dots, whereas Manufacturing does not have such a structure. The global degrees, ???

In Figure \ref{fig:synt_graph_local}, we represent the local degree distributions for all networks. Each curve within those plots represents a local degree distribution associated to the different classes. As the ground truth is not available for the real networks (Blogs and Manufacturing), classes have been determined with Louvain algorithm \cite{Blondel2008} and the local distribution defined according to the obtained classes.

The networks present different affinity with the preferential attachment effect.  As shown in Figure \ref{fig:corpuses}, Network1 and Blogs verify the  global preferential attachment whereas it is not the case for Network2 and Manufacturing.

The first section of Table \ref{table:me_gofit} (Training Datasets) reports the resulting $p$-values of the KS test as well as the values estimated for the parameters $\alpha$ in the global case (right) and in the local case (left).

The results confirm the previous analysis since the $p$-value is equal to 1 for Network1 and Manufacturing whereas it is null for Network2 and Manufacturing.

The local degree distributions per class, presented in Figure \ref{fig:synt_graph_local}, and the average results (with standard deviation) computed over the classes, for the $p$-value, are reported in Table \ref{table:me_gofit}. They lead to the same conclusion concerning the local preferential attachment which is well verified for Network1 and Blogs with a $p$-value equals to 1 but in a lesser extend for Network2 and Manufacturing with respectively a $p$-value equals to 0 and 0.4. Thus, for the preferential attachment, global and local, we can distinguish Network1 and Blogs which satisfy the preferential attachment and Network2 and Manufacturing which do not exhibit the property.


%\begin{figure}[h]
%    \centering
%
%    \advance\leftskip-3.5cm
%    \advance\rightskip-4cm
%	
%	\minipage{0.30\textwidth}
%	\includegraphics[scale=0.32]{img/g1}
%	\endminipage
%	\minipage{0.30\textwidth}
%	\includegraphics[scale=0.32]{img/g1_d}
%	\endminipage
%	\vspace{-0.4cm}
%	\minipage{0.30\textwidth}
%	\includegraphics[scale=0.32]{img/g2}
%	\endminipage
%	\minipage{0.30\textwidth}
%	\includegraphics[scale=0.32]{img/g2_d}
%	\endminipage
%
%	%\vspace{-0.4cm}
%
%	\minipage{0.30\textwidth}
%	\includegraphics[scale=0.32]{img/g3}
%	\endminipage
%	\minipage{0.30\textwidth}
%	\includegraphics[scale=0.32]{img/g3_d}
%	\endminipage
%	\vspace{-0.4cm}
%	\minipage{0.30\textwidth}
%	\includegraphics[scale=0.32]{img/g4}
%	\endminipage
%	\minipage{0.30\textwidth}
%	\includegraphics[scale=0.32]{img/g4_d}
%	\endminipage
%	
%	\caption{Adjacency matrices (left) and global degree distributions (right) for the artificial networks Network 1, ..,Network 4.}
%	\label{fig:synt_graph}
%\end{figure}
%
%\vspace{1cm}

\begin{figure}[h]
        \centering
        \begin{subfigure}[b]{0.480\textwidth}
            \centering
            \includegraphics[width=\textwidth]{img/corpus/network1_dd}
            \caption {{\small Network 1}}    
            \label{fig:mean and std of net14}
        \end{subfigure}
        \hfill
        \begin{subfigure}[b]{0.480\textwidth}  
            \centering 
            \includegraphics[width=\textwidth]{img/corpus/network2_dd}
            \caption {{\small Network 2}}    
            \label{fig:mean and std of net24}
        \end{subfigure}
        \vskip\baselineskip
        \begin{subfigure}[b]{0.480\textwidth}   
            \centering 
            \includegraphics[width=\textwidth]{img/corpus/network3_dd}
            \caption{{\small Network 3}}    
            \label{fig:mean and std of net34}
        \end{subfigure}
        \quad
        \begin{subfigure}[b]{0.480\textwidth}   
            \centering 
            \includegraphics[width=\textwidth]{img/corpus/network4_dd}
            \caption{{\small Network 4}}    
            \label{fig:mean and std of net44}
        \end{subfigure}
	\caption{Adjacency matrices (left) and global degree distributions (right) for the artificial networks Network 1, ..,Network 4.}
	\label{fig:synt_graph}
\end{figure}


\begin{figure}[h]
        \centering
        \begin{subfigure}[b]{0.480\textwidth}
            \centering
            \includegraphics[width=\textwidth]{\lpath/img/corpus/network1_1}
            \caption {{\small Network 1}}    
            \label{fig:mean and std of net14}
        \end{subfigure}
        \hfill
        \begin{subfigure}[b]{0.480\textwidth}  
            \centering 
            \includegraphics[width=\textwidth]{\lpath/img/corpus/network2_1}
            \caption {{\small Network 2}}    
            \label{fig:mean and std of net24}
        \end{subfigure}
        \vskip\baselineskip
        \begin{subfigure}[b]{0.480\textwidth}   
            \centering 
            \includegraphics[width=\textwidth]{\lpath/img/corpus/network3_1}
            \caption{{\small Network 3}}    
            \label{fig:mean and std of net34}
        \end{subfigure}
        \quad
        \begin{subfigure}[b]{0.480\textwidth}   
            \centering 
            \includegraphics[width=\textwidth]{\lpath/img/corpus/network4_1}
            \caption{{\small Network 4}}    
            \label{fig:mean and std of net44}
        \end{subfigure}
        \caption {Local degree distribution for the artificial networks. Inner(left) and outer(right) degree are separated.} 
	\label{fig:synt_graph_local}
\end{figure}




\subsection{Model evaluation}
For each dataset described earlier, we run a MCMC inference consisting of 200 iterations to learn the posterior distribution for the IMMSB and ILFM  models described previously. For IMMSB, the concentration parameters of HDP were optimized  using vague gamma priors $\alpha_0 \sim \text{Gamma}(1,1)$ and $\gamma \sim \text{Gamma}(1,1)$ following \cite{HDP}. The parameters for the matrix weights  $\lambda_0$ and $\lambda_1$ were fixed to 0.1. For ILFM, the weights hyperparameter  $\sigma_w$ was fixed to 1 and the IBP hyperparameter $\alpha$ to 0.5 in order to  have comparable number of classes with IMMSB.

This inference procedure was run ten times and the average values are reported as final results.


\paragraph{Homophily.} Figure \ref{fig:homo_mustach} presents boxplots describing the distributions of the similarity natural $s_n(i,j)$ and latent $s_l(i,j)$ computed respectively on linked and non-linked pairs of nodes for IMMSB (left) and ILFM (right). The results have been aggregated over the four datasets.  They confirms that the natural similarity is  higher for  pairs of nodes which are connected than between nodes which are not linked for both models. However, we can noticed that for IMMSB, the similarities computed on the non linked pairs are more concentrated around zero which provides an interesting insight of the bursty behavior of this model. For the latent similarity,  there is no difference between the linked and non-linked pairs, and consequently, we can not say that similar nodes are more likely to be connected. These experimental results are in accordance with our theoretical results presented in Section \ref{sec:homophily} which state that both ILFM and IMMSB are homophilic with respect to the natural similarity $s_n(i,j)$ and are not homophilic for the latent similarity $s_l(i,j)$.

\begin{figure}[h]
\centering
    \begin{minipage}{0.24\textwidth}
        \includegraphics[width=\textwidth]{img/corpus/homo_mustach_immsb}
    \end{minipage}
    \begin{minipage}{0.24\textwidth}
        \includegraphics[width=\textwidth]{img/corpus/homo_mustach_ilfm}
    \end{minipage}
    \caption{Boxplots of natural and latent similarities computed over the datasets respectively on the linked and non-linked pairs of nodes for IMMSB (left) and ILFM (right). }
    \label{fig:homo_mustach}
\end{figure}


\paragraph{Preferential attachment.} Table \ref{table:me_gofit} reports the value of the power-law goodness of fit for IMMSB and ILFM in the global case (left) and in the local case (right). It appears that for both models, the global preferential attachment is only verified for networks generated from datasets where the property was observed, namely in Network1 with p-value equals to 0.9 for IMMSB and 1 for ILFM, and in Blogs with a p-value equals to 1 for the both models whereas the property is not verified in Network2 and in Manufacturing, where p-values equal 0. This is in accordance with Proposition 2.1 according to which, both ILFM and IMMSB do not satisfy global preferential attachment. However, these models are able to capture this property if it exists in the training dataset.  Moreover, we can observe that, in the local case, IMMSB complies with the preferential attachment with $p$-values equals or close to 1 for the four networks while, ILFM obtained low p-values for the networks that were less locally bursty respectively equal to 0 and 0.3 for Network2 and Manufacturing. Also, the power-law coefficients $\alpha$ are significantly greater for IMMSB than for ILFM and specially for the bursty networks Network1 and Blogs.

Figure \ref{fig:me_local} illustrates the local preferential attachment by plotting the local degree distributions for Network1 (top) and Network2 (bottom) learned with IMMSB (left) and ILFM (right). The shape of local degree distributions appears more linear for IMMSB and with more fluctuations for ILFM. This illustrates the inability  of ILFM to capture local preferential attachment property,  as stated in Proposition 2.2. 

\begin{table}[t]
\caption{Preferential attachment measures for training datasets and networks generated with fitted models.}
\centering
\begin{tabular}{lrrrr}
  \multirow{2}{*}{\textbf{Training Datasets}}  &
  \multicolumn{2}{c}{Global} & \multicolumn{2}{c}{Local}\\
  \cmidrule(r){2-3} \cmidrule(l){4-5}
  &   $p$-value &   $\alpha$   & $p$-value & $\alpha$   \\
\hline
Network1       & 1 & 2.4 &   1.0 $\pm$ 0.0  &  1.8 $\pm$ 0.03  \\
Network2       & 0 & 1.3 &   0.0 $\pm$ 0.0  &  1.2 $\pm$ 0.01 \\
Blogs          & 1 & 1.5 &   1.0 $\pm$ 0.0  &  1.4 $\pm$ 0.03\\
Manufacturing  & 0 & 1.4 &   0.4 $\pm$ 0.3  &  1.3 $\pm$ 0.05 \\
\hline

  \ \textbf{IMMSB} &&&& \\
\hline
Network1       & 0.9 & 1.4 &   1.0 \(\pm\) 0.0   &  3.5 \(\pm\) 0.7 \\
Network2       & 0 & 1.3 &   0.9 \(\pm\) 0.0   &  1.6 \(\pm\) 0.2 \\
Blogs          & 1 & 1.3 &   1.0 \(\pm\) 0.0   &  4.3 \(\pm\) 1.1 \\
Manufacturing  & 0 & 1.2 &   0.9 \(\pm\) 0.01  &  1.6 \(\pm\) 0.1 \\
\hline

  \ \textbf{ILFM} &&&& \\
\hline
Network1      & 1 & 1.4 &   1.0 \(\pm\) 0.0  &  1.7 \(\pm\) 0.1 \\
Network2      & 0 & 1.2 &   0.0 \(\pm\) 0.0 &  1.2 \(\pm\) 0.0 \\
Blogs         & 1 & 1.3 &   0.9 \(\pm\) 0.2  &  1.5 \(\pm\) 0.1 \\
Manufacturing & 0 & 1.2 &   0.3 \(\pm\) 0.3  &  1.3 \(\pm\) 0.0 \\
\hline
\end{tabular}
\label{table:me_gofit}
\end{table}

\begin{figure}[h]
    \begin{minipage}{0.24\textwidth}
        \includegraphics[width=\textwidth]{img/corpus/immsb_network1_1}
    \end{minipage}
    \begin{minipage}{0.24\textwidth}
        \includegraphics[width=\textwidth]{img/corpus/ilfm_network1_1}
    \end{minipage}
    \vskip\baselineskip
    \begin{minipage}{0.24\textwidth}
        \includegraphics[width=\textwidth]{img/corpus/immsb_network2_1}
    \end{minipage}
    \begin{minipage}{0.24\textwidth}
        \includegraphics[width=\textwidth]{img/corpus/ilfm_network2_1}
    \end{minipage}
    \caption {Local degree distributions for Network1 (top row) and Network2 (bottom row) generated with fitted models IMMSB (first column) and ILFM (second column).} 
\label{fig:me_local}
\end{figure}

\begin{figure}[h]
\centering
    \begin{minipage}{0.24\textwidth}
        \includegraphics[width=\textwidth]{img/corpus/roc_network1_20_f}
    \end{minipage}
    \begin{minipage}{0.24\textwidth}
        \includegraphics[width=\textwidth]{img/corpus/roc_network2_20_f}
    \end{minipage}
    \begin{minipage}{0.4\textwidth}
        \includegraphics[width=\textwidth]{img/corpus/testset_max_20.png}
    \end{minipage}
    \caption{Top: AUC-ROC curves for Network1 (left) and Network2(right) with 20 percent of data used as learning set. Bottom: Relative performance of IMMSB and ILFM in function of percentage of data used as testing set.} 
\label{fig:auc}
\end{figure}


\paragraph{Prediction performance.}

Figure \ref{fig:auc} compares the performance  of the models on the different datasets in function of the training set size. In the bottom plot,  y-axis gives the relative performance defined as the difference of the AUC values obtained for IMMSB and ILFM: $AUC_{IMMSB} - AUC_{ILFM}$ whereas x-axis indicates the percentage of data (links) randomly removed from the datasets and  used as a testing set. Hence, the number of training data decreases with the x-axis and a positive value on the y-axis indicates that IMMSB outperforms ILFM.  Note that the relative performance corresponds to the difference of the MAX AUC values obtained for both models on the ten inference experiences.

The top plot illustrates a case where 20 percent of the data is used as testing set and where IMMSB dominates ILFM on Network1 (left) and the opposite for Network2 (right).

In general as shown in bottom plot, ILFM obtains better performance than IMMSB. However, the relative predictive performance of IMMSB  increases  when the quantity of training data decreases whereas for non-bursty networks the results are the opposite: the performance of ILFM increases when the size of the learning dataset decreases. This is particularly visible for Network2, more contrasted for Manufacturing, probably due to the small size of this last network which makes the prediction less challenging.


