\section{Feature Dynamic}

We defined the notion of latent features (or clases as terminologically equivalent) previously, where features define membership of nodes where links can occur inside (or outside, ie between class). 

%Hence one can have theintuition that if links are mostly created in a set of class (inner or outer), the shortest path between two random nodes is``proportional'' to the number of step needed to go through all classes $K$.

In a case of iMMSB, the number of class is unknown and evolves with the data through an HDP prior. The question that we ask here, is how this number evolve with the size of the network ie its number of vertices.

In the HDP, we have two step DP that form a so called Chinese
Restaurant Franchise (CRF) with infinite random component being:
\begin{itemize}
\item the number of
table in each restaurant, where restaurant symbolically correspond to
nodes and tables represent a couple of classes $(k,k')$ where a node generate some links (and non links) with other nodes. 
\item the
number of class in the franchise, where each table is associated to a couple of classes.
\end{itemize}

By definition of a \(DP(\alpha)\), we know that for each new draw given
\(i\) previous draws, the probability to generate a new group is equal
to \(\frac{\alpha}{\alpha +i-1}\) independently of the number of previous
component created. Moreover in the CRF, each table is drawn from a DP while the classes for each tables are drawn from one another DP (shared in all the franchise). Thus we can express the expectation of the number of tables ($\E[T|N, \alpha_0]$) and classes ($\E[K|T, \gamma]$). We start by the first quantity, for an undirected graph of size $N >> 0$:

\begin{align}
&\E[T|N, \alpha_0] = \sum_{j=1}^N\sum_{i=1}^{2N} \frac{\alpha_0}{\alpha_0 +i-1} \\
& \alpha_0 N (\Psi(\alpha_0+2N) - \Psi(\alpha_0))
\end{align}
Given the properties of the $\Psi$ function, we the asymptotic behavior of $\E[T|N]$ is :

\begin{equation} \label{eq:e_t}
\E[T|N, \alpha_0] \sim \alpha_0 N \log(\frac{2N}{\alpha_0}) \qquad \text{if} \quad N >> \alpha_0 >> 0
\end{equation}

Similarly, for the expectation of the number of class, we have:

\begin{equation}
\E[K|T, \gamma] = \sum_{i=1}^{T} \frac{\gamma}{\gamma +i-1} = \gamma (\Psi(\gamma+T) - \Psi(\gamma))
\end{equation}

One can remark, from equation \eqref{eq:e_t}, that $\lim_{N\to\infty} \E[T|N] \to  \infty$. Thus the expected limit for the number of cluster by chaining the two DP is (\textcolor{red}{Add the bigO to more precise on asymptotic development ?}):

\begin{equation}
\E[K] = \gamma \log(\frac{\alpha_0 N}{\gamma}\log(\frac{2N}{\alpha_0} )) \qquad \text{if} \quad N >> \gamma>>0
\end{equation}

We conclude that the latent feature has sub logarithmic increment whit the number of nodes with	 an HDP prior.

We show the empirical evolution of the number of class generated by the HDP in figure \ref{fig:gen_dyn}.

\begin{figure}[h]
	\centering
	
	\includegraphics[scale=0.4]{img/class_dynamics}
	%	\includegraphics[width=3.2cm, height=3.7cm]{img/class_dynamics}

	\caption{Each curve correspond to a different setting of hyperparameters for IMMSB. For each entry X-axis, we generate a full network with the number of node corresponding to it. Y-axis is the number of latent features to which HDP converged. Those figure illustrate the logarithm shape of the evolution of number of feature $K$ with the number of nodes $N$.}
	\label{fig:gen_dyn}
\end{figure}

