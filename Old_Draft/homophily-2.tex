\section{Homophily}
\label{sec:homophily}
\vspace{-0.2cm}
\begin{center} \emph{Birds of a feather flock together} \end{center}

\vspace{0.1cm}

Homophily refers to the tendency of individuals to connect to similar others: two individuals (and thus their corresponding nodes in a social network) are more likely to be connected if they share common characteristics~\cite{mcpherson2001birds,lazarsfeld1954friendship}. The characteristics often considered are inherent to the individuals: they may represent their social status, their preferences, their interest, ... A related notion is the one of {\it assortativity}, which is slightly more general as it applies to any network, and not just social networks, and refers to the tendency of nodes in networks to be connected to others that are similar in some way.

A definition of homophily has been proposed in~\cite{la2010randomization}. However, this definition, which relies on a single characteristic (as age or gender), does not allow one to assess whether latent models for link prediction capture the homophily effect or not. We thus introduce a new definition of homophily below, which directly aims at this:
%
\begin{definition}[Homophily]
	Let $\mathcal{M}=\{\mat{F},\mat{\Phi}\}$ be a link prediction model as defined above and $s$ a similarity measure between nodes. We say that \emph{$\mathcal{M}$ captures the homophily effect} iff, $\forall (i,j,i',j') \in V^4$:
%
\begin{equation}
s(i,j) > s(i',j')  \implies \pr(y_{ij}=1 \mid \mathcal{M}) > \pr(y_{i'j'}=1  \mid \mathcal{M}) \nonumber
\end{equation}
%
A model which verifies this condition is said to be \emph{strongly homophilic}.
\end{definition}
%
As one can note, this definition directly captures the effect "if two nodes are more similar, then they are more likely to be connected". The model $\mathcal{M}$ considered can either be regarded in a purely generative setting, or be learned from the data. The above definition encompasses both cases, the parameters $\mat{F}$ and $\mat{\Phi}$ being either generated or learned. Furthermore, in some cases, for example when one has access to users' interest or social information, one may use \textit{a priori} representations of nodes as well as given weight matrix between them. This situation corresponds to the case where the matrices $\mat{F}$ and $\mat{\Phi}$ are given, the variables $y_{ij}$ still being generated according to Eqs.~\ref{eq:link-ilfm} and \ref{eq:link-immsb}. The definition of homophily given above is still applicable in this setting.

The similarity function assesses to which extent two nodes share the same characteristics. The parameters $\mat{F}$ and $\mat{\Phi}$ respectively capture latent characteristics of nodes and weights (or correlations) between the dimensions of these representations; a natural similarity between node characteristics is thus provided by:
%
\begin{equation}
s(i,j) = \mat{f}_{i} \mat{\Phi} \mat{f}_j^\top
\label{eq:natural-sim}
\end{equation}
%
From this, one can state the following property:
%
\begin{proposition}[]
ILFM is strongly homophilic under the similarity measure defined by Eq.~\ref{eq:natural-sim}.
\end{proposition}
%
\noindent The proof of this proposition is straightforward, and directly stems from the fact that the sigmoid function is strictly increasing.

The situation for IMMSB is slightly different inasmuch as the selection of classes from the latent class representations may lead to reversing the order between two pairs of nodes for the similarity and the probability of connecting the nodes. Indeed, IMMSB is \textit{not} strongly homophilic. This said, IMMSB still captures an homophily effect in the following sense:
%
\begin{proposition}[]
Let $s$ be the similarity measure defined by Eq.~\ref{eq:natural-sim}. IMMSB is \emph{weakly homophilic}, \textit{i.e.}  $\forall (i,j,i',j') \in V^4$:
%
\begin{equation}
\begin{array}{l}
s(i,j) > s(i',j')  \implies \nonumber \\
\mathbb{E}_{\pr(z_{ij},z_{ji}|\mathcal{M})}[\pr(y_{ij}=1 \mid \mathcal{M})] > \mathbb{E}_{\pr(z_{ij},z_{ji}|\mathcal{M})}[\pr(y_{i'j'}=1  \mid \mathcal{M})] \nonumber
\end{array}
\end{equation}
\end{proposition}
%
\noindent The proof of this proposition is also straightforward and stems from the fact that:
%
\begin{equation}
\begin{array}{l}
\mathbb{E}_{\pr(z_{ij},z_{ji}|\mathcal{M})}[\pr(y_{ij}=1 \mid \mathcal{M})]\nonumber \\
= \sum_{k,k'} \phi_{k,k'} \pr(z_{ij}=k|\mathcal{M}) \pr(z_{ji}=k'|\mathcal{M}) \nonumber \\
= \sum_{k,k'} \phi_{k,k'} f_{ik} f_{jk'} = \mat{f}_i \mat{\Phi} \mat{f}_j^\top \nonumber
\end{array}
\end{equation}
%
where $f_{ik}$ denotes, as before, the element of $\mat{F}$ at the $i^{th}$ row and $k^{th}$ column. 

The above development shows that both ILFM and IMMSB can capture the homophily effect, with a strict adherence to it in the case of ILFM, and a looser one for IMMSB. We now turn to the burstiness effect.