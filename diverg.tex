\documentclass[]{article}
\usepackage{lmodern}
\usepackage{amssymb,amsmath}
\usepackage{ifxetex,ifluatex}
\usepackage{fixltx2e} % provides \textsubscript
\ifnum 0\ifxetex 1\fi\ifluatex 1\fi=0 % if pdftex
  \usepackage[T1]{fontenc}
  \usepackage[utf8]{inputenc}
\else % if luatex or xelatex
  \ifxetex
    \usepackage{mathspec}
  \else
    \usepackage{fontspec}
  \fi
  \defaultfontfeatures{Ligatures=TeX,Scale=MatchLowercase}
\fi
% use upquote if available, for straight quotes in verbatim environments
\IfFileExists{upquote.sty}{\usepackage{upquote}}{}
% use microtype if available
\IfFileExists{microtype.sty}{%
\usepackage{microtype}
\UseMicrotypeSet[protrusion]{basicmath} % disable protrusion for tt fonts
}{}
\usepackage[unicode=true]{hyperref}
\hypersetup{
            pdfborder={0 0 0},
            breaklinks=true}
\urlstyle{same}  % don't use monospace font for urls
\IfFileExists{parskip.sty}{%
\usepackage{parskip}
}{% else
\setlength{\parindent}{0pt}
\setlength{\parskip}{6pt plus 2pt minus 1pt}
}
\setlength{\emergencystretch}{3em}  % prevent overfull lines
\providecommand{\tightlist}{%
  \setlength{\itemsep}{0pt}\setlength{\parskip}{0pt}}
\setcounter{secnumdepth}{0}
% Redefines (sub)paragraphs to behave more like sections
\ifx\paragraph\undefined\else
\let\oldparagraph\paragraph
\renewcommand{\paragraph}[1]{\oldparagraph{#1}\mbox{}}
\fi
\ifx\subparagraph\undefined\else
\let\oldsubparagraph\subparagraph
\renewcommand{\subparagraph}[1]{\oldsubparagraph{#1}\mbox{}}
\fi

\date{}

\begin{document}

\section{Discussion sur les modéles
Baysien}\label{discussion-sur-les-moduxe9les-baysien}

Je ressent le besoin de vous ecrire un peu plus en details, mon point de
vue, et de le confronter à celui d'Eric, par rapport au modèles Bayesien
et notamment au concept \(M_e\) et \(M_g\). Le but de papier est de
trouver un terrain d'entente et sur le formalisme et le discours, car
sans cela, nous perdons beaucoup de temps a rediscuter cela, et avec un
impact aussi sur la cohésion du papier que l'on écrit . Et je vais
tenter de vous exposer pourquoi.

....[[Draft]]

Tout d'abord spécifions les hypothèses de base de notre \emph{système d'apprentissage} Bayesien.

On définie une structure aléatoire, qui est notre source d'apprentissage comme une variable aléatoire $X$ sur un espace mesurable. Le theoreme de représenation [ref Otz] tel que le theorems De Finetti pour les séqences de variable $(X_1, X_2,...)$ écheangle, le theorem de Haldous-Hoover pour les graph echeangable, et plus généralement que si une distribution sur structure $X$ dans a un groupe de symmetrie $G$, alors elle a une représentation de la forme

\begin{equation}
    P(X) = \int_T
\end{equation}

Sur les distribution ergodic.

Donc ce theroem définie une distribution sur les observation, communément appelé le likelihood, et une mesure sur les paramétre $\Theta$, qui represente le prior sur l'espace des observations. 

L'application du theoreme de baye consiste, à evaluer la distribution des paramètres aleatoires selon les observations $X$. 

L'inference statistique ou "l'apprentissage consiste à extraction de l'information contenue dans $\Theta$, (link to kulklback leibler divergence. minimisation and variational inference). Et l'eloignement de l'aléaoire contenue dans le likelihood (Minimisation de l'entropie du likelihood)

\end{document}
