\section{Burstiness}
\emph{The rich get richer and the poor get poorer} ~\\

%\paragraph{Formal Definition}~\\

The preferential attachment states that a node is more likely to create connections with nodes having a high degree. To take into account this behavior, in the BarabasiAlbert (BA) model, each node is connected to an existing node with a probability proportional to the number of links of the chosen node. This leads to scale-free networks, characterized by a degree distribution with a heavy tail which can be approximated by a power law distribution such that the fraction of nodes $\pr(d)$ having a degree $d$ follows $\pr(d) \sim d^{-\gamma}$ where $\gamma$ ranges typically between 2 and 3~\cite{barabasi1999emergence}. An equivalent notion is the burstiness, studied by~\cite{church1995poisson}, which conveys the same idea : rich get richer or the more you have, the more you will get. In~\cite{clinchant2008bnb}, a formalized definition has been proposed. According to the authors:

\begin{definition}[Burstiness]
	A discrete distribution $\pr$ is bursty if and only if for all integers $(n, n')$, $n \geq n'$ :
	\begin{equation}
	\pr(d \geq n'+1 \mid d \geq n') > \pr(d \geq n+1 \mid d \geq n) 
	\end{equation}
	 A distribution which verifies this condition is said to be bursty.
\end{definition}

In~\cite{clinchant2010information}, this definition has been generalized to the continuous case but, in the sequel, we will retain this first definition since we focus on discrete distributions.~\\

The burstiness can appear for different various variable in a model. In this paper we consider three different schemes at the node and feature level, that constitute some basic assumptions on networks:

\begin{proposition}~\\
for all $i,j \in V^2$ and $k \in \{1,.., K\}$, we have:
\begin{itemize}
	\item Preferential Attachment: the distribution of degree $d_i$ is bursty iff $f_i$ is a stricly increasing function of n with: $$f_i(n)=\pr(y_{ij}=1 \mid d_i=n)$$
	\item Local Preferential Attachment: Given a class couple $c= \{k, k'\} \in \{1,..,K\}^2$, and a degree restricted to nodes who belongs to this interaction couple $d_{ic}$, the distribution of degree $d_{ic}$ is bursty iff $f_{i,c}$ is a stricly increasing function of n with: $$f_{i,c}(n)=\pr(y_{ij}=1 \mid d_i=n, c)$$
	\item Feature burstiness (block/class burstiness ?:s): the distribution over the number of membership of each class is bursty iff  $f_k$  is a stricly increasing function of n with: $$f_k(n) = \pr(\theta_{ik} \mid \theta_{\bm{.}k}^{-ik}=n)$$ 
\end{itemize}
\end{proposition}

We justify this approach in the supplementary materials~\ref{burst_proof}. One can see that the approach makes the link with the classical definition of preferential attachment. Furthermore one can see that the similaraty between th functions we track $(f_i, f_{i,c}, f_k)$ and the typical Gibbs updates. The difference is that we want to assess the generative model given the data and parameters of the model (ie the model has converged). Hence we are looking if the topological property of burstiness can be handle by the model once it learned from the data.~\\


%Considering the nodes projection on latent variables, the projections for the nodes we evaluate might be considered either known or not known ($\Theta$ or $\Theta^{-ij}$) depending on the case study of preferential attachment or local preferential attachment.~\\

\subsection{Burstiness for ILFM}

In this model, the weight interaction matrix $\Phi$ are not conjugate of the likelihood. Thus it can not be integrated out into a closed form expression. As a matter of simplicity we consider this parameter as known, and omit it the following conditional distributions.

Let $F=\Theta$ and $W=\Phi$, each node $i$ has a fixed feature vector noted $F_i$ and a weighed interactions matrix $W$. In this case, the function $f_i$ is:
\begin{align}
&\pr(y_{ij}=1 \mid d_i, F^{-i\bm{.}}) = \sum_{F_i} \pr(y_{ij}=1 \mid d_i, F^{-i\bm{.}} F_i,) \pr(F_i \mid d_i, F^{-i\bm{.}}) \\
&= \sum_{F_i} \sigma(F_iWF_j^T) \frac{\pr(d_i \mid F^{-i\bm{.}}, F_i) \pr(F_i\mid F^{-i\bm{.}}) }{\pr(d_i\mid F^{-i\bm{.}})}\\
&\propto \sum_{F_i} \prod_{j' \in \V(i) \cup {j}} \sigma(F_iWF_{j'}^T)\prod_{j' \notin \V(i)} 1-\sigma(F_iWF_{j'}^T) \prod_k \frac{m_{ik}}{N}
\end{align}

The term $\prod_k \frac{m_{ik}}{N}$ latter equation comes from the conditional probability of a feature $f_{ik}$ for an IBP prior and applying a chain of product rule. The product is the only term that depend on $d_i$ and we refer to it as $f(d_i)$.
Under the assumption that all observed links have higher probability than the observed non-links to bind, we can choose an index dictionary $g$ to reorder the terms such as:
\begin{equation}
\underbrace{\sigma(F_iWF_{g(1)}^T) \geq .. \geq\sigma(F_iWF_{g(p)}^T) }_\text{$g(.) \in \V(i) \cup {j}$ }\geq \underbrace{ .. \geq \sigma(F_iWF_{g(N)}^T)}_\text{$g(.) \notin \V(i)$}
\end{equation}
We then have with some regularity:

\begin{align}
&f(d_i)\geq \sigma(F_iWF_{g(p)}^T)^{d_i +1} (1 - sigma(F_iWF_{g(p+1)}^T))^{N - d_i} \\
&log(f(d_i)) \geq d_i log(\frac{\sigma(F_iWF_{g(p)}^T)}{1-  \sigma(F_iWF_{g(p+1)}^T)}) + cst
\end{align}


Reste a valider 2 point:
\begin{itemize}
	\item Le passage à la proportion
	\item Borner $log(f(d_i))$ entre deux fonction croissantes et montrer qu'on oscille pas à l'intérieur ?!
\end{itemize}

Then a sufficient condition for the burstiness is to have: $\sigma(F_iWF_{g(p)}^T) > 1- \sigma(F_iWF_{g(p+1)}^T)$. If $\sigma$ is the sigmoid this is equivalent ot have $F_iWF_{g(p)}^T > - F_iWF_{g(p+1)}^T$. 

In other term to enable burstiness in the ILFM, the model need to ensure some deterministic behavior when regarding the realization of outcomes and there actual distribution.
%The probability that a matrix $F$ is generated from the IBP is ~\cite{IBP}:
%\begin{equation}
%P(F \mid \alpha) = \frac{\alpha^{K_+}}{\prod_{i=1}^N K_1^{(i)} } \exp(-\alpha H_N) \prod_{k=1}^{K_+} \frac{(N - m_k)!(m_k - 1)!}{N!}
%\end{equation}

\subsection{Burstiness for MMSB}

\label{burst_class}

In the latent class models each  dyads has two underlying  class assignments for each node of the couple. We note $Z \in N\times N\times 2$ the matrix that represents those class assignments.
We seek for the following form of the likelihood, that we marginalize over all the possible couples classes $c=(k, k')$:
\begin{equation} \label{eq:q}
\pr(y_{ij}=1 \mid Y^{-i\bm{.}}, Z^{-ij}, d_i ) = \sum_{c=(k, k')} \pr(y_{ij}=1 \mid Y^{-i\bm{.}}, d_i , c) \pr(c \mid  Z^{-ij} ) 
\end{equation}
Here note that within the sum, the left hand term is conditionally independent of $Z^{-ij}$. And the right hand term is independent of the adjacency terms $Y^{-i\bm{.}}$ since it do not belongs to the Markov blanket of $c$ random variable.

The first term is the likelihood for the links between $(i,j)$ given the class of each node (k, k'). Due to the Beta-Bernoulli conjugacy of the model, $\phi$ and $\theta$ can be marginalized out, and it simplify to: 
\begin{equation} \label{eq:qc}
\pr(y_{ij}=1 \mid Y^{-i\bm{.}}, d_i , c) = \frac{C_{c1}^{-i.} + d_{ic} + \lambda_1}{C_{c\bm{.}}^{-ij} + \lambda_0+\lambda_1} 
\end{equation}
Where $C_{c1}$ denotes the count matrix for all interactions having value 1 (link present) with the classes couple being $c=(k, k')$. Thus $C_{c1} = \sum_{i,j} \bm{1}(z_{i\rightarrow j}=k, z_{i\leftarrow j}=k', y_{ij}=1)$ and $C_{c.} = \sum_{i,j} \bm{1}(z_{i\rightarrow j}=k, z_{i\leftarrow j}=k')$\\

We recognize the likelihood form of the Gibbs update~\cite{HDP}, except that we isolate the term depending of the degree on $i$, $d_i$. Hence the term $d_{ic}$ is the element of the degree with a classes couple $c=(k,k')$ and $d_{ic} = \sum_{j' \neq j} \bm{1}(z_{i\rightarrow j'}=k, z_{i\leftarrow j'}=k', y_{ij'}=1) $.\\

The second term of equation \eqref{eq:q}, can be rewrited by noting that the classes of the couple $c$ are independent and that the term $Y^{-j\bm{.}}$ can be dropped because it is not present in the Markov blanket of the class assignment: 
\begin{equation} \label{eq:topic_assign}
\pr(c \mid  Z^{-ij} ) =  \pr(z_{i\rightarrow j}=k \mid Z^{-ij}) \pr(z_{i\leftarrow j}=k' \mid Z^{-ij})
\end{equation}

Again, the two members of the right hand equation \eqref{eq:topic_assign} are the Gibbs updates for the topic assignments of nodes for the interaction $(i,j)$. Both members reduce to simple form due to the conjugacy between the Dirichlet and Multinomial~\cite{DM} or concurrently from the Chinese Restaurant Franchise~\cite{HDP}:
\begin{align}
\pr(z_{i\rightarrow j}=k \mid Z^{-ij}) = \frac{N_{ik}^{-ij} + \alpha_k}{ N_{i.}^{-ij} + \alpha_{\bm{.}} } \\
\pr(z_{i\leftarrow j}=k' \mid Z^{-ij}) = \frac{N_{jk'}^{-ij} + \alpha_{k'}}{ N_{j\bm{.}}^{-ij} + \alpha_{\bm{.}} } 
\end{align}

Finally, one can see that the only term depending on the degree $d_i$ is isolated, and we can rewrite equation \eqref{eq:q}, with term depending only on $d_i$, $k$, $i$ and $j$:
\begin{equation}
\pr(y_{ij}=1 \mid Y^{-i\bm{.}}, Z^{-ij}, d_i ) = \sum_{c=(k, k')} A_c (B_c + d_{ic})
\end{equation}
Where $A_c$ and $B_c$ are two positive function of $c$.
\begin{align}
A_c &= \frac{N_{ik}^{-ij} + \alpha_k}{ N_{i.}^{-ij} + \alpha_{\bm{.}} } \frac{N_{jk'}^{-ij} + \alpha_{k'}}{ N_{j\bm{.}}^{-ij} + \alpha_{\bm{.}} } \frac{1}{C_{c\bm{.}}^{-ij} + \lambda_0+\lambda_1} \\
B_c &= C_{c1}^{-i.} + \lambda_1
\end{align}

As we sum over all possible couple classes, the probability to have a link will augment with the degree with the classes couple corresponding to the element of the degree with the same couple. Hence the probability to observe a link for node $i$ is strictly crescent with his degree $d_i$. 

\paragraph{Preferential Attachment}~\\

The model is bursty hence it can handle the preferential attachment at the network level.

\paragraph{Local Preferential Attachment}~\\

The Local preferential attachment is similar to the notion of burstiness but inside a community/class of the network. Assuming that we know the class of $i$ $z_{i\rightarrow j}$ to be $k$, the probability to have a link becomes: 
\begin{align}
&\pr(y_{ij}=1 \mid Y^{-i\bm{.}}, d_i,z_{i\rightarrow j})  = \sum_{k'} \pr(y_{ij}=1 \mid Y^{-i\bm{.}}, Z^{-ij}, d_i , c=(k,k')) \pr(z_{i\leftarrow j}=k' \mid Z^{-ij} ) \\
&= \sum_{k'} \frac{C_{c1}^{-i.} + d_{ic} + \lambda_1}{C_{c\bm{.}}^{-ij} + \lambda_0 + \lambda_1} \frac{N_{jk'}^{-ij} + \alpha_{k'}}{ N_{j\bm{.}}^{-ij} + \alpha_{\bm{.}} } \\
&= \sum_{k'} A'_{k'} (B'_{k'} + d_{i(k,k')})
\end{align}

Here the probability increases with the degree independently of the interactions classes. This means that burstiness is possible inside but also between communities.

\paragraph{Communities Distribution}~\\

....Need to count the table for each classes in Chinese Restaurant Franchise (CRF), to evaluate the distribution according to the hyperprior of HDP...



