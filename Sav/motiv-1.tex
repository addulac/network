\section{Motivations}
Recently, several complex Bayesian models based on latent variables to explain the structure of social networks have been introduced [mmsb, ilfrm, etc]. This work was mainly evaluated on prediction tasks, such as link prediction or communities detection. However, few works have been done concerning the study of the intrinsic capacity of the models to model basic properties that arise in social networks, such as the dynamics of degree distribution, known to exhibit the preferential attachment effect [barabasi, web..] or the homophily effect[ref].
% For exemple, the heavily study Latent Dirichlet Allocation Model LDA model, being a particular of Mixed Membership Stochastic Blockmodel (MMSB) for networks, made no epistomological claim about the conjugacy used. In this work we found that conjugacy played a role in the ability of the model to capture some properties.
~\\


(++ Indeed the most heavily studied properties in social networks was the degree distribution and the mixing pattern (homophily/assortativity) tableaux !)

(++ not clear consensus of the formalism of properties and their evaluation, and whatsoever for the homophily property, the feature the definition are usually for single attribute... We consider a general vector . (with a measure working for both latent and real features)

(++ Probabilistic models we are interested in provide two ways of representing the data or network. One fall in the paradigm of mixture models and the other in the latent feature modeling. A motivation of those two modeling paradigm is that they are consistent with two key nonparametric prior for discrete data, namely the Dirichlet process (DP) and the the Indian Buffet Process (IBP). Many baysian model can be view as equivalent to truncated models with nonparametric priors. This provide a motivation to study those models. Furthermore, they are used as priors to generate latent features, either as proposition vector (class/DP) or binary vector (feature/IBP). It is admitted that those priors gives bursty features [accounting for burstiness in topic model]. We seek to clarify why this is true and how the burstiness can propagate at the degree level.~\\


In the next section we will, first, explain the mathematical background in a machine learning context. Secondly, we will review the models of interest for dyadic data. Then, we will introduce the formal definition of properties of interest in social networks within the Bayesian frameworks, and how this is translated in terms of assumptions within Bayesian priors. Finally, we will show empirical results (on synthetic and real datasets) to support our claims.~\\

%Study the poisson binomial distribution for the out links of a node. This is the degree distribution and should be bursty to have commmunity in networks.

