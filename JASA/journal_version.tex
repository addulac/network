%%&latex
\documentclass[12pt]{article}
\usepackage{graphicx,psfrag,epsf}
\usepackage{graphicx}
\usepackage{natbib}

%\pdfminorversion=4
% NOTE: To produce blinded version, replace "0" with "1" below.
\newcommand{\blind}{0}

\setlength{\textheight}{24 cm}

% DON'T change margins - should be 1 inch all around.
\addtolength{\oddsidemargin}{-.5in}%
\addtolength{\evensidemargin}{-.5in}%
\addtolength{\textwidth}{1in}%
\addtolength{\textheight}{-.3in}%
\addtolength{\topmargin}{-.8in}%



\newcommand*{\lpath}{./}%
%\usepackage[cmex10]{amsmath, mathtools}
%\usepackage{amsmath,amssymb,amsbsy,amsfonts,amsthm}
\usepackage{amsmath,amssymb,amsbsy,amsfonts}
\usepackage{mathtools}
\usepackage{booktabs}
\usepackage{multirow}
\usepackage{bm}
\usepackage{enumerate}
\usepackage{url}
%\usepackage[ruled,vlined]{algorithm2e}
\usepackage{fancyvrb}
\usepackage{yfonts}
\usepackage{wrapfig}
\usepackage{subfigure}
\usepackage{tikz}
\usetikzlibrary{bayesnet}
\newcommand{\tikzmark}[1]{\tikz[overlay,remember picture] \node (#1) {};}
\usepackage{calc}%    For the \widthof macro
\usepackage{xparse}%  For \NewDocumentCommand

%\usepackage[notes]{biblatex-chicago}
%\usepackage[authordate-trad,backend=biber]{biblatex-chicago}
%\bibliographystyle{chicago}
%\usepackage[style=chicago-notes]{biblatex}
%\bibliography{./a}

%% Variable de compilation
\newif\ifbeamer
\beamerfalse
\newcommand{\beamer}[2]{\ifbeamer #1 \else #2 \fi}
%%%

%\usepackage[latin1]{inputenc}
\usepackage[utf8]{inputenc} % manage utf8 encodage 
%\usepackage[english]{babel} % for french document ! dirty enumerate style,+ bad change rectangle colors for section linking.
\usepackage{fancyhdr} % for heading
\usepackage{listings}
\usepackage[colorlinks=true, urlcolor=blue]{hyperref} % url, link
\usepackage{graphicx}
\usepackage{geometry}

%\usepackage[cmex10]{amsmath, mathtools}
\usepackage{amsmath,amssymb,amsbsy,amsfonts,amsthm}
\usepackage{multirow}
\usepackage{bm}
\usepackage{enumerate}
\usepackage{url}
\usepackage[ruled,vlined]{algorithm2e}
\usepackage{fancyvrb}
\usepackage{yfonts}

\usepackage{wrapfig}
\usepackage{tikz}
    %\input{../tikz.conf}
    
\usetikzlibrary{bayesnet}
    
%%%%%%%%%%% Box 
\usepackage{calc}%    For the \widthof macro
\usepackage{xparse}%  For \NewDocumentCommand
\newcommand{\tikzmark}[1]{\tikz[overlay,remember picture] \node (#1) {};}

%%%%%%%%%% Math
\renewcommand{\text}{\textnormal}
\newcommand{\pr}{\mathbf{p}}
\newcommand{\E}{\mathbb{E}}
\newcommand{\divkk}{\mathbb{K}}
\newcommand{\entropy}{\mathbb{H}}
\newcommand{\gem}{\mathrm{GEM}}
\newcommand{\Mult}{\mathrm{Mult}}
\newcommand{\DP}{\mathrm{DP}}
\newcommand{\IBP}{\mathrm{IBP}}
\newcommand{\M}{\mathcal{M}}
\newcommand{\V}{\mathcal{V}}
\newcommand{\N}{\mathcal{N}}
    
\makeatletter
\NewDocumentCommand{\DrawBox}{s O{}}{%
    \tikz[overlay,remember picture]{
    	\IfBooleanTF{#1}{%
    		\coordinate (RightPoint) at ($(left |- right)+(\linewidth-\labelsep-\labelwidth,0.0)$);
    	}{%
    	\coordinate (RightPoint) at (right.east);
    }%
    \draw[red,#2]
    ($(left)+(-0.2em,0.9em)$) rectangle
    ($(RightPoint)+(0.2em,-0.3em)$);}
}

\NewDocumentCommand{\DrawBoxWide}{s O{}}{%
	\tikz[overlay,remember picture]{
		\IfBooleanTF{#1}{%
			\coordinate (RightPoint) at ($(left |- right)+(\linewidth-\labelsep-\labelwidth,0.0)$);
		}{%
		\coordinate (RightPoint) at (right.east);
	}%
	\draw[red,#2]
	($(left)+(-\labelwidth,0.9em)$) rectangle
	($(RightPoint)+(0.2em,-0.3em)$);}
}
\makeatother
%%%%% ! Box

\geometry{
      a4paper,
	    body={160mm,260mm},
	    left=25mm,top=20mm,
	    headheight=4mm,headsep=8mm,
        footskip=10mm,
        }
                                              

%%%%%%%%%%%%%%%%%%%%%%%%%%%%%%%%%%%%%%%%%%%%%%%%%%%%%%%%%%%%%%%%%%%%%%%%%%%%%%%%%%%%%%%%%%%%%%%%%%%%%%
%%%%% => Internal
%%%%%%%%%%%%%%%%%%%%%%%%%%%%%%%%%%%%%%%%%%%%%%%%%%%%%%%%%%%%%%%%%%%%%%%%%%%%%%%%%%%%%%%%%%%%%%%%%%%%%%

% itemize item def
%% \begin{itemize}\itemsep2pt % example space betwew item
%\renewcommand{\FrenchLabelItem}{\textbullet}
\renewcommand{\labelitemi}{$\bullet$}
\renewcommand{\labelitemii}{$\cdot$}
\renewcommand{\labelitemiii}{$\diamond$}
\renewcommand{\labelitemiv}{$\ast$}

% equation reference
\renewcommand{\theequation}{\thesection.\arabic{equation}}

%%%%%%%%%%%%%%%%%%%%%%%%%%%%%%%%%%%%%%%%%%%%%%%%%%%%%%%%%%%%%%%%%%%%%%%%%%%%%%%%%%%%%%%%%%%%%%%%%%%%%%
%%%%% => Alias
%%%%%%%%%%%%%%%%%%%%%%%%%%%%%%%%%%%%%%%%%%%%%%%%%%%%%%%%%%%%%%%%%%%%%%%%%%%%%%%%%%%%%%%%%%%%%%%%%%%%%%

% write code
\lstnewenvironment{C}[1]
{\lstset{language=C,
      frame=tBRl,
      basicstyle=\scriptsize,stringstyle=\emph,showstringspaces=false,
      numbers=left,numberstyle=\tiny,
      breaklines=true, columns=flexible, title={#1}}
}{}
      
%%%%%%%%%%%%%%%%%%%%%%%%%%%%%%%%%%%%%%%%%%%%%%%%%%%%%%%%%%%%%%%%%%%%%%%%%%%%%%%%%%%%%%%%%%%%%%%%%%%%%%
%%%%% => Preambles Pages
%%%%%%%%%%%%%%%%%%%%%%%%%%%%%%%%%%%%%%%%%%%%%%%%%%%%%%%%%%%%%%%%%%%%%%%%%%%%%%%%%%%%%%%%%%%%%%%%%%%%%%

\pagestyle{fancy}
\fancyhf{} % remove default headers
\fancyfoot[C]{\thepage}
\renewcommand{\footrulewidth}{0.3pt}
\renewcommand{\headrulewidth}{0.3pt}






%%%%%%%%%% Math
\renewcommand{\text}{\textnormal}
\newcommand{\ifm}{\texttt{ILFM}}
\newcommand{\ilfm}{\texttt{ILFM}}
\newcommand{\imb}{\texttt{IMMSB}}
\newcommand{\pr}{P}
\newcommand{\p}{P}
\newcommand{\E}{\mathbb{E}}
\newcommand{\divkk}{\mathbb{K}}
\newcommand{\entropy}{\mathbb{H}}
\newcommand{\gem}{\mathrm{GEM}}
\newcommand{\Mult}{\mathrm{Mult}}
\newcommand{\DP}{\mathrm{DP}}
\newcommand{\IBP}{\mathrm{IBP}}
\newcommand{\V}{\mathcal{V}}
\newcommand{\N}{\mathcal{N}}
\newcommand{\mat}[1]{\mathbf{#1}}
\newcommand{\unit}{1\!\!1}
\newcommand{\mg}{\mathcal{M}_g}
\newcommand{\me}{\mathcal{M}_e}
\newcommand{\M}{\mathcal{M}}

\newtheorem{definition}{Definition}[section]
\newtheorem{lemma}{Lemma}[section]
\newtheorem{proposition}{Proposition}[section]
\newtheorem{theorem}{Theorem}[section]
\newtheorem{corollary}{Corollary}[section]
\newtheorem{proof}{Proof}[section]



\begin{document}

\def\spacingset#1{\renewcommand{\baselinestretch}%
{#1}\small\normalsize} \spacingset{1}


%%%%%%%%%%%%%%%%%%%%%%%%%%%%%%%%%%%%%%%%%%%%%%%%%%%%%%%%%%%%%%%%%%%%%%%%%%%%%%

\if1\blind
{
  \title{\bf Stochastic Mixed Membership Models and Preferential Attachment in Social Networks}
  \author{Author 1\thanks{
    The authors gratefully acknowledge \textit{please remember to list all relevant funding sources in the unblinded version}}\hspace{.2cm}\\
    Department of YYY, University of XXX\\
    and \\
    Author 2 \\
    Department of ZZZ, University of WWW}
  \maketitle
} \fi

\if0\blind
{
  \title{\bf Stochastic Mixed Membership Models and Preferential Attachment in Social Networks}
%  \bigskip
%  \bigskip
%  \bigskip
%  \begin{center}
%    {\LARGE\bf Stochastic mixed membership models and preferential attachment in social networks}
%\end{center}
%  \medskip
 \maketitle
} \fi

\bigskip
\begin{abstract}
We assess here whether standard stochastic mixed membership models are adapted for link prediction in social networks by studying how they handle preferential attachment. Preferential attachment charaterizes the propensity of users, in a given social network, to connect to users who have already a lot of connections. Preferential attachement can be global, in which case nodes are connected across communities, and/or local to the network communities. We first introduce in this study formal definitions for both global and local preferential attachement, in different settings, prior to study how standard stochastic mixed membership models behave. Our theoretical analysis reveals that these models do not comply with global preferential attachment, whereas their compliance to local preferential attachment depends on whether the memberships to latent factors are hard or soft, and in the latter case on whether the underlying latent factor distribution is bursty or not. We illustrate these elements on both synthetic and real networks.
\end{abstract}

%\noindent%
%{\it Keywords:} mixed membership models, social networks, preferential attachment.
%\vfill

%\newpage
\spacingset{1.45} % DON'T change the spacing!


\section{Introduction}
\label{sec:intro}

Several powerful relational learning models have been proposed to solve the problem commonly referred to as \textit{link prediction} that consists in predicting the likelihood of a future association between two nodes in a network \cite{LibenNowell07,HassanZaki11}. Among such models, the class of stochastic mixed membership models has received much attention as such models can be used to discover hidden properties and infer new links in social networks. Two main models in this class have been proposed and studied in the literature: the latent feature model \cite{BMF} and its non-parametric extension \cite{ILFRM}, and the mixed-membership stochastic block model \cite{MMSB}, and its non parametric extension \cite{iMMSB,fan2015dynamic}. More generally, these models fall in the category of mixed-membership models who establish a common theoretical framework that encompass a wide range of models (such as admixture and topic model) that are able to  learn complex patterns from structured data \cite{airoldi2014handbook}.~\\
Nevertheless, although drawn from a wide range of domains, real world social networks exhibit general properties and one can wonder if these models are able to capture these properties. In this work, we focus on the \textit{preferential attachment} effect \cite{Newman2010, Barabasi2003}. Preferential attachment states that a vertex is more likely to create connections with vertices having a high degree. In graph theory, preferential attachment is used to explain the emergence of scale-free networks that are characterized by a power-law degree distribution. The aim of our study is to assess to which extent link prediction models, as the ones mentioned above, comply with this property. However, as these models belong to the family of Bayesian model,  we can study their behavior in two settings.  In the first one, denoted $\mathcal{M}_g$, we consider the model as a pure generative model and given the parameters, we use it to generate artificial networks. Like this, we can study the model properties based on their expectation over the random parameters. In the second setting, denoted $\mathcal{M}_e$ and corresponding to the typical use of these models for link prediction, we consider that the parameters are unknown but some observations (\textit{i.e.} an existing network ) are available and are used to estimate the distribution underlying the models. 


The remainder of the paper is organized as follows: in the next section (Section~\ref{sec:rel-work}), we present the related work. In Sections~\ref{sec:background} and \ref{sec:inference}, we describe the main stochastic mixed membership models used for link prediction in social networks, relying on their non-parametric version that generalizes the parametric one and we present the settings $\mathcal{M}_g$ and $\mathcal{M}_e$ according to which we study their compliance to preferential attachment. In Section \ref{sec:burstiness}, we introduce formal definitions of preferential attachment and study how stochastic mixed membership models relate to them. In Section~\ref{sec:exps}, we illustrate our theoretical development on two synthetic networks and two real networks, prior to restate our conclusions in Section~\ref{sec:concl}. 

\section{Related Work}
\label{sec:rel-work}

Recently,  the class of stochastic mixed membership models have been successfully used for link prediction and structure discovery in social networks.  In \cite{AMMSB}, the authors  propose an adaptation of mixed-membership stochastic block model (MMSB), called a-MMSB where "a" stands for assortative, and they used it for discovering overlapping communities in large size networks having millions of nodes since a-MMSB  scales well using stochastic variational inference. They constrained the weight matrix to have weights in the diagonal and a fixed small value elsewhere. A non parametric dynamic version of MMSB model has also been introduced to  handle temporal networks \cite{fan2015dynamic}. The latent feature model (LFM) has also been extended in several way, to handle non-negative weights in \cite{IMRM} and with a more subtle latent feature structure in \cite{ILAM}. Nevertheless, the characterization of these models with regards to the properties of the networks remains to be explored \cite{jacobs2014unified}. ~\\

In this article, we focus on \textit{preferential attachment} \cite{Newman2010, Barabasi2003}. The interest of this property has been widely emphasized in previous works notably for modeling and generating networks reflecting properties of real networks, as in the Barab\`asi-Albert model \cite{albert2002statistical} or Buckley and Osthus model \cite{Buckley2001} that integrate a preferential attachment mechanism or in the Dancer model for generating dynamic attributed networks with community structure \cite{Largeron2017}. Preferential attachment has also  been exploited for improving the results obtained in classical tasks such as community detection \cite{Ciglan2013} or link prediction \cite{Zeng2016}. \\

That said, few theoretical works have been done to study to what extent models comply with this property. 
Orbanz and Roy  pointed out that models belonging to the family of infinitely exchangeable Bayesian graph models cannot generate sparse networks and are thus less compatible with power law degree distributions \cite{orbanz2015bayesian}. Consequently, Lee \textit{et al.}  proposed a random network model in order to capture the power law typical of the degree distribution in social networks \cite{Lee2015}. However the model remains challenging to use in practice, especially for link prediction, due to the relaxation of the exchangeability assumption.~\\

% Concerning the homophily effect, \cite{hoff2008modeling} pointed out that the latent eigen model (called MLFM, an extension of LFM) can comply with both homophily and stochastic equivalence in undirected graphs but without providing a formal definitions of these properties. Furthermore, Li \textit{et al.}, suggest that the latent eigen model  MLFM fails to model homophily  for directed graphs and, for correcting that, designed the GLFM model \cite{Li11}.~\\

Following these previous studies, we study, in a theoretical way, how the non-parametric versions of the classical stochastic mixed membership models handle preferential attachment. For this purpose, we introduce formal definitions of this phenomenon and then study how the models behave with respect  to these definitions but, first,  we present these models and the settings in which we study their behavior.

\section{Models}
\emph{Yet another view} ~\\
\label{fig:bayes_net}

Our two chosen baseline use prior distributions that fall into the two major classes of discrete nonparametric priors. The Hierarchical Dirichlet Process (HDP) that generalizes the Latent Dirichlet Allocation (LDA) for infinite mixtures models. On the other hand, the Indian Buffet Process (IBP), which is the generalization of the Beta-Bernoulli compound distribution (ie Beta Process), which generates infinite binary matrices. The nonparametric models in their truncated version are equivalent to well-known models such as LDA, widely used for text analysis, and Mixed Membership Stochastic Blockmodel which is an adaptation of the latter for relational learning.~\\


We adopt the following notation; if a matrix has a negative index superscripted, it indicates that the values corresponding to this index are excluded. A dot $\bm{.}$ in the index means that we marginalize over all possible values.


\subsection{Binary Feature}
In the latent feature models, the features are distributed according to an Indian Buffet Process (IBP), and the weights interaction according to a Gaussian :
\begin{align}
\Theta &\sim \IBP(\alpha)  \quad\text{ is a } N\times K \text{ matrix}\\
\phi_{mn} &\sim N(0, \sigma_w) \quad\text{ for } m,n \in \{1, .., K\}^2
\end{align}
The observation level is defined by deterministically selecting the row of $\Theta$ which is distributed according an IBP prior. Hence for a node $i$, we note his feature vector by $F_i = \theta_i$ and for $i, j \in V$ we have:

\begin{align}
y_{ij} &\sim \mathrm{Bern}(\sigma(F_i \Phi F_j^\top))
\end{align}
Finally the function $\sigma(x)= \frac{1}{1 + e^{-x}}$ is the sigmoid function to map $[-\infty, +\infty]$ values to [0,1], a probability space.

\begin{figure}[h]
	\centering
	\minipage{0.25\textwidth}
	\scalebox{0.88}{
	/home/dulac/Documents/workInProgress/networkofgraphs/papers/personal/figures/draw/ilfrm2.tex}
	\endminipage
	\minipage{0.25\textwidth}
	\scalebox{0.88}{
		/home/dulac/Documents/workInProgress/networkofgraphs/papers/personal/figures/draw/mmsb2.tex}
	\endminipage
	\caption{The two Graphical model for (left) the latent feature model and (right) the latent class model. The conceptual difference between the two model are the actualization of latent variables for each interaction. The latent feature model uses the same latent features vector for each node's interaction, while the latent class model draws new variables at each interactions.}
	\label{fig:ilfrm}
\end{figure}

\subsubsection{MCMC Updates for Posterior Inference}

The update for the latent features are obtained by the following Gibbs updates:
\begin{align}
& P(f_{ik} = 1 \mid F^{-ik}) = \frac{m^{-ik}}{N} \\
& P(f_{ik} = 0 \mid F^{-ik}) = 1 - \frac{m^{-ik}}{N}
\end{align}
Where $m^{-ik}$ represents the number of active features $k$ for all entities excluding entity $i$, hence $m^{-ik} = \sum_{j\neq i}f_{jk}$. 

The learning of the weight matrix $W$ is computed using a Metropolis-Hasting algorithm. Thus, we sample sequentially each weight corresponding to non-zeros features interaction.

\begin{equation}
P(\phi_{mn} \mid Y, F, \phi_{-mn}, \sigma_w) \propto P(Y \mid F, \Phi) P(\phi_{mn} \mid \sigma_w)
\end{equation}

We choose a jumping distribution in the same family of our prior on weight centered around the previous sample:

\begin{equation} \label{eq:j_w}
J(\phi_{mn}^* \mid \phi_{mn}) = \mathcal{N}(\phi_{mn}, \eta)
\end{equation}
with $\eta$ a parameter letting us controlling the acceptance ratio.

The acceptance ratio of $\phi_{mn}^*$ is thus:
\begin{equation} \label{eq:r_w}
r_{\phi_{mn}\rightarrow \phi_{mn}^*} = \frac{ P(Y \mid F, \Phi^*)P(\phi_{mn}^* \mid \sigma_w)J(\phi_{mn} \mid \phi_{mn}^*) }{ P(Y \mid F, \Phi)P(\phi_{mn} \mid \sigma_w)J(\phi_{mn}^* \mid \phi_{mn} )}
\end{equation}



\subsection{Proportion Feature}

In the latent class model, the rows of the feature matrix $\Theta$ are Dirichlet distributed, and the weights interaction are Beta distributed : 
\begin{align}
\theta_i &\sim \mathrm{Dirichlet}(\alpha_0) \quad\text{ for }  i \in \{1, .., N\} \\
\phi_{mn} &\sim \mathrm{Beta}(\lambda) \quad\text{ for }  m,n \in \{1, .., K\}^2 
\end{align}
The observation level is defined by multinomial draws of class assignments for each nodes. The likelihood for a links depends only on the class assignments of the nodes  and for $i, j \in V$, we have:  
\begin{align}
z_{i\rightarrow j} &\sim \mathrm{Mult}(\theta_i) \\
z_{i\leftarrow j} &\sim \mathrm{Mult}(\theta_j) \\
y_{ij} &\sim \mathrm{Bern}(z_{i\rightarrow j} \Phi z_{i\leftarrow j}^\top)
\end{align}

In this special case, the mapping function $\sigma$ is the identity.

\paragraph{Altenartive Description ! (wich one we keep ?)}~\\

An alternate view, maybe more consistent with the IBP representation is to say that:
\begin{align}
Z &\sim CRF(\alpha_0, \gamma) \\
\phi_{mn} &\sim \mathrm{Beta}(\lambda) \quad\text{ for }  m,n \in \{1, .., K\}^2  \\
y_{ij} &\sim \mathrm{Bern}(\Phi_{z_{i\rightarrow j} , z_{i\leftarrow j}})
\end{align}

In the (not printed here) corresponding graphical model, the arrow between $z_{\rightarrow}$ and $Z$ are dashed as for the IBP representation. Note that $Z \in \{1,.., K\}^{N\times N \times 2}$. ~\\

@ERIC: This representation would justify to explain our interpretation for the Chinese Restaurant Franchise (CRF), because we see it explicitly here...

\subsubsection{MCMC Updates for Posterior Inference}~\\

In the latent class model, the collapsed gibbs sampling approach allow us to only sample the classes couple for each observations:

\begin{equation}
\pr(z_{ji}=k, z_{ij}=k' \mid .) \propto \pr(z_{ji}=k \mid .) \pr(z_{ij}=k' \mid .) f_{(k,k')}^{-ji}(y_{ji})
\end{equation}


The class assignment updates for the node couple are:
\begin{align}
\pr(z_{i\rightarrow j} =k \mid Z^{-ij}) &\propto N_{ik}^{-ij} + \alpha_0 \\
\pr(z_{i\leftarrow j} =k \mid Z^{-ij}) &\propto N_{jk}^{-ij} + \alpha_0 
\end{align}

And the likelihood of a link given the couple $c=(k, k')$ is:
\begin{equation} \label{eq:cgs_mmsb}
f_{(k, k')}^{-ij}(y_{ij}=r) = \frac{C_{(k,k')r}^{-ij} + \lambda_r}{C_{(k,k')\bm{.}}^{-ij} + \sum_r\lambda_{r'}} 
\end{equation}

Finally $\Theta$ and $\Phi$ can be reconstructed from the count matrices with the following equations:
\begin{align}
\theta_{ik} &= \frac{N_{ik} + \alpha_0}{ N_{i\bm{.}} + K\alpha_0} \\
\phi_{rc} &= \frac{C_{cr} + \lambda_r}{ C_{c.} + \lambda_{\bm{.}}}
\end{align}

$C$ and $N$ are counts matrices and are describes in section \ref{burst_class}. We refer to the supplementary material for detail of derivations for MCMC updates. 


\subsection{Comparison of models}
class proportion vs feature vector \\
Class strength vs feature correlation/metric ? \\
HDP vs IBP \\
complextity $O((E^2K^2)$ vs $O(NK^3)$ \\
property yes/no


%\section{Bayesian Context}
\label{sec:inference}
In a Bayesian context, where the parameters underlying the model are known \textbf{AD: the models are hierarchical with latent("unknow") parameters and "known" hyper-parameters. What parameters are you refering to ?}, the learning process consists in finding the posterior distribution of the random parameters $F$ and $\Phi$, given the observed data $Y$, such that : 

\begin{equation}
    \p(F, \Phi | Y, \mg) = \frac{\p(Y|F,\Phi)\p(F|\mg)\p(\Phi|\mg)}{\p(Y|\mg)}
\end{equation}

Where $\mg$ is the set of the hyperparameters for the current model, respectively $\alpha$ and $\sigma_w$ for ILFM and $\gamma$,  $\alpha_0$, $\lambda_0$ and $\lambda_1$ for IMMSB.


For mixed membership models the evidence $\p(Y|\mg)$ has no closed form solution which makes a direct MAP inference procedure infeasible. Thus, the learning process relies on an approximate inference. It consists in a iterative procedure that updates the posterior through typically MCMC updates for true posterior recovery.

Typically MCMC update consists in the following updates of parameters :  

\begin{align}
    \hat f_{ik} &\sim \p(f_{ik} | F^{-ik}, Y, \mg) \\
    \hat \phi_{kk'} &\sim \p(\phi_{kk'} | \Phi^{-kk'}, Y, F, \mg)
\end{align}

At the end of the inference process, assuming that the MCMC has reached its  equilibrium, one can reconstruct the posterior parameters such that $\hat F = (\hat f_{ik})_{i,k \in V\times[0,.., K-1]}$ and $\hat \Phi = (\hat \phi_{kk'})_{k,k' \in [0,.., K-1]^2}$ and :

\begin{equation}
    \p(F, \Phi|Y, \mg) \approx \p(\hat F) \p(\hat \Phi)
\end{equation}

Finally the prediction task consists of measuring an unobserved variable $y^{new}$ given that the information from the observed data was transferred to the posterior distribution : 

\begin{align*}
    \p(y^{new} | Y, \mg) &= \int_F \int_\Phi \p(y^{new}|F,\Phi) \p(F,\Phi|\mg) dF d\Phi \\
                          &\approx \E_{\hat F, \hat \Phi} [y^{new} | \hat F, \hat \Phi] = \mathrm{Bern}(\mathcal{K}(f_i \Phi f_j^T)) \\
                          &= \p(y^{new} | \me)
\end{align*}

Where $\mathcal{K}$ is an isomorphism used to map the support of the bilinear product to a probability space, wich is a sigmoid and the identity for respectively ILFM and IMMSB. Furthermore we denote the set of estimated parameters $\me = \{\hat F,  \hat \Phi \}$.

In the typical use of the above models for link prediction, some observations (\textit{i.e.} an existing network, observed till a certain time) are available and are used to estimate $\mat{F}$ and $\mat{\Phi}$, from which new links are predicted. In the remainder, we denote by $\mat{\hat{F}}$ and $\mat{\hat{\Phi}}$ the estimates of $\mat{F}$ and $\mat{\Phi}$, that can be obtained through standard (collapsed) Gibbs sampling and Metropolis-Hastings algorithms. We do not detail them here and refer the interested reader to \cite{ILFRM,IBP,HDP,fan2015dynamic}. We furthermore denote by $\mathcal{M}_e$ the version of both \ifm\ and \imb\ models in which $\mat{F}$ and $\mat{\Phi}$ are assumed known and fixed to $\mat{\hat{F}}$ and $\mat{\hat{\Phi}}$. We now investigate whether, from the learned parameters $\mat{\hat{F}}$ and $\mat{\hat{\Phi}}$, the new links generated produce a network that comply with preferential attachment.


In the rest of this paper, we will refers to both context in which we study a Bayesian model, and we recall here the fundamental property of each one :
\begin{itemize}
    \item $\mg$ that represents the set of hyperparameters of $\mathcal{M}$, such that the random graph $G$ is exchangeable, thus for any permutation $\pi$ on integers, one has: \[ P((y_{ij})_{i,j\in \mathcal{R}} | \mg) = P((y_{\pi(i)\pi(j)})_{i,j\in \mathcal{R}} | \mg) \]
    \item $\me$ that represents the set of elements ($\bm{F}$ and $\bm{\Phi}$) \textbf{ CL :  $\me = \{\hat F,  \hat \Phi \}$ ?} of $\mathcal{M}$ such that interactions are conditionally independent:  \[P((y_{ij})_{i,j\in \mathcal{R}} | \me) = \prod_{i,j\in \mathcal{R}} P(y_{ij} | \me)  \]
\end{itemize}

%
%
%  Theritical Extension
%
%

% Links to conjugate prior

%\paragraph{Remark (Diaconis-Ylvisaker characterisation of conjugate priors.}~\\
%In the case of conjugate distribution, the Diaconis-Ylvisaker theorem give insight about the form of the predictive distribution $\p(y^{new} | \me)$ \cite{orbanz2009functional}.
%
%\begin{theorem}[Diaconis-Ylvisaker characterisation of conjugate priors]
%Let $P_x(.|\Theta)$ be a natural exponential family model dominated by Lebesgue measure, with open parameter space $\Omega_\theta \subset \mathbb{R}^d$.
%    Let $P_\theta$ b a prior on $\Theta$ which does not concentrate on a singleton. Then $P_\theta$ is a  conjugate prior of $P_X$  w.r.t Lebesgue measure on $\mathbb{R}^d$ if and only if :
%
%\begin{equation}
%    \E_{P_\Theta(\Theta|X_1=x_1,...,X_n=x_n)} [\E_{P_X(x|\Theta=\theta)}[X]] = \frac{y+n\hat x}{a+n}
%\end{equation}
%
%\end{theorem}
%
%That is, given observation $x_1,...,x_n$, the expected value of a new draw $x$ under unknown value of the parameter is linear in the sample average $\hat x = \frac{1}{n}\sum x_i$.



% Relation of Me/Mg and aldoos-hoover theory representation.

\section{Preferential attachment}
\label{sec:burstiness}

We first study global preferential for the models ILFM and IMMSB in the two contexts defined by $\mathcal{M}_g$ and $\mathcal{M}_e$. We then turn our attention to local preferential attachment.
%The preferential attachment effect can be observed directly over the global network, or indirectly on latent classes.

\subsection{Global preferential attachment}

Probabilistic models naturally lead to the following generative process for creating links between nodes in a network\footnote{For simplicity in the notation, we consider that nodes can be linked to themselves. Excluding such links does not raise particular problems.}. This process considers all possible pairs of nodes in turn and generates or not a link between them:

\begin{enumerate}
 \item For each node $i \in \{1, .., N\}$,
 \item For each node $j \in \{1, .., N\}$,
 \item Generate a link between $i$ and $j$ with probability $P(y_{ij}=1 | \mathcal{M})$ where $\mathcal{M}$ is either $\mathcal{M}_e$ or $\mathcal{M}_g$.
\end{enumerate}

As one can note, this process considers all nodes in turn, from node 1 to node $N$. Any indexing, \textit{i.e.} a mapping between nodes and integers in $\{1,\cdots,N\}$, is however arbitrary and conclusions drawn from the above process should be independent of the indexing. As we will see, the results we establish below are indeed independent of the indexing.

For a given node $i$ at step $p$ of the above process, $p$ nodes, from node 1 to node $p$, have been considered and links from these nodes to node $i$ generated or not. We will denote by $d_i^{(p)}$ the degree of node $i$, i.e. the number of links of node $i$, at the $p^{th}$ step of this process. By definition:
%
\begin{equation} \label{eq:degree_def}
d_i^{(p)} = \sum_{j=1}^p y_{ij}
\end{equation}
%
As mentioned before, preferential attachment characterizes the propensity of nodes in social networks to connect to nodes that already have a lot of connections and can be stated as \textit{the higher the number of links a node has, the more likely it will get new links}. The following definition directly captures this idea:
%
\begin{definition}[global preferential attachment]
We say that a probabilistic model satisfies the global preferential attachment effect iff for any indexing, for node $i, \, 1 \leq i \leq N$, for any $p, \, 1 \leq p < N$, $P(d_i^{(N)} \geq n+1 | d_i^{(p)} = n; \M)$ increases with $n$ ($1 \leq n < p$). If $P(d_i^{(N)} \geq n+1 | d_i^{(p)} = n; \M)$ is independent of $n$, the model is said to be neutral \textit{w.r.t.} the global preferential attachment effect. As before, $\mathcal{M}$ is either $\mathcal{M}_e$ or $\mathcal{M}_g$.
\end{definition}
%
Thus, a model satisfies the global preferential attachment effect iff the more links a node $i$ has at some point in the process, the more likely a new link will be created with that node.

For both ILFM and IMMSB, in $\M_e$, the generation of links are independent of each other. The fact that $n$ links have been created after $p$ steps has no impact on the future links to a given node. In $\M_g$, as one first needs to generate $F$ and $\Phi$ prior to generate all the links, a similar behavior is likely to be observed. Intuitively thus, both ILFM and IMMSB are neutral wrt the global preferential attachment effect. The following property formalizes this intuition.
%
\begin{proposition}[Global preferential attachment] \label{th:mg_glob}
Both ILFM and IMMSB, for both $\M_e$ and $\M_g$, are neutral wrt the global preferential attachment effect.
\end{proposition}
%
\begin{proof}
We first consider model $\M_e$. Fix any indexing, a node $i$, $i \leq i \leq N$, and a step $p$, $1 \leq p < N$. One has, $\forall n, 1 \leq n < p$ :
%
\begin{align*}
P(d_i^{(N)} \geq n+1 | d_i^{(p)}=n, \M_e) &= 1 - P(d_i^{(N)}=n | d_i^{(p)}=n, \M_e) \\
        &= 1 -P(y_{i,p+1}=0, ...,y_{iN}=0 | \M_e ) \\
        &= 1 - \prod_{j=p+1}^N P(y_{ij}=0 | \M_e)
\end{align*}
%
where the last equality comes from the fact that, in $\M_e$, links are independently generated. Similarly:
%
\begin{align*}
P(d_i^{(N)} \geq n+2 | d_i^{(p)}=n+1, \M_e) &= 1 - \prod_{j=p+1}^N P(y_{ij}=0 | \M_e) \\
                    &=P(d_i^{(N)} \geq n+1 | d_i^{(p)}=n, \M_e)
\end{align*}
%
which shows that both ILFM and IMMSB are neutral wrt to global preferential attachment with $\M_e$.

For $\M_g$, it suffices to observe that:
%
\begin{equation*}
P(d_i^{(N)} \geq n+1 | d_i^{(p)}=n, \M_g)  = \int_{\M_e} P(\M_e|\M_g) P(d_i^{(N)} \geq n+1 | d_i^{(p)}=n, \M_e) \ d\M_e
\end{equation*}
%
As the models are neutral with $\M_e$, $P(d_i^{(N)} \geq n+1 | d_i^{(p)}=n, \M_e) = P(d_i^{(N)} \geq n+2 | d_i^{(p)}=n+1, \M_e)$ and thus:
%
\begin{equation*}
P(d_i^{(N)} \geq n+2 | d_i^{(p)}=n+1, \M_g) = P(d_i^{(N)} \geq n+1 | d_i^{(p)}=n, \M_g)
\end{equation*}
%
which completes the proof. \hspace{11cm} $\Box$
\end{proof}
%
We now turn to local preferential attachment that deals with the fact that preferential attachment can be also observed within classes of nodes, as exemplified in \cite{LeskovecBKT08}. The classes we consider here are the latent classes of the stochastic mixed-membership models.

\subsection{Local preferential attachment}
\label{sec:local_me}

\paragraph{Local preferential attachement for ILFM}

For ILFM, the situation wrt to local preferential attachment is very similar to the one for global preferential attachment. This is due to the fact that, in $\M_e$ (i.e. given $F$ and $\Phi$), a local degree can be defined in the same way as the global degree above.

Considering the same generative process as before, for $\M_e$ and ILFM, the local degree in class $k$, $0\leq k\leq K-1$, for a node $i$ such that $f_{ik}=1$ is defined by:
%
\begin{equation*}
d_{i,k}^{(p)} = \sum_{j=1, f_{jk}=1}^p y_{ij}
\end{equation*}
%
Note that if $f_{ik}=0$, $d_{i,k}^{(p)} = 0$ for all $p$. This then leads to the following definition at the local preferential attachment for ILFM.
%
\begin{definition}[ILFM - local preferential attachment, $\M_e$]
We say that ILFM, in $\M_e$, satisfies the local preferential attachment iff for any indexing, for any node $i, \, 1\leq i \leq N$ such that $f_{ik}=1$, and for any step $p, \, 1\leq p < N$, $P(d_{i,k}^{(N)} \geq n+1 | d_{i,k}^{(p)}=n,\M_e)$ increases with $n$ ($1\leq n < p$). If $P(d_{i,k}^{(N)} \geq n+1 | d_{i,k}^{(p)}=n,\M_e)$ is independent of $n$, the model is said to be neutral wrt to the local preferential attachment effect.
\end{definition}
%
As before we have the following property.
%
\begin{proposition}
ILFM, with $\M_e$, is neutral \textit{w.r.t.} the local preferential attachment effect.
\end{proposition}

\begin{proof}
The proof is identical to the first part of the proof for Property \ref{th:mg_glob}.
\end{proof}

\paragrap{IMMSB and local preferential attachment}
For IMMSB in $\M_e$, the situation is similar to the one of ILFM in $\M_g$ as we do not have a direct access to classes, encoded in the $Z$ variables.

One can nevertheless define  local random variables $y_{ij,k}$ that are 1 if a link is generated between nodes $i$ and $j$ within class $k$ and 0 otherwise. One has:

\begin{align*}
P(y_{ij,k}=1 | \M_e) &= P(y_{ij}=1 | z_{i\rightarrow j} = z_{i\leftarrow j}=k, \Phi) P(z_{i\rightarrow j}=k|F)P(z_{i\leftarrow j}=k|F) \\
    &= f_{ik} \phi_{kk} f_{jk}
\end{align*}

The local degree $d_{i,k}^{(p)}$ can then be defined as the expectation over the nodes $1,...,p$ of $y_{ij,k}$:
\begin{align*}
d_{i,k}^{(p)} &= \sum_{j=1}^p P(y_{ij,k}=1 | \M_e)  \\
    &= \sum_{j=1}^p f_{ik} \phi_{kk} f_{jk}
\end{align*}

With this definition of the local degree, we can then used the same definition of the local preferential attachment, in $\M_e$, for IMMSB, than the one for ILFM. 

\begin{proposition}
IMMSB satisfies the local preferential attachment in $\M_e$.
\end{proposition}

\begin{proof}
The positive reinforcement effect of the Dirichlet Process \cite{HDP} at the basis of IMMSB corresponds to a burstiness phenomenon and directly translates, for any $i$ and any $k$, as: $\pr(\hat{f}_{ik} \ge x'+\epsilon' \mid \hat{f}_{ik} \ge x',\mathcal{M}_e)$ increases with $x'$ (for all $\epsilon'$ and $x'$ chosen according to the domain of definition of $\hat{f}_{ik}$). Setting $x=x'(\sum_{j\in V} \hat{\Phi}_{kk} \hat{f}_{jk})$ and $\epsilon = \epsilon'(\sum_{j\in V} \hat{\Phi}_{kk} \hat{f}_{jk})$ and exploiting the fact that $\sum_{j\in V} \hat{\Phi}_{kk} \hat{f}_{jk}$ is positive and independent of $i$ leads to: $\pr(d_{i,k} \ge x+\epsilon \mid d_{i,k} \ge x, \mathcal{M}_e)$ increases with $x$, which proves that IMMSB satisfies the local preferential attachment effect.

\end{proof}


\textcolor{red}{Devellopment for IMMSB in Mg...}



\subsection{Illustration}
\label{sec:exps}

To illustrate our theoretical results, we evaluate the predictive performance and the ability of the models to capture the preferential attachment on artificial and real networks. In order to evaluate this property we used several measures.

%We observed the degree distribution and its corresponding best fitting line in log-log scale. In addition, we use the measure developed in \cite{clauset2009power} for assessing whether empirical data behaves according to a power law.

The measures considered to evaluate the preferential attachment rely on a goodness of fit. Indeed, it has been reported that preferential attachment leads to networks characterized by a degree distribution with heavy tail drawn from a power law \cite{barabasi1999emergence}. A graphical method, most often used to verify that the observations are consistent with this law  consists in constructing the histogram representing the degree distribution and if the plot on doubly logarithmic axes approximately falls on a straight line, then one can assume that the distribution follows a power law. Thus, the comparison of the degree distribution in the log-log scale with a linear function gives us a qualitative measure for the preferential attachment. To obtain a second evaluation of the power law hypothesis for the degree distribution, we follow the statistical framework, introduced by \cite{clauset2009power}, for discerning and quantifying power-law behavior in empirical data. This framework combines maximum-likelihood fitting methods with goodness-of-fit tests based on the Kolmogorov-Smirnov statistic. It includes the following steps:


\begin{itemize}
	\item Estimate the parameters $\alpha$ and $x_\text{min}$ of the power law model. $\alpha$ is the scaling parameter of the law and $x_\text{min}$, the lower bound for the tail. It has been fixed to the smallest value observed in the distributions evaluated, in our experiments to allow their comparisons.
	\item  Using the Kolmogorov-Smirnov (KS) statistic, compute the distance $KS_{obs}$  between the degree distribution obtained on the network with the theoretical distribution corresponding to the power law with the estimated parameters.
	\item Sample $S$ synthetic datasets from the power law with the estimated parameters. For each sample  dataset $s \in S$, compute the distance $KS_{s}$ between the distribution obtained on this synthetic dataset, drawn from the power law, with the corresponding theoretical distribution using the Kolmogorov-Smirnov statistic. 
    \item Decide how many sample dataset $S$ to use, with a rule of thumb, based on a worst-case performance analysis of the test \cite{clauset2009power}. To obtain a precision of the $p$-value about $\epsilon$, on should choose $S = \frac{1}{4}\epsilon^{-2}$.  
    \item  The p-value is defined as the fraction of the resulting statistics $KS_s, s \in \{1,...,S\}$ obtained on the samples larger than the value $KS_{obs}$ computed on the network distribution.  
\end{itemize}

 If  p-value is large (close to 1), then the difference between the data and the model can be attributed to statistical fluctuations alone; if it is small, the model is not a plausible fit for the data and we can not conclude that there is an evidence for the preferential attachment in the network. 
However, as mentioned in \cite{clauset2009power} high value of the $p$-value should be considered with caution for at least two reasons. First, there may be other distribution that match the data equally or better. Second, a small number of samples of the data may lead to high p-value and reflect the fact that is hard to rule out a hypothesis in such a case.



%This framework combines maximum-likelihood methods with goodness-of-fit tests based on the Kolmogorov-Smirnov statistics to compute a $p$-value. If the obtained $p$-value is large (close to 1), then the data is likely to be distributed according to a power law and the associated network displays preferential attachment;  on the other hand, if it is small, the data is likely not distributed according to a power law and the associated network does not display preferential attachment.

For local preferential attachment, we follow the same approach as before to compute the $p$-value, the only difference being that the empirical data does not correspond any longer to the global adjacency matrix, but to reduced matrices for each class. The computation of the reduced adjacency matrices varies from one model to the other:
%
\begin{itemize}
    \item For \imb, for a given class $k$, the reduced adjacency matrix $Y^k$ is defined by:~$y_{ij,k}=1$ if $y_{ij}=1, z_{i\rightarrow j}=z_{i\leftarrow j}=k$ and $0$ otherwise.
    \item For \ifm, the reduced adjacency matrix $Y^k$ is defined by:~$ y_{ij,k}=1$ if $y_{ij}=1 , f_{ik}=f_{jk}=1$ and $0$ otherwise.
\end{itemize}~\\


Note that all our experiments where realized in a platform that we developed and maintain in order to help reproducibility of machine learning experiments. It is available online \footnote{https://github.com/dtrckd/pymake} under a GNU GPL license.

\subsection{Datasets and model parameters}

To illustrate the above developments, we consider two artificial and two real networks, the characteristics of which are summarized in Table~\ref{table:networks_measures}.

\input{table/net_measures}

The non-oriented artificial networks (Network1 and Network2) have been generated with the DANCer-Generator \cite{largeron2015}. This generator has been chosen because it allows one to build an attributed graph having a community structure as well as known properties of real-world networks such as preferential attachment and homophily. In order to test link prediction models on different types of networks, Network1 was generated, by design, to comply with preferential attachment whereas Network2 was not.

The first real network, denoted Blogs \footnote{moreno.ss.uci.edu/data.html\#blogs}, contains front-page hyperlinks between blogs in the context of the 2004 US election. A node represents a blog and an oriented link represents a hyperlink between two blogs. The second one, denoted Manufacturing \footnote{www.ii.pwr.edu.pl/~michalski/index.php?content=datasets\#manufacturing}, is an internal email communication network between employees of a mid-sized manufacturing company. Each node is associated to an employee and an oriented link represents an email sent between the two employees. One can notice that the second network is specific since it is an enterprise network in which the relationships between the employees are (professionally) constrained. This means that this network is less likely to display some of the properties that occur in unconstrained social networks.

The adjacency matrices and global degree distributions of these networks are presented in Figure \ref{fig:corpuses}. The adjacency matrices enable us to visualize some characteristics of the networks such as their density and their clustering patterns: as one can note, Blogs and the two artificial networks (Network1 and Network2) have a clear community structure, corresponding to the blocks of white dots on the figure, whereas Manufacturing, the denser network, does not have such a structure. Furthermore, the log-log scale plots show that Network1 and Blogs verify the  global preferential attachment (the fitted line represents relatively well the data points) whereas neither Network2 nor Manufacturing verify it. This is confirmed by the $p$-values reported in the first section of Table \ref{table:me_gofit} (Training Datasets): the $p$-value is 1 for Network1 and Blogs, whereas it is null for Network2 and Manufacturing. The parameter $\alpha$ reported in Table~\ref{table:me_gofit} corresponds to the parameter of the estimated power law distribution (\textit{i.e.} the slope of the best fitting line in log-log scale).

Figure \ref{fig:synt_graph_local} represents the local degree distributions for all networks, each curve in each plot being associated to a different class. As the ground truth is not available for the real networks (Blogs and Manufacturing), classes have been determined with Louvain algorithm \cite{Blondel2008} and the local distribution defined according to the obtained classes. As one can note, the plots for Network1 and Blogs are linear for the most frequent degrees, whereas the plots for Network2 and Manufacturing do not display any clear linearity, suggesting that Network1 and Blogs satisfy, at least partly, local preferential attachment whereas Network2 and Manufacturing do not. This is confirmed by the $p$-values reported in Table~\ref{table:me_gofit}: the $p$-value equals  $1$ for Network1 and Blogs,  $0$ for Network2 and $0.4$ for Manufacturing.

\begin{figure}[ht!]
    \centering
        \begin{minipage}{0.4\textwidth}
            \includegraphics[width=\textwidth]{img/corpus/network1_dd}
        \end{minipage}
        \begin{minipage}{0.4\textwidth}
            \includegraphics[width=\textwidth]{img/corpus/network2_dd}
        \end{minipage}
        %\vskip\baselineskip
        \begin{minipage}{0.4\textwidth}
            \includegraphics[width=\textwidth]{img/corpus/blogs_dd}
        \end{minipage}
        \begin{minipage}{0.4\textwidth}
            \includegraphics[width=\textwidth]{img/corpus/manufacturing_dd}
        \end{minipage}
	\caption{Adjacency matrices (left) and global degree distributions (right) for the four training datasets. In the adjacency matrices, a white dot corresponds to a 1 and a black dot to a 0.}
	\label{fig:corpuses}
\end{figure}

\begin{figure}[h]
    \centering
        \begin{minipage}{0.4\textwidth}
            \includegraphics[width=\textwidth]{img/corpus/network1_1}
        \end{minipage}
        \begin{minipage}{0.4\textwidth}
            \includegraphics[width=\textwidth]{img/corpus/network2_1}
        \end{minipage}
        \vskip\baselineskip
        \begin{minipage}{0.4\textwidth}
            \includegraphics[width=\textwidth]{img/corpus/blogs_1}
        \end{minipage}
        \begin{minipage}{0.4\textwidth}
            \includegraphics[width=\textwidth]{img/corpus/manufacturing_1}
        \end{minipage}
        \caption {Local degree distributions for the four training datasets. For Network1 and Network2 the classes come from ground-truth. For Blogs and Manufacturing, classes are obtained by Louvain algorithm.} 
	\label{fig:synt_graph_local}
\end{figure}

For each dataset, we estimate the model parameters through Markov Chain Monte Carlo inference consisting of 200 iterations. For \imb, the concentration parameters of HDP were optimized  using vague gamma priors $\alpha_0 \sim \text{Gamma}(1,1)$ and $\gamma \sim \text{Gamma}(1,1)$ following \cite{HDP}. The parameters for the matrix weights  $\lambda_0$ and $\lambda_1$ were fixed to 0.1. For \ifm, the hyperparameter  $\sigma_w$ was fixed to 1 and the IBP hyperparameter $\alpha$ to 0.5 in order to  have comparable number of classes with \imb. Once the models have been learned, they are used to generate links (or non-links) between the entire set of network nodes. The whole procedure is repeated 10 times and the average values are reported as final results.


\subsection{Preferential attachment in $\M_e$}

Table \ref{table:me_gofit} reports the value of the power-law goodness of fit for \imb\ and \ifm\ in the global case (left) and in the local case (right). It appears that for both models, the global preferential attachment is only verified for networks generated from datasets where the property was observed, namely in Network1 with p-value equal to 0.9 for \imb\ and 1 for \ifm, and in Blogs with a p-value equal to 1 for both models; the property is not verified in Network2 and in Manufacturing, where p-values are equal to 0. This is in accordance with Proposition 2.1 according to which both \ifm\ and \imb\ do not satisfy global preferential attachment. However, these models are able to capture this property if it exists in the training datasets.  Moreover, one can observe that, in the local case, \imb\ complies with the preferential attachment with $p$-values equal or close to 1 for the four networks, while \ifm\ obtained low p-values for the networks that were less locally bursty (respectively  0  for Network2 and 0.3 for Manufacturing). In addition, the power-law coefficients $\alpha$ are significantly greater for \imb\ than for \ifm, and specially for the bursty networks Network1 and Blogs.

Figure \ref{fig:me_local} illustrates the local preferential attachment for Network1 (top) and Network2 (bottom) estimated with \imb\ (left) and \ifm\ (right). The shape of the local degree distributions appears more linear for \imb\ and with more fluctuations for \ifm. This illustrates the fact that \ifm\ does not capture local preferential attachment whereas \imb\ does, as stated in Proposition 2.2. 


\begin{table}[t]
\caption{Preferential attachment measures for training datasets and networks generated with fitted models.}
\centering
\begin{tabular}{lrrrr}
  \multirow{2}{*}{\textbf{Training Datasets}}  &
  \multicolumn{2}{c}{Global} & \multicolumn{2}{c}{Local}\\
  \cmidrule(r){2-3} \cmidrule(l){4-5}
  &   $p$-value &   $\alpha$   & $p$-value & $\alpha$   \\
\hline
Network1       & 1 & 2.4 &   1.0 $\pm$ 0.0  &  1.8 $\pm$ 0.03  \\
Network2       & 0 & 1.3 &   0.0 $\pm$ 0.0  &  1.2 $\pm$ 0.01 \\
Blogs          & 1 & 1.5 &   1.0 $\pm$ 0.0  &  1.4 $\pm$ 0.03\\
Manufacturing  & 0 & 1.4 &   0.4 $\pm$ 0.3  &  1.3 $\pm$ 0.05 \\
\hline

  \ \textbf{\imb} &&&& \\
\hline
Network1       & 0.9 & 1.4 &   1.0 \(\pm\) 0.0   &  3.5 \(\pm\) 0.7 \\
Network2       & 0 & 1.3 &   0.9 \(\pm\) 0.0   &  1.6 \(\pm\) 0.2 \\
Blogs          & 1 & 1.3 &   1.0 \(\pm\) 0.0   &  4.3 \(\pm\) 1.1 \\
Manufacturing  & 0 & 1.2 &   0.9 \(\pm\) 0.01  &  1.6 \(\pm\) 0.1 \\
\hline

  \ \textbf{\ifm} &&&& \\
\hline
Network1      & 1 & 1.4 &   1.0 \(\pm\) 0.0  &  1.7 \(\pm\) 0.1 \\
Network2      & 0 & 1.2 &   0.0 \(\pm\) 0.0 &  1.2 \(\pm\) 0.0 \\
Blogs         & 1 & 1.3 &   0.9 \(\pm\) 0.2  &  1.5 \(\pm\) 0.1 \\
Manufacturing & 0 & 1.2 &   0.3 \(\pm\) 0.3  &  1.3 \(\pm\) 0.0 \\
\hline
\end{tabular}
\label{table:me_gofit}
\end{table}


\begin{figure}[h]
    \centering
    \begin{minipage}{0.4\textwidth}
        \includegraphics[width=\textwidth]{img/corpus/immsb_network1_1}
    \end{minipage}
    \begin{minipage}{0.4\textwidth}
        \includegraphics[width=\textwidth]{img/corpus/ilfm_network1_1}
    \end{minipage}
    \vskip\baselineskip
    \begin{minipage}{0.4\textwidth}
        \includegraphics[width=\textwidth]{img/corpus/immsb_network2_1}
    \end{minipage}
    \begin{minipage}{0.4\textwidth}
        \includegraphics[width=\textwidth]{img/corpus/ilfm_network2_1}
    \end{minipage}
    \caption {Local degree distributions for Network1 (top row) and Network2 (bottom row) generated with fitted models \imb\ (first column) and \ifm\ (second column).} 
\label{fig:me_local}
\end{figure}

\begin{figure}[ht!]
\centering
    \begin{minipage}{0.4\textwidth}
        \includegraphics[width=\textwidth]{img/corpus/roc_network1_75_f}
    \end{minipage}
    \begin{minipage}{0.4\textwidth}
        \includegraphics[width=\textwidth]{img/corpus/roc_network2_75_f}
    \end{minipage}
    \begin{minipage}{0.5\textwidth}
        \includegraphics[width=\textwidth]{img/corpus/testset_max_20}
    \end{minipage}
    \caption{Top: AUC-ROC curves for Network1 (left) and Network2 (right) with 75 percent of data used for learning that compares the performance of models. Bottom: Relative performance of \imb\ and \ifm\ according to the percentage of data used for testing, the rest being used for learning.} 
\label{fig:auc}
\end{figure}

Lastly, Figure \ref{fig:auc} compares the performance of the models for predicting new links using the Area Under the Curve (AUC) measure as a function of the training set size. In the bottom plot, the y-axis gives the relative performance defined as the difference of the AUC values for \imb\ and \ifm\ ($AUC_{\imb} - AUC_{\ifm}$) whereas the x-axis indicates the percentage of links randomly removed from the datasets and used as test examples. Hence, the number of training data decreases with the x-axis and a positive value on the y-axis indicates that \imb\ outperforms \ifm.  The relative performance corresponds to the difference of the MAX AUC values obtained for both models on the 10 inference experiences. The top plots illustrate a case where 75 percent of the data is used as test set and where \imb\ dominates \ifm\ on Network1 (left), and the opposite on Network2 (right).

In general, as shown in the bottom plot, \ifm\ obtains better performance than \imb. However, the relative predictive performance of \imb\  increases  when the quantity of training data decreases on bursty networks, whereas for non-bursty networks the results are the opposite: the performance of \ifm\ increases when the size of the learning dataset decreases. This is particularly visible for Network2. The results for Manufacturing are less marked, which is certainly due to the small size of this network, making the prediction less challenging.

The above behavior can be explained by the fact that \imb\ satisfies the local preferential attachment whereas \ifm\ does not: as links are randomly removed, one is more likely to remove links from large classes than from small ones; a model that enforces local preferential attachment on bursty networks is thus more likely to reconstruct those removed links. This is what is happening on Network1 and Blogs for \imb. On the contrary, for non-bursty networks, a model enforcing local preferential attachment is penalized.

In figure \ref{fig:burst_immsb} and \ref{fig:burst_ilfm} we show respectively for IMMSB and ILFM the evolution of the local preferential attachment following the definition given in section \ref{sec:local_me}, for the networks Manufacturing and Networks1 and for two different values of the generating process step $p$ (for 85\% and 95\% of the number nodes $N$). For IMMSB one can show that the probability of generating new links increase with the degree (the degree in reported in the x-axis for each class grouped together.). However, for ILFM, one can observe some classes where the preferential attachment is no true such as class 3 in Manufacturing where the probability to generate new links decrease with the degree or contains some plateau. Thus in Networks1, one can also a observe a tendency of the probability of new links to increase with the degree because the model successfully fitted the data, but still, the increase contains plateau and locally decreasing parts.

\begin{figure}[h]
\centering

\begin{subfigure}
     \centering
         \includegraphics[width=0.45\textwidth]{img/burst/3_prop2_process_local_me__85}
\end{subfigure}
\begin{subfigure}
         \centering
      \includegraphics[width=0.45\textwidth]{img/burst/3_prop2_process_local_me__95} 
\end{subfigure}                                                                          
\begin{subfigure}                                                                        
         \centering                                                                      
      \includegraphics[width=0.45\textwidth]{img/burst/5_prop2_process_local_me__85}
\end{subfigure}                                                                          
\begin{subfigure}                                                                        
         \centering                                                                      
      \includegraphics[width=0.45\textwidth]{img/burst/5_prop2_process_local_me__95}
\end{subfigure}                                                                          
\caption{burstiness process illustrated by the probability to generate new links for degree at step $p$.}
\label{fig:burst_immsb}



\end{figure}

\begin{figure}[h]
\centering

\begin{subfigure}
     \centering
         \includegraphics[width=0.45\textwidth]{img/burst/2_prop2_process_local_me__85}
\end{subfigure}
\begin{subfigure}
         \centering
      \includegraphics[width=0.45\textwidth]{img/burst/2_prop2_process_local_me__95} 
\end{subfigure}                                                                          
\begin{subfigure}                                                                        
         \centering                                                                      
      \includegraphics[width=0.45\textwidth]{img/burst/4_prop2_process_local_me__85}
\end{subfigure}                                                                          
\begin{subfigure}                                                                        
         \centering                                                                      
      \includegraphics[width=0.45\textwidth]{img/burst/4_prop2_process_local_me__95}
\end{subfigure}                                                                          
\caption{burstiness process illustrated by the probability to generate new links for degree at step $p$.}
\label{fig:burst_ilfm}



\end{figure}


\subsection{Preferential attachment in $\M_g$}

Illustrations in the $\M_g$ case are based on the simulation of the models where the parameters $F$ and $\Phi$ have been marginalized out. In other words the degrees that we are going to observe are the expected degree for a large numbers (in the sense of the theory of large numbers) of generated parameters, given the hyper-parameters of the model. ~\\

For all our experiments we set the hyper-parameters of the models in order to generate a random network of 200 nodes. We went trough the generative process 100 times or epochs. A significatif number of epoch is crucial to capture the generative property of the models when the latent parameters are marginalized-out, which simulate the $\M_g$ mode.~\\

More precisely, to generate one random network, we first generate the random parameters $F, \Phi$, and then generate the adjacency matrix $Y$. Thus, each generation epoch consists in a triple $(Y, F, \Phi)$. Given this triple, we can compute the different degrees of the network's nodes defined in section \ref{sec:burstiness}.~\\

An important note is that, to be able to compute the statistics for the local degree, the latent classes need to be aligned between the different epochs. But, the mixed membership models do not defined unique label over the latent classes. Consequently they are non-identifiable in principle. Nevertheless, the properties of the Dirichlet Process and the Indian Buffet Process, enable to identify the classes by ordering them with their size (or concentration). For example, the stick breaking process interpretation of the DP provides a natural class ordering by a descending (or ascending) order of the class representation. While the IBP generate a row-exchangeable feature matrix, it is possible to reorder the rows to obtain a $F$ matrix where the size of the classes keep the same descending (or ascending) order.~\\

We chose comparative and standard values of hyper-parameters to setup the models. For both model we set $\lambda_0=\lambda_1=0.5$ and $\alpha=\alpha_0=1$, and for IMMSB, $\gamma=0.5$.~\\

In figure \ref{fig:mg_deg}, we observe the global and local degree distributions for both IMMSB and ILFM. We see that the global degree distributions are not monotone, with several peaks, and that the range values of the outcome degrees are concentrated in a small segment determined by the hyper-parameters of the models. The shape of the global degree distributions show that the global preferential attachment is not satisfied. But for the local degree, one can see that, for IMMSB, the shape of the distributions are characteristic of the preferential attachment effect (linear decrease in a log-log space) while it is not the case for ILFM. % which confirm our statetment.
~\\

\input{img/0201/mg_deg}


%Figures \ref{fig:mg_process} gives a simulation of the evolution of the expected (global and local) degree during the sampling process according to the equation \eqref{eq:degree_def}. We will refer this process as the degree count process which basically is a cummulative count over the nodes of the link measure (eitheir the true link/non-link observation for ILFM or the likelihood of a link for IMMSB) . Because, we observe the degree count averaged on all the nodes, we also plot the variance over the nodes of those degree count process.
%Interestingly, one can show that for both models, the global and local expected degree count processes follow a linear increase, while the variance of the degree count processes are slightly convex. Furthermore, it is more pronounced for IMMSB for the global degree while it seems that concerning the local degree, the models are not differentiable. Nevertheless, if we look at the standard deviation of the variance for each local degree we see that for IMMSB it seems to be correlated to the variance (the more variance (over nodes) in a local degree count, the more this variance can deviate. In other words, the variance over the epoch increase too.). This is an interesting interpretation of the source of emergence of the local preferential attachment for IMMSB.
%
%\input{img/0201/mg_process}




\section{Related Work}
\label{sec:rel-work}

Recently,  the class of stochastic mixed membership models has been successfully used for link prediction and structure discovery in social networks. For example, in \cite{AMMSB}, the authors  propose an adaptation of mixed-membership stochastic block model (MMSB) called a-MMSB, where "a" stands for assortative, and use it for discovering overlapping communities in large networks having millions of nodes. The weight matrix is constrained to have a fixed small value outside its diagonal. A non parametric dynamic version of MMSB model has also been introduced to  handle temporal networks \cite{fan2015dynamic}. The latent feature model (LFM) has also been extended in several ways, to handle non-negative weights in \cite{IMRM} and with a more subtle latent feature structure in \cite{ILAM}. Nevertheless, the characterization of these models with regards to the properties of the networks remains to be explored, as mentioned in \cite{jacobs2014unified}.

In this article, we focus on \textit{preferential attachment}, a well known property of social networks \cite{Newman2010, Barabasi2003}. This property has been emphasized in previous studies, for example for modeling and generating artificial networks reflecting properties of real networks, as in the model by Barab\`asi-Albert \cite{albert2002statistical}, the model by Buckley and Osthus \cite{Buckley2001}, which integrate a preferential attachment mechanism, or in the Dancer model for generating dynamic attributed networks with community structures \cite{Largeron2017}. Preferential attachment has also  been exploited for improving the results obtained in classical tasks such as community detection \cite{Ciglan2013} or link prediction \cite{Zeng2016}.

That said, few theoretical works have been conducted to study to what extent stochastic models comply with this property. 
Orbanz and Roy  pointed out that models belonging to the family of infinitely exchangeable Bayesian graph models cannot generate sparse networks and are thus less compatible with power law degree distributions \cite{orbanz2015bayesian}. Consequently, Lee \textit{et al.}  proposed a random network model in order to capture the power law typical of the degree distribution in social networks \cite{Lee2015}. However the model remains challenging to use in practice, especially for link prediction, due to the relaxation of the exchangeability assumption.~\\

% Concerning the homophily effect, \cite{hoff2008modeling} pointed out that the latent eigen model (called MLFM, an extension of LFM) can comply with both homophily and stochastic equivalence in undirected graphs but without providing a formal definitions of these properties. Furthermore, Li \textit{et al.}, suggest that the latent eigen model  MLFM fails to model homophily  for directed graphs and, for correcting that, designed the GLFM model \cite{Li11}.~\\

A preliminary version of this study was published in \cite{dulac-dsaa}. However, the definitions of preferential attachment and local degrees we proposed in this paper are not entirely satisfying inasmuch as the dynamic aspect of preferential attachment was not taken into account. The definitions we propose here and the developments concerning stochastic block models are new and we believe better founded than in this previous work.

We study, in a theoretical way, how the non-parametric versions of the classical stochastic mixed membership models handle preferential attachment. For this purpose, we introduce formal definitions of this phenomenon and then study how the models behave with respect  to these definitions but, first,  we present these models and the settings in which we study their behavior.


\section{Conclusion}
\label{sec:concl}

We have studied whether stochastic mixed membership models, such as \ifm\ and \imb\, can generate new links while satisfying an important property frequently observed in real social networks, namely the preferential attachment. To do so, we have introduced formal definitions adapted for this property in a global and a local context where edge are either considered across the full network or inside communities. We have analyzed how these models behave according to those definitions. We  have shown that stochastic mixed membership models do not comply with global preferential attachment. The situation is however more contrasted when the property is considered at the local level: \imb\ enforces a partial local preferential attachment whereas \ifm\ does not.~\\

These findings have been validated experimentally on two real and two artificial networks that have different degrees of global and local preferential attachment. An important, practical finding of our study is that \imb, usually considered of lesser "quality" than \ifm, can indeed yield better results on bursty networks (\textit{i.e.} networks with preferential attachment) when the number of training data is limited.~\\

There are many directions to extend this work with the motivation of improving our theoretical understanding of graphical models for link prediction in complex networks. A interesting extension is to examine the relation between the local preferential attachment and the dynamic of the latent classes.
%For example, some special value of the hyperparameters, of the non-parametric priors, could lead to a very large number of classes or, at the opposite, just one class. Between these two extremes, that goes from a vanishing local aspect of the degree distribution to a number of classes that overfit the data, on can ask how this parameter affect the global and the local degree distribution of a random graph.  
An other direction of interest in the line of this work is to study how the preferential attachment relate to the sparsity of a network and how the exchangeability assumption should be relaxed in order to have models that naturally comply with both the preferential attachment and the sparsity properties.

%For instance, a fundamental result is the Aldous-Hoover theorem, which implies that exchangeable random graphs cannot be sparse \cite{orbanz2015bayesian}. It seems that the sparsity is related in some way to the preferential attachment in a network. Thus, the following question arises: would it be realistic to assume the exchangeability hypothesis for the local case but not for the global case, and how this fact impacts the burstiness of the global degree distribution and the sparsity of the graph.

%We believe that answering to those questions open a way to develop and design Bayesian models able to better capture the fundamental properties of real social networks.


~\\

\bibliographystyle{chicago}
\bibliography{./a}

%\appendix
%\section{Appendix}

\subsection{Gibbs update for IMMSB}

We provide here the derivation of the Gibbs update rules given in Section~\ref{sec:models} for IMMSB.

From the definition of the model, one has: $\pr(z_{ij} = k \mid \mat{f}_i) = f_{ik}$.

\textcolor{red}{Adrien, peux-tu donner la d\'erivation ?; la forme actuelle n'est valable que pour MMSB.} 



\end{document}

