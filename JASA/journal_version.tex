%%&latex
\documentclass[12pt]{article}
\usepackage{graphicx,psfrag,epsf}
\usepackage{graphicx}
\usepackage{natbib}

%\pdfminorversion=4
% NOTE: To produce blinded version, replace "0" with "1" below.
\newcommand{\blind}{0}

\setlength{\textheight}{24 cm}

% DON'T change margins - should be 1 inch all around.
\addtolength{\oddsidemargin}{-.5in}%
\addtolength{\evensidemargin}{-.5in}%
\addtolength{\textwidth}{1in}%
\addtolength{\textheight}{-.3in}%
\addtolength{\topmargin}{-.8in}%



\newcommand*{\lpath}{./}%
%\usepackage[cmex10]{amsmath, mathtools}
%\usepackage{amsmath,amssymb,amsbsy,amsfonts,amsthm}
\usepackage{amsmath,amssymb,amsbsy,amsfonts}
\usepackage{mathtools}
\usepackage{booktabs}
\usepackage{multirow}
\usepackage{bm}
\usepackage{enumerate}
\usepackage{url}
%\usepackage[ruled,vlined]{algorithm2e}
\usepackage{fancyvrb}
\usepackage{yfonts}
\usepackage{wrapfig}
\usepackage{subfigure}
\usepackage{tikz}
\usetikzlibrary{bayesnet}
\newcommand{\tikzmark}[1]{\tikz[overlay,remember picture] \node (#1) {};}
\usepackage{calc}%    For the \widthof macro
\usepackage{xparse}%  For \NewDocumentCommand


%% Variable de compilation
\newif\ifbeamer
\beamerfalse
\newcommand{\beamer}[2]{\ifbeamer #1 \else #2 \fi}
%%%

%\usepackage[latin1]{inputenc}
\usepackage[utf8]{inputenc} % manage utf8 encodage 
%\usepackage[english]{babel} % for french document ! dirty enumerate style,+ bad change rectangle colors for section linking.
\usepackage{fancyhdr} % for heading
\usepackage{listings}
\usepackage[colorlinks=true, urlcolor=blue]{hyperref} % url, link
\usepackage{graphicx}
\usepackage{geometry}

%\usepackage[cmex10]{amsmath, mathtools}
\usepackage{amsmath,amssymb,amsbsy,amsfonts,amsthm}
\usepackage{multirow}
\usepackage{bm}
\usepackage{enumerate}
\usepackage{url}
\usepackage[ruled,vlined]{algorithm2e}
\usepackage{fancyvrb}
\usepackage{yfonts}

\usepackage{wrapfig}
\usepackage{tikz}
    %\input{../tikz.conf}
    
\usetikzlibrary{bayesnet}
    
%%%%%%%%%%% Box 
\usepackage{calc}%    For the \widthof macro
\usepackage{xparse}%  For \NewDocumentCommand
\newcommand{\tikzmark}[1]{\tikz[overlay,remember picture] \node (#1) {};}

%%%%%%%%%% Math
\renewcommand{\text}{\textnormal}
\newcommand{\pr}{\mathbf{p}}
\newcommand{\E}{\mathbb{E}}
\newcommand{\divkk}{\mathbb{K}}
\newcommand{\entropy}{\mathbb{H}}
\newcommand{\gem}{\mathrm{GEM}}
\newcommand{\Mult}{\mathrm{Mult}}
\newcommand{\DP}{\mathrm{DP}}
\newcommand{\IBP}{\mathrm{IBP}}
\newcommand{\M}{\mathcal{M}}
\newcommand{\V}{\mathcal{V}}
\newcommand{\N}{\mathcal{N}}
    
\makeatletter
\NewDocumentCommand{\DrawBox}{s O{}}{%
    \tikz[overlay,remember picture]{
    	\IfBooleanTF{#1}{%
    		\coordinate (RightPoint) at ($(left |- right)+(\linewidth-\labelsep-\labelwidth,0.0)$);
    	}{%
    	\coordinate (RightPoint) at (right.east);
    }%
    \draw[red,#2]
    ($(left)+(-0.2em,0.9em)$) rectangle
    ($(RightPoint)+(0.2em,-0.3em)$);}
}

\NewDocumentCommand{\DrawBoxWide}{s O{}}{%
	\tikz[overlay,remember picture]{
		\IfBooleanTF{#1}{%
			\coordinate (RightPoint) at ($(left |- right)+(\linewidth-\labelsep-\labelwidth,0.0)$);
		}{%
		\coordinate (RightPoint) at (right.east);
	}%
	\draw[red,#2]
	($(left)+(-\labelwidth,0.9em)$) rectangle
	($(RightPoint)+(0.2em,-0.3em)$);}
}
\makeatother
%%%%% ! Box

\geometry{
      a4paper,
	    body={160mm,260mm},
	    left=25mm,top=20mm,
	    headheight=4mm,headsep=8mm,
        footskip=10mm,
        }
                                              

%%%%%%%%%%%%%%%%%%%%%%%%%%%%%%%%%%%%%%%%%%%%%%%%%%%%%%%%%%%%%%%%%%%%%%%%%%%%%%%%%%%%%%%%%%%%%%%%%%%%%%
%%%%% => Internal
%%%%%%%%%%%%%%%%%%%%%%%%%%%%%%%%%%%%%%%%%%%%%%%%%%%%%%%%%%%%%%%%%%%%%%%%%%%%%%%%%%%%%%%%%%%%%%%%%%%%%%

% itemize item def
%% \begin{itemize}\itemsep2pt % example space betwew item
%\renewcommand{\FrenchLabelItem}{\textbullet}
\renewcommand{\labelitemi}{$\bullet$}
\renewcommand{\labelitemii}{$\cdot$}
\renewcommand{\labelitemiii}{$\diamond$}
\renewcommand{\labelitemiv}{$\ast$}

% equation reference
\renewcommand{\theequation}{\thesection.\arabic{equation}}

%%%%%%%%%%%%%%%%%%%%%%%%%%%%%%%%%%%%%%%%%%%%%%%%%%%%%%%%%%%%%%%%%%%%%%%%%%%%%%%%%%%%%%%%%%%%%%%%%%%%%%
%%%%% => Alias
%%%%%%%%%%%%%%%%%%%%%%%%%%%%%%%%%%%%%%%%%%%%%%%%%%%%%%%%%%%%%%%%%%%%%%%%%%%%%%%%%%%%%%%%%%%%%%%%%%%%%%

% write code
\lstnewenvironment{C}[1]
{\lstset{language=C,
      frame=tBRl,
      basicstyle=\scriptsize,stringstyle=\emph,showstringspaces=false,
      numbers=left,numberstyle=\tiny,
      breaklines=true, columns=flexible, title={#1}}
}{}
      
%%%%%%%%%%%%%%%%%%%%%%%%%%%%%%%%%%%%%%%%%%%%%%%%%%%%%%%%%%%%%%%%%%%%%%%%%%%%%%%%%%%%%%%%%%%%%%%%%%%%%%
%%%%% => Preambles Pages
%%%%%%%%%%%%%%%%%%%%%%%%%%%%%%%%%%%%%%%%%%%%%%%%%%%%%%%%%%%%%%%%%%%%%%%%%%%%%%%%%%%%%%%%%%%%%%%%%%%%%%

\pagestyle{fancy}
\fancyhf{} % remove default headers
\fancyfoot[C]{\thepage}
\renewcommand{\footrulewidth}{0.3pt}
\renewcommand{\headrulewidth}{0.3pt}

%%%%%%%%%% Math
\renewcommand{\text}{\textnormal}
\newcommand{\ifm}{\texttt{ILFM}}
\newcommand{\imb}{\texttt{IMMSB}}
\newcommand{\pr}{P}
\newcommand{\p}{P}
\newcommand{\E}{\mathbb{E}}
\newcommand{\divkk}{\mathbb{K}}
\newcommand{\entropy}{\mathbb{H}}
\newcommand{\gem}{\mathrm{GEM}}
\newcommand{\Mult}{\mathrm{Mult}}
\newcommand{\DP}{\mathrm{DP}}
\newcommand{\IBP}{\mathrm{IBP}}
\newcommand{\V}{\mathcal{V}}
\newcommand{\N}{\mathcal{N}}
\newcommand{\mat}[1]{\mathbf{#1}}
\newcommand{\unit}{1\!\!1}
\newcommand{\mg}{\mathcal{M}_g}
\newcommand{\me}{\mathcal{M}_e}
\newcommand{\M}{\mathcal{M}}

\newtheorem{definition}{Definition}[section]
\newtheorem{lemma}{Lemma}[section]
\newtheorem{proposition}{Proposition}[section]
\newtheorem{theorem}{Theorem}[section]
\newtheorem{corollary}{Corollary}[section]
\newtheorem{proof}{Proof}[section]



\begin{document}

\def\spacingset#1{\renewcommand{\baselinestretch}%
{#1}\small\normalsize} \spacingset{1}


%%%%%%%%%%%%%%%%%%%%%%%%%%%%%%%%%%%%%%%%%%%%%%%%%%%%%%%%%%%%%%%%%%%%%%%%%%%%%%

\if1\blind
{
  \title{\bf A study of stochastic mixed membership models for link prediction in social networks}
  \author{Author 1\thanks{
    The authors gratefully acknowledge \textit{please remember to list all relevant funding sources in the unblinded version}}\hspace{.2cm}\\
    Department of YYY, University of XXX\\
    and \\
    Author 2 \\
    Department of ZZZ, University of WWW}
  \maketitle
} \fi

\if0\blind
{
  \bigskip
  \bigskip
  \bigskip
  \begin{center}
    {\LARGE\bf A study of stochastic mixed membership models for link prediction in social networks}
\end{center}
  \medskip
} \fi

\bigskip
\begin{abstract}
We assess here whether standard stochastic mixed membership models are adapted for link prediction in social networks by studying how they handle homophily and preferential attachment. To do so, we first introduce formal definitions of these phenomena; we then study how stochastic mixed membership models relate to these definitions. Our theoretical analysis reveals that standard stochastic mixed membership models comply with homophily with the similarity that underlies them. For preferential attachment, the situation is more contrasted: if these models do not comply with global preferential attachment, their compliance to local preferential attachment depends on whether the memberships to latent factors are hard or soft, and in the latter case on whether the underlying latent factor distribution is bursty or not. We illustrate these elements on synthetic and real networks by using the generative properties of Bayesian model.
\end{abstract}

\noindent%
{\it Keywords:}  Hierarchical Bayesian Models, Random Graph, Machine Learning.
\vfill

\newpage
\spacingset{1.45} % DON'T change the spacing!

\section{Introduction}
\label{sec:introduction}
In recent years, several powerful relational learning models have been proposed to solve the problem commonly referred to as \textit{link prediction} that consists in predicting the likelihood of a future association between two nodes in a network \cite{Liben-Nowell07, HassanZaki11}. Among such models, the class of probabilistic, generative models has received much attention as such models can be used to both generate artificial networks and infer new links from existing ones. Two main class of models have been proposed and studied in the literature: the latent feature model \cite{BMF} and its non-parametric extension \cite{ILFRM}, and the mixed-membership stochastic block model \cite{MMSB} and its non parametric extensions \cite{iMMSB,diMMSB}. In this paper, we focus on this two model, and study some of its properties related to link prediction in social networks. 

Indeed, although drawn from a wide range of domains, most real world social networks exhibit common properties, such as the \textit{homophily}, \textit{preferential attachement} and \textit{small world} effects \cite{Newman2010, Barabasi2003}. 


A natural question that arises is thus whether or not models as IMMSB and ILFM comply with such properties. Link prediction model, typically learned or given, describe a set of nodes and links between them; Such data defines a random structure $Y$. Given that, we learn a model parameters $\hat \theta$, such that one can then predict the probability that a new link will be drawn between two given nodes of the network by studying the following quantity:
\begin{equation}
\p(y | \hat \theta)
\end{equation}
This quantity is called a predictive likelihood.


A question we ask in this setting is: \textit{Do link prediction models learned can generate networks with the homophily and preferential attachment}.

A second possible use of Bayesian models is as a pure generative model to generate artificial networks. In this setting, we study models properties based on their expectation over their random parameters, defined as follows:

\begin{equation}
\p(y) = \int_{\theta} \p(y,\theta) d\theta
\end{equation}
This quantity is know as the evidence for the data.

The question we ask ourselves in this setting is thus: \textit{Do link prediction models comply with the homophily and preferential attachment effects}.


The remainder of the paper is organized as follows. In the second section \ref{sec:background} we set up the probabilistic context of our analysis, Then in section \ref{sec:models} we present two general class of models in this settings know as class based models and feature based models. Then in section \ref{sec:homophily} and \ref{sec:burstiness} we propose respectively formal definition for the homophily and preferential effects and study how models comply with this propoerties. \textcolor{red}{Then section on feature dynamics and sparsity}. In section \ref{sec:experiments} we report an empirical study of the predictive performance on synthetic and real networks with regards to the properties. We conclude in the last section \ref{sec:concl}. 

\section{Models}
%\emph{Yet another view} ~\\
\label{sec:models}

As mentioned before, we focus in this study on two major representatives of the latent models used for link prediction in social networks, namely the latent feature model \cite{BMF} and the mixed-membership stochastic block model \cite{MMSB}. To be as general as possible, we consider non-parametric extensions of these models, respectively based on the Indian Buffet Process (IBP) and the Hierarchical Dirichlet Process (HDP). Similar extensions have already been considered in the past, {\it e.g.} through the Infinite Latent Feature model \cite{ILFRM} and through conditional random fields \cite{iMMSB} or a dynamic version of the Hierarchical Dirichlet Process \cite{diMMSB}.
%\textcolor{red}{To be completed - maybe second extension not considered yet}

We now briefly describe the two models retained.
%Our two chosen baseline use prior distributions that fall into the two major classes of discrete nonparametric priors. The Hierarchical Dirichlet Process (HDP) that generalizes the Latent Dirichlet Allocation (LDA) for infinite mixtures models. On the other hand, the Indian Buffet Process (IBP), which is the generalization of the Beta-Bernoulli compound distribution (ie Beta Process), which generates infinite binary matrices. The nonparametric models in their truncated version are equivalent to well-known models such as LDA, widely used for text analysis, and Mixed Membership Stochastic Blockmodel which is an adaptation of the latter for relational learning.~\\

%We adopt the following notation; if a matrix has a negative index superscripted, it indicates that the values corresponding to this index are excluded. A dot $\bm{.}$ in the index means that we marginalize over all possible values.

\subsection{Infinite Latent Feature Model (ILFM)}

In the latent feature model, each node is represented by a vector of binary features. The probability of linking two nodes is then based on a weighted similarity between their feature vectors, the weight matrix being generated according to a normal distribution. In its non-parametric version, the feature vectors are now generated according to an IBP, leading feature vectors of infinite dimensions (even though only a finite number of dimensions are actually active). The following steps summarizes this process:
%
\begin{enumerate}
\item Generate a feature matrix $\mat{F}_{N \times \infty}$ representing the feature vector of each node: $\mat{F} \sim \IBP(\alpha)$
\item Generate a weight matrix for each latent feature:\\
 $\mat{\phi}_{mn} \sim N(0, \sigma_w), \, m,n \in \mathbb{N}^{+*}$
\item Generate or not a link between any node $i$ and any node $j$ according to: 
%
\begin{equation}
y_{ij} \sim \mathrm{Bern}(\sigma(\mat{f}_{i} \mat{\Phi} \mat{f}_{j}^\top))
\label{eq:link-ilfm}
\end{equation}
\end{enumerate}
%
where $\sigma()$ is the sigmoid function, mapping $[-\infty, +\infty]$ values to [0,1], and where $y_{ij}$ is a binary variable indicating that a link has been generated ($y_{ij}=1$) or not ($y_{ij}=0$). We will denote by $\mat{Y}$ the $N \times N$ matrix with elements $y_{ij}$. Finally, $\mat{f}_{i}$ denotes the row vector corresponding to the $i^{th}$ row of $\mat{F}$.

This model makes use of two real hyper-parameters, one for the IBP process ($\alpha$), and one for the variance of the normal distribution underlying the weight matrix ($\sigma_w$). In the case of undirected networks, the matrices $\mat{Y}$ and $\mat{\Phi}$ are symmetric and only their upper (or lower) diagonal parts are generated. Lastly, both $\mat{F}$ and $\mat{\Phi}$ are infinite matrices. In practice however, one always deal with a finite number of latent features. A graphical representation of this model is given in Figure~\ref{fig:ilfrm}.

\begin{figure}[t]
	\centering
	\minipage{0.25\textwidth}\vspace{1cm}
	\scalebox{0.88}{
	\begin{tikzpicture}
  % Define nodes
  \node[obs]                      (y) {$y_{ij}$};
  \node[latent, left=1.2cm of y] (fi) {$\mat{f}_i$};
  \node[latent, right=1.2cm of y] (fj) {$\mat{f}_j$};
  \node[latent, above= of y]    (ibp) {$\mat{F}$};;
  \node[latent, below= of y, yshift=-0.3cm]   (W) {$\mat{\Phi}$};
  \node[const, left=0.7cm of ibp]   (a) {$\alpha$};
  \node[const, right=0.7cm of W]   (sw) {$\sigma_w$};

  % Connect the nodes
  \edge {fi,fj,W} {y} ;
  \edge[dashed] {ibp} {fi,fj} ;
  \edge {sw} {W} ; 
  \edge {a} {ibp} ; 

  % Plates
  \plate {yx} {(fj)(y)} {$N$} ;
  \plate[label={[label distance=-0.6cm]195:$N$}] {} {(fi)(y)(yx.north west)(yx.south west)} {} ;
  %\plate {} {(W)} {$K\times K$};
  %\plate {} {(fi)(y)(yx.north west)(yx.south west)} {$N$} ;
\end{tikzpicture}
}
	\endminipage
	\minipage{0.25\textwidth}
	\scalebox{0.88}{
		/home/dulac/Documents/workInProgress/networkofgraphs/papers/personal/figures/draw/mmsb2.tex}
	\endminipage
	\caption{The two graphical representations of (left) the latent feature model and (right) the latent class model. The difference between the two models lies in the way representations are associated to nodes: a fixed representation is used in the case of the latent feature model, whereas the representation in the latent class model varies according to the link considered.}
	\label{fig:ilfrm}
\end{figure}

Standard Gibbs sampling and Metropolis-Hastings algorithms can be used for inference in this model. We do not detail them here and refer the interested reader to \cite{ILFRM}.

%We here only provide the main updates, useful for the developments presented in the next sections, and refer the reader to \cite{IBP} for a detailed treatment. The Gibbs update for the matrix $\mat{F}$ are given by:
%%
%\begin{align}
%& P(f_{ik} = 1 \mid \mat{F}^{-ik}) = \frac{m_k^{-i}}{N} \nonumber \\
%& P(f_{ik} = 0 \mid \mat{F}^{-ik}) = 1 - \frac{m_k^{-i}}{N} \nonumber
%\end{align}
%%
%where $m_k^{-i}$ represents the number of active features $k$ for all nodes excluding node $i$, hence $m_k^{-i} = \sum_{j=1, j\neq i}^N f_{jk}$. $\mat{F}^{-ik}$ represents the matrix $\mat{F}$ without its element on the $i^{th}$ row and $k^{th}$ column.
%
%The learning of the weight matrix $W$ is computed using a Metropolis-Hasting algorithm in which each weight is sequentially sampled according to (\cite{IBP}): 
%%
%\begin{equation}
%P(\phi_{mn} \mid \mat{Y}, \mat{F}, \mat{\Phi}^{-mn}, \sigma_w) \propto P(\mat{Y} \mid \mat{F}, \mat{\Phi}) P(\phi_{mn} \mid \sigma_w) \nonumber
%\end{equation}
%%
%One can then choose a jumping distribution in the normal family (as for the prior), with a mean based on the previous sample:
%%
%\begin{equation} \label{eq:j_w}
%J(\phi_{mn}^* \mid \phi_{mn}) = \mathcal{N}(\phi_{mn}, \eta) \nonumber
%\end{equation}
%%
%where $\eta$ is a parameter controlling the acceptance ratio, $r_{\phi_{mn}\rightarrow \phi_{mn}^*}$, defined by:
%%
%\begin{equation} \label{eq:r_w}
%r_{\phi_{mn}\rightarrow \phi_{mn}^*} = \frac{ P(\mat{Y} \mid \mat{F}, \mat{\Phi}^*)P(\phi_{mn}^* \mid \sigma_w)J(\phi_{mn} \mid \phi_{mn}^*) }{ P(\mat{Y} \mid \mat{F}, \mat{\Phi})P(\phi_{mn} \mid \sigma_w)J(\phi_{mn}^* \mid \phi_{mn} )} \nonumber
%\end{equation}

\subsection{Infinite Mixed-Membership Stochastic Block Model (IMMSB)}

The MMSB model generates class membership distributions per node on the basis of a Dirichlet distribution. Then, for each connection between two nodes, a particular class for each node is first sampled from the class membership distribution, and the probability of connecting the two nodes is, as in the previous model, based on a Bernoulli distribution integrating the weight of the two classes. 

The non-parametric version parallels this development but considers, in lieu of the Dirichlet distribution, a hierarchical Dirichlet process, leading to the following generative model:
%
\begin{enumerate}
\item Generate the class membership matrix $\mat{F}_{N \times \infty}$:
   \begin{align}
    &\bm{\beta} \sim \gem(\gamma) \nonumber \\
    \mat{f}_i &\sim \DP(\alpha_0, \beta) \quad\text{ for }  i \in \{1, .., N\} \nonumber
   \end{align}
where $\gem$ denotes the Griffiths, ??? distribution over the set of natural numbers and $\DP$ a Dirichlet Process  \cite{HDP}.
\item Generate a weight matrix for each latent class:\\
\[ \phi_{mn} \sim \mathrm{Beta}(\lambda_0,\lambda_1), \, m,n \in \mathbb{N}^{+*} \]
\item For any node $i$ and any node $j$, choose a class from their class membership distribution and generate or not a link according to:
   \begin{align}
    z_{i \rightarrow j} &\sim \mbox{Cat}(\mat{f}_i) \nonumber \\
    z_{i \leftarrow j} &\sim \mbox{Cat}(\mat{f}_j) \nonumber \\
    y_{ij} &\sim \mathrm{Bern}(\phi_{z_{i \rightarrow j}f_{i \leftarrow j}})
    \label{eq:link-immsb}
   \end{align}
\end{enumerate}
%
We have this time four real hyper-parameters, two for the hierarchical Dirichlet process ($\gamma$ and $\alpha_0$) and two for the Beta distribution underlying the weight matrix ($\lambda_0$ and $\lambda_1$). As for the previous model, in the case of undirected networks, the matrices $\mat{Y}$ and $\mat{\Phi}$ are symmetric and only their upper (or lower) diagonal parts are generated; as before again, both $\mat{F}$ and $\mat{\Phi}$ are infinite matrices. A graphical representation of this model is given in Figure~\ref{fig:ilfrm}.

The inference is this model can be performed via collapsed Gibbs sampling updates. Most updates can be found in \cite{HDP} and \cite{diMMSB}. For completeness, we provide them in Appendix~\ref{sec:append}.
%%
%\begin{enumerate}
%\item If the class $k$ has already been observed:
%   \begin{align}
%    \pr(z_{ij} =k \mid \mat{F}^{-ij}) &\propto N_{ik}^{-ij} + \alpha_0 \beta_k
%%    \pr(f_{ji} =k \mid \mat{F}^{-ij}) &\propto N_{jk}^{-ij} + \alpha_0 \beta_k
%    \label{eq:update-immsb}
%   \end{align}
%\item In case of a new class $k_n$:
%   \begin{align}
%    \pr(z_{ij} =k_n \mid \mat{F}^{-ij}) &\propto \alpha_0 \beta_{k_n} \nonumber
%%    \pr(f_{ji} =k_n \mid \mat{F}^{-ij}) &\propto \alpha_0 \beta_{k_n} \nonumber
%   \end{align}
%\end{enumerate}

\subsection{Model Comparison}

Both ILFM and IMMSB are based on a latent representation of each node in the network (matrix $\mat{F}$). However, in the case of ILFM, this representation takes the form of a binary vector, whereas in IMMSB it is a vector of proportion. Interpreting the latent dimensions as characteristics or classes, one can view ILFM as performing a hard assignment on those classes whereas IMMSB performs a soft assignment. 

Another distinction between the two lies in the use of the sigmoid function in ILFM to obtain the parameter of a Bernoulli distribution from the latent representations and their weights (or correlations). Because of that, the weight matrix $\mat{\Phi}$ can take on a very general form in this model and can easily be generalized to a multivariate distribution. This is not the case in IMMSB where the elements of $\mat{\Phi}$ should lie in the interval $[0;1]$. 

Lastly, a major difference lies in the fact that the complete latent representation of each node is used in ILFM to generate a link, whereas in IMMSB, for each link to be generated, one first selects one component from the latent representations of the nodes involved in the link. This allows one to capture the fact that different classes may explain different links for the same node. At the same time, it has an impact on the homophily effect, as shown in the next section. Note that ILFM is also able to capture the fact that different classes may explain different links for the same node, even though more implicitly, by relying on different dimensions of the latent representations.

\textcolor{red}{Say a few words on the complexity and inference time of each model}

\section{Inference Context}

In a Bayesian context, the learning process consists in finding the posterior distribution of the random parameters $F$ and $\Phi$, given the observed data $Y$, such that : 

\begin{equation}
    \p(F, \Phi | Y, \M_g) = \frac{\p(Y|F,\Phi)\p(F|\M_g)\p(\Phi|\M,g)}{\p(Y|\M_g)}
\end{equation}

Where $\M_g$ is the set the hyperparamters of the current model.


For mixed membership models the evidence $\p(Y|\M_g$ has no closed form solution which makes a direct MAP inference procedure infeasible. Thus the learning process rely on an approximate inference. It consists in a iterative procedure that updates the posterior through typically MCMC updates for true posterior recovery.

Typically MCMC update consists in the following updates of parameters :  

\begin{align}
    \hat f_{ik} &\sim \p(f_{ik} | F^{-ik}, Y, \M_g) \\
    \hat \phi_{kk'} &\sim \p(\phi_{kk'} | \Phi^{-kk'}, Y, F, \M_g)
\end{align}

At the end of the inference process, assuming that the MCMC has reached its  equilibrium, one can reconstruct the posterior parameters such that $\hat F = (\hat \phi_{ik})_{i,k \in V\times[0,.., K-1]}$ and $\hat \Phi = (\hat \phi_{kk'})_{k,k' \in [0,.., K-1]^2}$ and :

\begin{equation}
    \p(F, \Phi|Y, \M_g) \approx \p(\hat F) \p(\hat \Phi)
\end{equation}

Finally the prediction task consists of measuring an unobserved variable $y^{new}$ given that the information from the observed data was transferred to the posterior distribution : 

\begin{align*}
    \p(y^{new} | Y, \M_g) &= \int_F \int_\Phi \p(y^{new}|F,\Phi) \p(F,\Phi|\M_g) dF d\Phi \\
                          &\approx \E_{\hat F, \hat \Phi} [y^{new} | \hat F, \hat \Phi] = \mathrm{Bern}(\mathcal{K}(f_i \Phi f_j^T)) \\
                          &= \p(y^{new} | \M_e)
\end{align*}

Where $\mathcal{K}$ is an isomorphism used to map the support of the bilinear product to a probability space, wich is a sigmoid and the identity for respectively ILFM and IMMSB. Furthermore we denote the set of estimated parameters $\M_e = \{\hat F,  \hat \Phi \}$.


In the rest of this paper, we will refers to both context in which we study a Bayesian model :
\begin{itemize}
    \item Predictive model $\M_e = \{\hat F, \hat \Phi\}$
    \item Generative model $\M_g = \{\alpha_0, \alpha, \lambda \}$
\end{itemize}

\paragraph{Remark (Diaconis-Ylvisaker characterisation of conjugate priors.}~\\
In the case of conjugate distribution, the Diaconis-Ylvisaker theorem give insight about the form of the predictive distribution $\p(y^{new} | \M_e)$ \cite{orbanz2009functional}.

\begin{theorem}[Diaconis-Ylvisaker characterisation of conjugate priors]
Let $P_x(.|\Theta)$ be a natural exponential family model dominated by Lebesgue measure, with open parameter space $\Omega_\theta \subset \mathbb{R}^d$.
    Let $P_\theta$ b a prior on $\Theta$ which does not concentrate on a singleton. Then $P_\theta$ is a  conjugate prior of $P_X$  w.r.t Lebesgue measure on $\mathbb{R}^d$ if and only if :

\begin{equation}
    \E_{P_\Theta(\Theta|X_1=x_1,...,X_n=x_n)} [\E_{P_X(x|\Theta=\theta)}[X]] = \frac{y+n\hat x}{a+n}
\end{equation}

\end{theorem}

That is, given observation $x_1,...,x_n$, the expected value of a new draw $x$ under unknown value of the parameter is linear in the sample average $\hat x = \frac{1}{n}\sum x_i$.


\section{Preferential attachment}
\label{sec:burstiness}

As mentioned before, preferential attachement can be global, in which case nodes are connected across communities, and/or local to the network communities. Preferential attachment is reminiscent of a phenomenon called \textit{burstiness}, studied in different contexts (\cite{barabasi_burst}). We introduce here definitions for the local and global preferential attachment effects that are extensions of the definitions for burstiness proposed in \cite{clinchant2010information} for text collections. We will first study global preferential attachment for the models \ifm\ and \imb\ in the two contexts defined by $\mathcal{M}_g$ and $\mathcal{M}_e$. We will then turn our attention to local preferential attachment.
%The preferential attachment effect can be observed directly over the global network, or indirectly on latent classes.

\subsection{Global preferential attachment}

Probabilistic models naturally lead to the following generative process for creating links between nodes in a network\footnote{For simplicity in the notation, we consider that nodes can be linked to themselves. Excluding such links does not raise particular problems.}. This process considers all possible pairs of nodes in turn and generates or not a link between them:

\begin{description}
 \item[1.] \textit{For each node $i \in \{1, \dotsc, N\}$},
 \begin{description}
    \item[2.] \textit{For each node $j \in \{1, \dotsc, N\}$},
       \begin{description}
          \item[3.] \textit{Generate a link between $i$ and $j$ with probability $P(y_{ij}=1 | \mathcal{M})$ where $\mathcal{M}$ is either $\mathcal{M}_e$ or $\mathcal{M}_g$}.
       \end{description}
  \end{description}
\end{description}

%\begin{algorithm}[t!]
%\caption{Deep $k$-Means algorithm}
%\label{algo:train}
%\DontPrintSemicolon
%    \KwIn{data $\mathcal{X}$, number of clusters $K$, balancing parameter $\lambda$, scheme for $\alpha$, number of epochs $T$, number of minibatches $N$, learning rate $\eta$}
%    %$\mathcal{X}, \lambda, \eta, T, m_{\alpha}, M_{\alpha}$}
%    \KwOut{autoencoder parameters $\theta$, cluster representatives $\mathcal{R}$}
%    %$\theta$, $\mathcal{R}$}
%	%\STATE{\textbf{Random initialization of} $\theta$ \textbf{and} $\vect{r}_k, \, 1 \le k \le K$}
%	Initialize $\theta$ and $\vect{r}_k, \, 1 \le k \le K$ (randomly or through pretraining)\;
%	\For(\Comment*[f]{inverse temperature levels}){$\alpha = m_{\alpha}$ to $M_{\alpha}$} {
%		\For(\Comment*[f]{epochs per $\alpha$}){$t=1$ to $T$} {
%			\For(\Comment*[f]{minibatches}){$n=1$ to $N$} {
%				Draw a minibatch $\tilde{\mathcal{X}} \subset \mathcal{X}$\;
%				Update $(\theta, \, \mathcal{R})$ using SGD (Eq.~\ref{eq:update})\;
%			}
%		}
%	}
%\end{algorithm}

As one can note, this process considers all nodes in turn, from node 1 to node $N$. An indexing, \textit{i.e.} a mapping between nodes and integers in $\{1,\cdots,N\}$, is however arbitrary and conclusions drawn from the above process should be independent of the indexing. As we will see, the results we establish below are indeed independent of the indexing.

For a given node $i$ at step $p$ of the above process, $p$ nodes, from node 1 to node $p$, have been considered and links from these nodes to node $i$ generated or not. We will denote by $d_i^{(p)}$ the degree of node $i$, i.e. the number of links of node $i$, at the $p^{th}$ step of this process. By definition:
%
\begin{equation} \label{eq:degree_def}
d_i^{(p)} = \sum_{j=1}^p y_{ij}
\end{equation}
%
As mentioned before, preferential attachment characterizes the propensity of nodes in social networks to connect to nodes that already have a lot of connections and can be stated as \textit{the higher the number of links a node has, the more likely it will get new links}. The following definition directly captures this idea:
%
\begin{definition}[Global preferential attachment]
In the above setting, a probabilistic model satisfies the global preferential attachment effect iff for any indexing, for any node $i, \, 1 \leq i \leq N$, for any $p, \, 1 \leq p < N$, $P(d_i^{(N)} \geq n+1 | d_i^{(p)} = n; \M)$ increases with $n$ ($1 \leq n < p$). If $P(d_i^{(N)} \geq n+1 | d_i^{(p)} = n; \M)$ is independent of $n$, the model is said to be neutral \textit{w.r.t.} the global preferential attachment effect. As before, $\mathcal{M}$ is either $\mathcal{M}_e$ or $\mathcal{M}_g$.
\end{definition}
%
Thus, a model satisfies the global preferential attachment effect if and only if the more links a node $i$ has at some point in the process, the more likely a new link will be created with that node.

For both \ifm\ and \imb, in $\M_e$, the generation of links are independent of each other. The fact that $n$ links have been created after $p$ steps has thus no impact on the future links to a given node. In $\M_g$, as one first needs to generate $F$ and $\Phi$ prior to generate all the links, a similar behavior is likely to be observed. Intuitively thus, both \ifm\ and \imb\ are neutral wrt the global preferential attachment effect. The following property formalizes this intuition.
%
\begin{proposition} \label{th:mg_glob}
Both \ifm\ and \imb, for both $\M_e$ and $\M_g$, are neutral wrt the global preferential attachment effect.
\end{proposition}
%
\begin{proof}
We first consider model $\M_e$. Fix any indexing, a node $i$, $i \leq i \leq N$, and a step $p$, $1 \leq p < N$. One has, $\forall n, 1 \leq n < p$ :
%
\begin{align*}
P(d_i^{(N)} \geq n+1 | d_i^{(p)}=n, \M_e) &= 1 - P(d_i^{(N)}=n | d_i^{(p)}=n, \M_e) \\
        &= 1 -P(y_{i,p+1}=0, \dotsc,y_{iN}=0 | \M_e ) \\
        &= 1 - \prod_{j=p+1}^N P(y_{ij}=0 | \M_e)
\end{align*}
%
where the last equality comes from the fact that, in $\M_e$, links are independently generated. Similarly:
%
\begin{align*}
P(d_i^{(N)} \geq n+2 | d_i^{(p)}=n+1, \M_e) &= 1 - \prod_{j=p+1}^N P(y_{ij}=0 | \M_e) \\
                    &=P(d_i^{(N)} \geq n+1 | d_i^{(p)}=n, \M_e)
\end{align*}
%
which shows that both \ifm\ and \imb\ are neutral wrt to global preferential attachment with $\M_e$.

For $\M_g$, it suffices to observe that the above result holds for all $\mat{F}$ and $\mat{\Phi}$, and not only for $\mat{\hat{F}}$ and $\mat{\hat{\Phi}}$, so that:
%
\begin{equation*}
P(d_i^{(N)} \geq n+1 | d_i^{(p)}=n, \M_g)  = \int_{\mat{F},\mat{\Phi}} P(\mat{F},\mat{\Phi}|\M_g) P(d_i^{(N)} \geq n+1 | d_i^{(p)}=n, \mat{F},\mat{\Phi}) \ d\mat{F} d\mat{\Phi}
\end{equation*}
%
As the models are neutral with $(\mat{F},\mat{\Phi})$, $P(d_i^{(N)} \geq n+1 | d_i^{(p)}=n, \mat{F},\mat{\Phi}) = P(d_i^{(N)} \geq n+2 | d_i^{(p)}=n+1, \mat{F},\mat{\Phi})$ and thus:
%
\begin{equation*}
P(d_i^{(N)} \geq n+2 | d_i^{(p)}=n+1, \M_g) = P(d_i^{(N)} \geq n+1 | d_i^{(p)}=n, \M_g)
\end{equation*}
%
which completes the proof.
\end{proof}

\vspace{0.1cm}
We now turn to local preferential attachment that deals with the fact that preferential attachment can be also observed within classes of nodes, as exemplified in \cite{LeskovecBKT08}. The classes we consider here are the latent classes of the stochastic mixed-membership models.

\subsection{Local preferential attachment}
\label{sec:local_me}

Local preferential attachment is a restriction of global preferential attachment at the community level and aims at capturing the fact that the more links a node has in a given community, the more links it will have in the future within this community. The latent classes used in \ifm\ and \imb\ play the role of latent communities gathering nodes sharing unobserved properties. Local preferential attachement can thus be studied in stochastic mixed membership models by studying how preferential attachement is captured within latent classes. This nevertheless entails that the latent classes be set in one way or another, meaning that the question of whether stochastic mixed membership models comply with the local preferential attachment effect only makes sense in $\me$, and not in $\mg$.

For \textbf{\ifm}, the situation wrt to local preferential attachment is very similar to the one for global preferential attachment. This is due to the fact that, in $\M_e$ (i.e. given $F$ and $\Phi$), a local degree can be defined in the same way as the global degree above.

Considering the same generative process as before, for $\M_e$ and \ifm, the local degree in class $k$ ($0\leq k\leq K-1$), for a node $i$ such that $f_{ik}=1$, is defined by:
%
\begin{equation*}
d_{i,k}^{(p)} = \sum_{j=1, f_{jk}=1}^p y_{ij}
\end{equation*}
%
Note that if $f_{ik}=0$, $d_{i,k}^{(p)} = 0$ for all $p$. This then leads to the following definition for local preferential attachment with \ifm.
%
\begin{definition}[\ifm\ - local preferential attachment, $\M_e$]\label{def:locdeg-discrete}
We say that \ifm, in $\M_e$, satisfies the local preferential attachment effect iff for any indexing, for any node $i, \, 1\leq i \leq N$ such that $f_{ik}=1$, and for any step $p, \, 1\leq p < N$, $P(d_{i,k}^{(N)} \geq n+1 | d_{i,k}^{(p)}=n,\M_e)$ increases with $n$ ($1\leq n < p$). If $P(d_{i,k}^{(N)} \geq n+1 | d_{i,k}^{(p)}=n,\M_e)$ is independent of $n$, the model is said to be neutral wrt to the local preferential attachment effect.
\end{definition}
%
As before, we have the following property.
%
\begin{proposition}
\ifm, with $\M_e$, is neutral \textit{w.r.t.} the local preferential attachment effect.
\end{proposition}

\begin{proof}
The proof is identical to the first part of the proof of Property \ref{th:mg_glob}.
\end{proof}

\vspace{0.1cm}
For \textbf{\imb} in $\M_e$, we do not have a direct access to classes, encoded in the $Z$ variables. One can nevertheless define  local random variables $y_{ij,k}$ that are 1 if a link is generated between nodes $i$ and $j$ within class $k$ and 0 otherwise. One has:
%
\begin{align*}
P(y_{ij,k}=1 | \M_e) &= P(y_{ij}=1 | z_{i\rightarrow j} = z_{i\leftarrow j}=k, \Phi) P(z_{i\rightarrow j}=k|F)P(z_{i\leftarrow j}=k|F) \\
    &= f_{ik} \phi_{kk} f_{jk}
\end{align*}
%
The local degree $d_{i,k}^{(p)}$ can then be defined as the expectation of $y_{ij,k}$ over the nodes $1,\dotsc,p$:
%
\begin{align}\label{def:locdegimb}
d_{i,k}^{(p)} &= \sum_{j=1}^p P(y_{ij,k}=1 | \M_e)  \nonumber \\
    &= \sum_{j=1}^p f_{ik} \phi_{kk} f_{jk}
\end{align}
%
As one can note, such a local degree is not necessarily an integer and the definition of the local preferential attachment has to be adapted accordingly. 
%
\begin{definition}[\imb\ - local preferential attachment, $\M_e$]
We say that \imb, in $\M_e$, satisfies the local preferential attachment effect iff for any indexing, for any node $i, \, 1\leq i \leq N$ such that $f_{ik}>0$, for any step $p, \, 1\leq p < N$, and  for all $\epsilon$ compatible with the domain of definition of $d_{i,k}$ and $x$, $P(d_{i,k}^{(N)} \geq x+\epsilon | d_{i,k}^{(p)} \geq x,\M_e)$ increases with $x$. If $P(d_{i,k}^{(N)} \geq x+\epsilon | d_{i,k}^{(p)} \geq x,\M_e)$ is independent of $x$, the model is said to be neutral wrt to the local preferential attachment effect.
\end{definition}

This definition can be seen as the continuous counterpart of Definition~\ref{def:locdeg-discrete}. If $\epsilon$ is too large, the probability is null and is independent on $x$, hence the compatibility requirement with the domain of definition of $x$ and $d_{i,k}$. 

Because of the Hierarchical Dirichlet Process underlying the \imb\ model, $\mat{f}_i$ follows a Dirichlet distribution: $\mat{f}_i \sim \Dir((\alpha_0 \beta_k + N_{ik})_{1 \le k \le K})$, with $\mat{\beta}\sim \gem(\gamma)$ and $N_{ik}$ being the number of edges connecting node $i$ through class $k$ (see for example \cite{teh2006hierarchical}) and $K$ the number of latent classes obtained. The marginals $f_{ik}$ are thus distributed according to a Beta distribution: $f_{ik} \sim \Beta(a_{ik}, b_{ik})$ with $a_{ik} = \alpha_0\beta_k + N_{ik}$ and $b_{ik} = \sum_{k'=1, k' \ne k}^{K} \alpha_0\beta_k' + N_{ik'}$.

The following property displays a sufficient condition on $x$, $\epsilon$, $a_{ik}$ and $b_{ik}$ for \imb\ to satisfy the local preferential attachment.

\begin{proposition}\label{prop:IMBlocal}
%Let $F_k^p = \sum_{j=1}^p \hat{f}_{jk}$, $x'=\frac{x}{F_k^p \phi_{kk}}$ and $\epsilon'=\frac{\epsilon}{F_k^N \phi_{kk}}$. In the region where $x$ and $\epsilon$ are such that:
%\[ F_k^N a_{ik} x'^{a_{ik}-1} (1-\epsilon') > F_k^p a_{ik} x'^{a_{ik}} + b_{ik} F_k^p (1-x'^{a_{ik}}), \]
Let $F_k^p = \sum_{j=1}^p \hat{f}_{jk}$ and $E_{d_{ik}}=\sum_{j=1}^N f_{ik}\phi_{kk}f_{jk}$. In the regions where $x$ and $\epsilon$ are such that
\begin{equation*}
x^{a_{ik}-1}\frac{f_{ik}}{E_{d_{ik}}}\left(a_{ik}(1-\epsilon) -x(a_{ik}-b_{ik}) \right) \geq b_{ik}\frac{(F_k^p)^{a_{ik}}\phi_{kk}^{a_{ik}-1}}{F_k^N}
\end{equation*}
$P(d_{i,k}^{(N)} \geq x+\epsilon | d_{i,k}^{(p)} \geq x,\M_e)$ increases with $x$.
\end{proposition}

%As one can note, when $F_k^p$ is small (typically in the first steps of the process), then the above condition is likely to be met and \imb\ satisfies the preferential attachment effect. However, when $F_k^p$ gets closer to $F_k^N$ the above condition is no longer met.
As one can note, when $x$ is large and $\epsilon$ is small $b_{ik}-a_{ik}>0$ (which roughly means that the class $k$ concentrates less that than half of the capacaity of the network) then the above condition is likely to be met and \imb satisfies the local preferential attachment effect. 

\begin{proof}

We consider IMMSB in $\me$. Let $F_k^P=\sum_{j=1}^p \hat f_{jk}$ and $F_k^N=\sum_{j=1}^N \hat f_{jk}$. 
Using the change of variables $x'=\frac{x}{F_k^p \phi_{kk'}}$ and $\epsilon' = \frac{\epsilon}{F_k^N \phi_{kk'}}$, one gets
\begin{align*}
p(d_{ik}^{(N)} > x+\epsilon | d_{ik}^{(p)} > x, \me) &= p(f_{ik} > qx'+\epsilon' | f_{ik} > x') \\
&=\frac{P(\hat f_{ik} > qx'+\epsilon')}{P(\hat f_{ik} > x')} = g(x')
\end{align*}
where $q=\frac{F_{ik}^p}{F_{ik}^N}$. The conditional distribution $g(x')$ is not trivially equal to one in the case where $qx'+\epsilon' > x' \Leftrightarrow \epsilon' > x'(1-q)$. Further, the survival function of $\hat f_{ik}$ is $P(\hat f_{ik} >x') = 1-\int_0^{x'} f(x')$ where $f(x')$ is the density of $\hat f_{ik}$. One can show that the marginal distribution of $\hat f_{ik}$ is a Beta of the form $f(x')=\Beta(a_{ik}, b_{ik})$  where $a_{ik} = \alpha_0\beta_k + N_{ik}$ and $b_{ik} = \sum_{k'\neq k} \alpha_0\beta_{k'} + N_{ik'}$  (this is a consequence of the form of the posterior distribution of the DP). In the following we ommit the references to $i$ and $k$ for the parameters of the Beta, simply noting $\Beta(a, b)$ for short. The derivative of $g$ is
\begin{equation*}
g'(x') = \frac{-q (qx'+\epsilon')^{a-1}(1-qx'-\epsilon')^{b-1}\int_{x'}^1t^{a-1} (1-t)^{b-1} dt + x'^{a-1}(1-x')^{b-1}\int_{qx'+\epsilon'}^1 t^{a-1}(1-t)^{b-1} dt}{\left(\int_{x'}^1t^{a-1}(1-t)^{b-1}dt\right)^2}
\end{equation*}
But one has
\begin{equation*}
\int_{qx'+\epsilon'}^1 t^{a-1}(1-t)^{b-1}dt \geq (qx'+\epsilon')^{a-1}\int_{qx'+\epsilon'}^1 (1-t)^{b-1}dt = (qx'+a)^{a-1}\frac{(1-qx'-\epsilon')^b}{b}
\end{equation*}
and
\begin{equation*}
\int_{x'}^1 t^{a-1}(1-t)^{b-1}dt \leq (1-x')^{b-1}\int_{x'}^1 t^{a-1}dt = (1-x')^{b-1}\frac{(1-x'^a)}{a}
\end{equation*}
Thus one can show that
\begin{align*}
Cg'(x') &\geq x'^{a-1} \frac{1-qx'-\epsilon'}{b} -q\frac{1-x'^a}{a}  \\
        &= \frac{1}{abF_k^N} \left[ ax'^{a-1}(1-\epsilon')F_k^N + F_k^P (b(x'^a-1) - ax'^a) \right]
\end{align*}
where $C=\frac{\left(\int_{x'}^1t^{a-1}(1-t)^{b-1}dt\right)^2}{(qx'+\epsilon')^{a-1}(1-qx'-\epsilon')^{b-1}(1-x')^{b-1}}$, is a positive constant. Thus, A sufficient condition for $g$ to be increasing is that
\begin{align*}
&F_k^N a x'^{a-1} (1-\epsilon') \geq F_k^p a x'^a + b F_k^p (1-x'^a) \\
&\Leftrightarrow x'^{a-1} a(1-\epsilon') \geq \frac{F_k^p}{F_k^N} ( x'^a(a-b) + b) \\
&\Leftrightarrow x'^{a-1} \left(a(1-\epsilon') - x'(a-b)\frac{F_k^p}{F_k^N} \right) \geq b \frac{F_k^p}{F_k^N}
\end{align*}
Finally, let $E_{d_{ik}}=\sum_{j=1}^N f_{ik}\phi_{kk}f_{jk}=f_{ik}\phi_{kk'}F_k^N$, by rolling back the variable changes,  one obtains
\begin{equation*}
x^{a-1}\frac{f_{ik}}{E_{d_{ik}}}\left(a(1-\epsilon) -x(a-b) \right) \geq b\frac{(F_k^p)^{a}\phi_{kk}^{a-1}}{F_k^N}
\end{equation*}


\end{proof}

%\vspace{0.1cm}
% %The power law distribution assumed here for $\hat{f}_{ik}$ directly translates the positive reinforcement effect of the Dirichlet Process \cite{HDP} at the basis of \imb, that corresponds to a burstiness phenomenon. 
%The definitions that we have considered for the (local and global) preferential attachment effects are extensions of the definitions for burstiness proposed in \cite{clinchant2010information}. Power law distributions have been often used to model bursty phenomena (\cite{barabasi_burst,clauset2009power} to cite but a few), and our assumption on $\hat{f}_{ik}$ reflects this usage.
% % It is easy to see, using the same development as the one we have used, that power distributions are bursty in the sense of \cite{clinchant2010information}.

We now present an experimental illustration of the above theoretical results.





\section{Conclusion}
\label{sec:concl}

We have studied whether stochastic mixed membership models, such as \ifm\ and \imb\, can generate new links while satisfying properties frequently verified in real  social networks, namely homophily and preferential attachment. To do so, we have introduced formal definitions of these properties and have analyzed how these models behave according to those definitions. We have shown, in particular, that both models are \textit{homophilic} with the natural similarity that underlies them. Concerning preferential attachment, we have shown that stochastic mixed membership models do not comply with global preferential attachment. The situation is however more contrasted when the property is considered at the local level: \imb\ enforces local preferential attachment whereas \ifm\ does not.~\\

These findings have been validated experimentally on two real and two artificial networks that have different degrees of global and local preferential attachment. An important, practical finding of our study is that \imb, usually considered of lesser "quality" than \ifm, can indeed yield better results on bursty networks (\textit{i.e.} networks with preferential attachment) when the number of training data is limited.~\\

There are many directions to extend this work with the motivation of improving our theoretical understanding of graphical models for link prediction in complex networks. A straightforward extension is to examine the relation between the local preferential attachment and the dynamic of the latent classes.  
%For example, some special value of the hyperparameters, of the non-parametric priors, could lead to a very large number of classes or, at the opposite, just one class. Between these two extremes, that goes from a vanishing local aspect of the degree distribution to a number of classes that overfit the data, on can ask how this parameter affect the global and the local degree distribution of a random graph.  
For instance, a fundamental result is the Aldous-Hoover theorem, which implies that exchangeable random graphs cannot be sparse \cite{orbanz2015bayesian}. It seems that the sparsity is related in some way to the preferential attachment in a network. Thus, the following question arises: would it be realistic to assume the exchangeability hypothesis for the local case but not for the global case, and how this fact impacts the burstiness of the global degree distribution and the sparsity of the graph.

We believe that answering to those questions open a way to develop and design Bayesian models able to better capture the fundamental properties of  real social networks.


~\\


\bibliographystyle{chicago}
\bibliography{./a}

\appendix
\section{Appendix}
\label{sec:append}

\subsection{Mixed membership Models}
\label{sec:mixmembership}
In the Mixed Membership Models \cite{MMM}, the models can be defined at the link level by the likelihood of generating a link between two nodes given the contribution of each classes (or features). For IMMSB, this likelihood is straightforward, but for ILFM the class membership is defined deterministically by the binary vector $\mat{f}$. If the $k^{th}$ row is active (equal to one) then the node has the membership, else it doesn't. Hence for ILFM, We can write the likelihood  using the Dirac distribution $\delta(x)$ that gives one for $x=0$ as follows:

\begin{align}
    \pr(y_{ij} \mid \mat{F}, \mat{\Phi } ) &= \sigma \sum_{k, k'} \pr(y_{ij}\mid\phi_{k,k'}) \pr(k \mid \mat{f}_i) \pr(k' \mid \mat{f}_j) \\
    &=  \sigma \sum_{k, k'} \phi_{k,k'}  \delta(1-f_{ik}) \delta(1-f_{jk'})
    \end{align}
    

\subsection{Collapsed Gibbs sampling updates for IMMSB}

We provide here the derivation of the updates of the IMMSB model, described in Section~\ref{sec:models}.

%From the definition of the model, one has: $\pr(z_{ij} = k \mid \mat{f}_i) = f_{ik}$.

%\textcolor{red}{Adrien, peux-tu donner la d\'erivation ? La forme actuelle n'est valable que pour MMSB.} 

%heeeere \alpha is \alpha_0

Inference for the IMMSB model by using the Collapse Gibbs sampler gives updates for class assignment $Z \in N\times N \times 2$ for each interactions $Y \in N\times N$. Thus for all pair of interaction (i,j) we jointly sample the classes $(z_{ij}, z_{ji})$ who implicitly, take the values $(k,k')$ :
\begin{align} \label{eq:cgs}
&\pr(z_{ij}, z_{ji} \mid Z^-, Y,  \mat{\beta}, \alpha, \mat{\lambda} )  \\
&\propto\pr(z_{ij}, z_{ji} \mid Z^-, \alpha,\mat{\beta}) \pr(y_{ij} \mid Y^{-ij},  Z^-,z_{ij}, z_{ji},  \mat{\lambda} ) \nonumber
\end{align}
The term $Z^-$ denote that both $z_{ij}$ and $z_{ji}$ are exclude from $Z$. We now treat the first term of equation \ref{eq:cgs}.  
\begin{align}
& \pr(z_{ij}, z_{ji} \mid Z^-, \alpha,\mat{\beta})\\
&\propto \pr(z_{ij} \mid \mat{z}_i^{-j}, \mat{z}_j, \alpha,\mat{\beta})  \pr(z_{ji} \mid \mat{z}_j^{-i}, \mat{z}_i, \alpha,\mat{\beta}) \nonumber
\end{align}
Let's consider the density of $z_{ij}$:
\begin{align}
&\pr(z_{ij} \mid \mat{z}_i^{-j}, \mat{z}_j, \alpha,\mat{\beta}) \propto \pr(z_{ij},  \mat{z}_i^{-j}, \mat{z}_j, \alpha,\mat{\beta}) \\
&= \int_{f_i} \pr(f_i \mid \mat{\beta}, \alpha) \pr(z_{ij} \mid f_i) \prod_{j_0\neq j} \pr(z_{ij_0} \mid f_i) \prod_{j_0 =  1}^N  \pr(z_{j_0 i} \mid f_i)  df_i \nonumber
\end{align}


Due to the an augmented representation of the Chinese Restaurant Franchise (CRF) with the Stick Breaking Process \cite{HDP}, the density of the features can be approximated by the following Dirichlet distribution;
\begin{equation}
f_i \mid \mat{\beta}, \alpha \sim Dir(\alpha \beta_1,..,\alpha\beta_K, \alpha\beta_{new})
\end{equation}
Where $\alpha\beta_{new}$ represent the contribution for sampling a new class. Since $\pr(z_{ij} \mid f_i)$ is drawn from a multinomial, the model is said to be conjugate and reduce to a simple closed form expression:
\begin{enumerate}
\item If the class $k$ has already been observed:
   \begin{align}
    \pr(z_{ij} =k \mid .) &\propto N_{ik}^{-ij} + \alpha_0 \beta_k
    \label{eq:update-immsb}
   \end{align}
\item In case of a new class $k_{new}$:
   \begin{align}
    \pr(z_{ij} =k_{new} \mid.) &\propto \alpha_0 \beta_{new} \nonumber   
   \end{align}
\end{enumerate}
 Where  $N_{ik}$ is the count for node $i$ being assigned to class $k$. As we show that the equations are symmetric, sampling for $z_{ji}$ is straightforward.

~\\
Again, referring the CRF, the sampling of the tables configuration $\mat{m}$ is given by: 
\begin{equation}
\pr(m_{ik} \mid Z, \bm{m}^{-ik}, \mat{\beta} ) = \frac{\Gamma(\alpha_0 \beta_k)}{\Gamma(\alpha_0 \beta_k + n_{j\bm{.   }k})} s(n_{j\bm{.}k}, m) (\alpha_0 \beta_k)^m
\end{equation}
And, finnaly  $\mat{\beta}$ is obtained by:
\begin{equation}
\mat{\beta} \sim Dir(m_1,.., m_K, \gamma)  
\end{equation}
Where $s(n,m)$ is the unsigned Stirling number of the first kind.


~\\
Finally, when the markov chain reach the stationnary distribution, the models parameters $\M = \mat{Phi}, \mat{F}$ can be recovered by averaging the topics assignement counts for each membership and each relation:
\begin{align}
&\pr(f_{ik}) =\frac{ N_{ik} + \alpha\beta_k}{ N_{i\bm{.}} + \sum_k\alpha_k }\\
&\pr(\phi_c ) = \frac{M_{c1} + \lambda_1}{M_{c\bm{.}} + \lambda_0 + \lambda_1}
\end{align}


The count for node $i$ being assigned to membership $k$ is $N_{ik}$. And the count of for all couple of classes $c=(k,k')$ being associated to relation $r$ is $M_{cr}$. Note that in our case, the relation $r$ take values in (0,1) accounting for link or non-link between two node.


\end{document}
