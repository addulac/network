\section{Appendix}

\begin{proof}[Proof of proposition \ref{th:me2mg}]
for a given distribution $P$ under a model $mg$ one has : 
if a distribution $P$ is bursty in $\mg$ : one has
\begin{equation*}
    p(y \geq n+1 | y \geq n, \mg) = \int_{\me} P(y \geq n+1 | y \geq n, \me) P(\me) d\me
\end{equation*}

Let's denote the discrete derivate of a function $f$  as $\Delta_n f(n) = f(n+1) -f(n)$.

Concerning the burstiness of $P$ one has : 
\begin{equation}
    \Delta_n P(y \geq n+1 | y \geq n, \mg) = \int_{\me}\Delta_n P(y \geq n+1 | y \geq n, \me) P(\me)d\me
\end{equation}

As $P(\me)$ is strictly positive, the proposition is straightforward.
\end{proof}



\begin{proof}[Proof of proposition \ref{th:ldegree_exp_me}]
For IMMSB : ~\\

One as 
$D_{ik} = \sum_{j\in\mathcal{V}} \hat f_{ik} \hat \phi_{kk} \hat f_{jk} = \hat f_{ik} \sum_{j\in\mathcal{V}} \hat \phi_{kk} \hat f_{jk}$

And from $DP$ property on has
$P(\hat f_{ik} \geq x' + \epsilon | \hat f_{ik} \geq x', \me)$ is increasing with $x'$

Setting $x = x' (\sum_{j\in\mathcal{V}})$ one as that $P(D_{ik} \geq x + \epsilon | D_{ik}
    \geq x, \me)$ is increasing with $x$ since $\sum_{j\in \mathcal{V}} \hat \phi_{kk}
    \hat f_{jk}$ is positive and independent of $i$. (\textcolor{red}{but it depends on
    $k$ ?})


For ILFM :~\\

 let $C_{i,k} = |\{j \in V, \hat{f}_{jk} = \hat{f}_{ik} = 1\}|$. As the factor matrix is
    binary, one has:
%
\[ 
d_{i,k} = \sum_{j\in V} \sigma(\hat{f}_{ik} \hat{\Phi}_{kk} \hat{f}_{jk}) =  C_{i,k} (\sigma(\hat{\Phi}_{kk})-0.5) + \frac{N}{2}
\]
%
As $\hat{f}_{ik}$ is binary, there is no positive reinforcement effect: $C_{i,k}$ does not
    increase if $\hat{f}_{ik}=1$, thus ILFM does not satisfy local preferential
    attachment.
\end{proof}

%%%%%%%%%%%%%%
%%% Feature Burstiness, MMSB/ILFM, Me
%%%%%%%%%%%%%%
\begin{proof}[Proof of proposition \ref{th:feature_exp_me}]

For IMMSB : ~\\

one has $C_k = \sum_{i,j \in V^2} \hat f_{ik} \hat f_{jk}$
From the reinforcement of the DP process $C_k$ is bursty over $\me$.

For ILFM : ~\\

one has $C_k = \sum_{i \in V} \hat f_{ik}$
From the reinforcement of the IBP process $C_k$ is bursty over $\me$.
\end{proof}



%%%%%%%%%%%%%%
%%% Local Preferential Attachment, MMSB, Mg.
%%%%%%%%%%%%%%
\begin{proof}[Proof if proposition \ref{th:ldegree_mg}]
    Assuming exchangeability in $\mg$ and that we have $c_k$ relation that occurs in the
    class $k$. The probability to observe a local degree $d_{ik}$ equal to $n$ can be written as the
    union of all sequence of relations that will have $n$ edges :

\begin{equation}
    P(d_{ik}=n | \mg) = P( \cup I_{k,n}, \cup I_{r,*})
\end{equation}

Where $I_{k,n}$ represents the union of all sequences of relations in the class $k$ where
    $d_{ik} = n$, and  $I_{r,*}$ the union of all sequences of relations outside the class
    $k$ regardless of the node degrees. 

Because we assume that $c_k$ is know, the size of the class $k$k is hold constant, the
    relation inside the class $k$ only, do not depend of relation outside this class, then
    the sequences of relation inside $k$ and outside $k$ are independent : 

    \begin{equation}
        P(d_{ik}=n | \mg) = P( \cup I_{k,n}) P( \cup I_{r,*})
    \end{equation}
    
    From the exchangeability assumptions \textcolor{red}{Actually the exchangeability for
    the sequences in the local case, should be proved also, as I am not sure that the
    following statement is trivially true : "\emph{In a HDP, the distributions of the
    sequences limited to only one component (one class) are exchangeable}" ?}, all the
    sequence that have the same number of edges have the same outcome probability. One can
    write : 

\begin{equation}
P(d_{ik}=n | \mg) = p(\cup I_r) \dbinom{c_k}{n} P( (y_{i0, k}=1,\dotsc,  y_{in, k}=1,
    y_{i(n+1), k}=0,\dotsc,y_{i(c_k), k}=0) | \mg)
\end{equation}

Applying product rule, one has the following for any node $j$ : 
\begin{equation} \label{eq:bb1}
    \frac{P(d_{ik} = n+1)}{P(d_{ik} = n)} = \frac{c_k - n}{n+1} \frac{P(y_{ij,k}=1
    | Y_i^{-ij}, \mg)}{P(y_{ij,k}=0 | Y_i^{-ij}, \mg)}
\end{equation}

Where $Y^{-ij}$ denote any sequence of relations between node $i$ and nodes  in $j \in \{0,1,\dotsc,j-1, j+1, N-1\}$,
such that the sequence of relations contains $n$ edges in the class $k$.

From the bernoulli-beta compound distribution in IMMSB, one has (it is the
gibbs sambling equations of IMMSB). : 

\begin{align}
    p(y_{ij,k}=1 | Y_i^{-ij}, \mg) = \frac{n+ \lambda_1}{c_k + \lambda_1 + \lambda_0}\\
    p(y_{ij,k}=0 | Y_i^{-ij}, \mg) = \frac{c_k - n + \lambda_0}{c_k + \lambda_1 + \lambda_0}
\end{align}

By plugin those identities in equation \eqref{eq:bb1}, it reduces to :
\begin{equation} \label{eq:bb2}
    \frac{P(d_{ik} = n+1)}{P(d_{ik} = n)} = \frac{\left(c_k - n\right) \left(\lambda_1
    + n\right)}{\left(n + 1\right) \left(c_k + \lambda_0 - n\right)}
\end{equation}


One need to evaluate the sign of the derivative of equation \eqref{eq:bb2}, thus one has :

\begin{align}
    &\frac{P(d_{ik} = n+2)}{P(d_{ik} = n+1)} - \frac{P(d_{ik} = n+1)}{P(d_{ik} = n)}
    = \frac{\left(c_k - n - 1\right) \left(\lambda_1 + n + 1\right)}{\left(n + 2\right) \left(c_k
    + \lambda_0 - n - 1\right)} - \frac{\left(c_k - n\right) \left(\lambda_1
    + n\right)}{\left(n + 1\right) \left(c_k + \lambda_0 - n\right)} \nonumber \\
    &= \frac{ \left(n + 1\right) \left(- c_k + n + 1\right) \left(c_k
    + \lambda_0 - n\right) \left(\lambda_1 + n + 1\right) - \left(c_k - n\right) \left(\lambda_1 + n\right) \left(n + 2\right) \left(c_k
    + \lambda_0 - n - 1\right)}{\left(n + 1\right) \left(n
    + 2\right) \left(c_k + \lambda_0 - n\right) \left(c_k + \lambda_0 - n - 1\right)} \nonumber \\
    &= \frac{1}{\left(n + 1\right) \left(n + 2\right) \left(c_k + \lambda_0 - n\right) \left(c_k + \lambda_0 - n - 1\right)} \biggl(n^2(1-(\lambda_1+\lambda_0))+n(2c_k(\lambda_1-1) \nonumber \\
    & \qquad \qquad  -3\lambda_0-\lambda_1+1) +c_k^2(1-\lambda_1)+c_k(\lambda_1+\lambda_0-\lambda_1\lambda_0-1)-\lambda_1\lambda_0-\lambda_0 \biggr)
\end{align}

We know that $n < c_k -2$, thus the denominator is always positive. Note also that if  $c_k >> n >> 0$, the
burstiness is satisfy if $\lambda_1 < 1$, otherwise it depends on the values of
the zeros of the polynomial. The solution of the zeros of the numerator are given by the following equations : 

\begin{multline} \label{eq:sqrt1}
    n_{1/2} =  \frac{1}{2 \left(1 -(\lambda_1 + \lambda_0)\right)} \biggl(2 c_k (1-\lambda_1
    ) + \lambda_1 + 3 \lambda_0 -1 \pm 
    \biggl( 4 c_k^{2}\lambda_0(1 -  \lambda_1) 
    - 4 c_k  (\lambda_1^{2} \lambda_0 - \lambda_1 \lambda_0^{2} + \lambda_0^{2} + \lambda_0 ) \\
    - 4 \lambda_1^{2} \lambda_0 + \lambda_1^{2} - 4 \lambda_1 \lambda_0^{2} + 6 \lambda_1
    \lambda_0 - 2 \lambda_1 + 5 \lambda_0^{2} - 2 \lambda_0 + 1\biggr)^{\frac{1}{2}}\biggr) 
\end{multline}

Here, when $c_k >> 0$ the square is positive if $\lambda_1 < 1$ and admits a positive solution if $\lambda_1 + \lambda_0 < 1$. In order to obtain a solution not in an imaginary space, one can look for the constraint over
$c_k$ to obtain positive values in the square root of \eqref{eq:sqrt1}. Then the solution of the polynomial in the square root of equation \eqref{eq:sqrt1} are : 

\begin{align}
    c_{k,1/2} = \frac{1}{2} \left(-  \left(\lambda_1
    + \lambda_0 + 1\right) \pm \sqrt{\frac{\left(\lambda_0(\lambda_1 - 1) + 1\right)}{\lambda_0 \left(\lambda_1 - 1\right)}} \left(\lambda_1 + \lambda_0 - 1\right)\right) 
\end{align}

\end{proof}


%%%%%%%%%%%%%%
%%% Feature Burstiness, Mg
%%%%%%%%%%%%%%
\begin{proof}[Proof if proposition \label{th:feature_mg}]
on paper, to complete.
\end{proof}

