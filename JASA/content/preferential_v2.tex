\section{Preferential attachment}
\label{sec:burstiness}

The preferential attachment effect can be observed directly over the global network, or indirectly on latent classes.

\subsection{Global preferential attachment}

Stochastic blockmodels lead to the following generative process for links \footnote{For simplicity in the notation, we consider that nodes can be linked to themselves. Excluding such links does not raise particular problems.} :

\begin{align*}
    &\textrm{For each node } i \in \{1, .., N\}  \\
    &\qquad\textrm{For each node } j \in \{1, .., N\}; \\
    &\quad \qquad\textrm{generate a link between } i \textrm{ and } j \textrm{ with probability} \\
    &\quad \qquad P(y_{ij}=1 | \mathcal{M})
\end{align*}

Where $\mathcal{M}$ is either $\mathcal{M}_e$ or $\mathcal{M}_g$.

This processus, for any given node $i$, considers all nodes in turn, from node 1 to node $N$. The indexing, i.e. the mapping between nodes and integers in $[1..N]$, is however arbitrary and the results that follow are \textcolor{red}{indexed?} for every all possible indexing.

For a given node $i$ at step $p$ fo the above process, $p$ nodes, from node 1 tp node $p$, have been considered and links from these nodes to node $i$ generated or not. We will denote by $d_i^{(p)}$ the degree atnode $i$, i.e. the number of links of node $i$, at the $p^{th}$ step of this process. By definition:

\begin{equation*}
d_i^{(p)} = \sum_{j=1}^p yij
\end{equation*}

The following definition \textcolor{red}{?} how a model $\M$ (as defined above) behaves with the global preferential preferential attachment.

\begin{definition}[global preferential attachment]
Let $\M$ be a model as defined above. We say that $\M$ satisfies the globak preferential attachment effect iff for any indexing, for node $i$, $i \leq i \leq N$ and $p$, $1 \leq p < N$; 
$P(d_i^{(N)} \geq n+1 | d_i^{(p)} = n; \M)$ increases with $n$, $1 \leq n < p$.

If $P(d_i^{(N)} \geq n+1 | d_i^{(p)} = n; \M)$ is independant of $n$, the model is said to be neutral wrt the global preferential attachment effect.

\end{definition}

This definition direclty translates the fact that the more links a node $i$ has at some point in the process, the more mokely a new link will be created with that node. this is exactly the preferential attachment effect.

For both ILFM and IMMSB, in $\M_e$, the generation of links are independant of each other. The fact that $n$ links have been created after $p$ steps has no impact on the future links to given node. In $\M_g$, as one first needs to generate $F$ and $\Phi$ prior to generate all the links, a similar behavior is likely to be observed. Intuitively thus, both ILFM and IMMSB are neutral wrt the global preferential attachment effect. The following property formalises this intuition.


\begin{proposition}[Global preferential attachment] \label{th:mg_glob}
Both ILFM and IMMSB, for both $\M_e$ and $\M_g$, are neutral wrt the global preferential attachment effect.
\end{proposition}

\begin{proof}
We first consider model $\M_e$. Fix any indexing, a node $i$, $i \leq i \leq N$, and a step $p$, $1 \leq p < N$. One has, $\forall n, n \leq n < p$ :

\begin{align*}
P(d_i^{(N)} \geq n+1 | d_i^{(p)}=n, \M_e) &= 1 - P(d_i^{(N)}=n | d_i^{(p)}=n, \M_e) \\
        &= 1 -P(y_{i,p+1}=0, ...,y_{iN}=0 | \M_e ) \\
        &= 1 - \prod_{j=p+1}^N P(y_{ij}=0 | \M_e)
\end{align*}

where the last equality comes from the fact, in $\M_e$, the links are independantly generated.

Similarly:

\begin{align*}
P(d_i^{(N)} \geq n+2 | d_i^{(p)=n+1, \M_e} &= 1 - \prod_{j=p+1}^N P(y_{ij}=0 | \M_e) \\
                    &=P(d_i^{(N)} \geq n+1 | d_i^{(p)}=n, \M_e)
\end{align*}

This complete the proof for $\M_e$.

For $\M_g$, it suffices to observe that the observe result is true for any $F, \Phi$ and that:

\begin{equation*}
P(d_i^{(N)} \geq n+1 | d_i^{(p)}=n, \M_g)  = \int\int_{F,\Phi} P(F|\M_g)P(\Phi|\M_g) P(d_i^{(N)} \geq n+1 | d_i^{(p)}=n, F, \Phi) \ dF d\Phi
\end{equation*}

As $P(d_i^{(N)} \geq n+1 | d_i^{(p)}=n, F, \Phi) = P(d_i^{(N)} \geq n+2 | d_i^{(p)}=n+1, F, \Phi)$, one has:
\begin{equation*}
P(d_i^{(N)} \geq n+2 | d_i^{(p)}=n+1, \M_g) = P(d_i^{(N)} \geq n+1 | d_i^{(p)}=n, \M_g)
\end{equation*}

\end{proof}


We now turn to local preferential attachment, as exemplified in \textcolor{red}{ADD ref to leskovek}


\subsection{Local preferential attachment}

\subsubsection{ILFM}

For ILFM, the situation wrt local preferential attachment is very similar to the one for global preferential attacment.
This is due to the fact that, in $\M_e$ (i.e. given $F$ and $\Phi$), a local degree can be defined in the same way as the global degree above.

Considering the same generative process as before, for $\M_e$ and ILFM, the local degree in class $k$, $1\leq k\leq K$, for a node $i$ such that $f_{ik}=1$ is defined by:

\begin{equation*}
d_{i,k}^{(p)} = \sum_{j=1, f_{ik}=1}^p y_{ij}
\end{equation*}

Note that if $f_{ik}=0$, $d_{i,k}^{(p)} = 0$ for all $p$.

This then leads to the followin definition at the ocal preferential attachment for ILFM.

\begin{definition}[ILFM - local preferential attachment, $\M_e$]

We say that ILFM, in $\M_e$ statisfies the local preferential attachment iff for any indexing, for any node $i$, $1\leq i \leq N$ such that $f_{ik}=1$, and for any step $p$, $1\leq p < N$:

\begin{equation*}
P(d_{i,k}^{(N)} \geq n+1 | d_{i,k}^{(p)}=n,\M_e) \textrm{ increases with } n , 1\leq n < p
\end{equation*}

if $P(d_{i,k}^{(N)} \geq n+1 | d_{i,k}^{(p)}=n,\M_e)$ is independant of $n$, the model is said to be neutral wrt to the local preferential attachment effect.

\end{definition}
As before we have the following property.

\begin{proposition}
ILFM, for $\M_e$, is neutral wrt the preferential attachment effect.
\end{proposition}

\begin{proof}
The proof is identical to the first part of the proof for Property \ref{th:mg_glob}.
\end{proof}

The definition of the local degree we have used does not directly translate to $\M_g$ as it is based on th knwoledge of $F$. One can nevertheless define an expected local degree for $\M_g$ as:

\begin{align*}
gd_{ik}^{(p)} &= \E_{F,\Phi}[d_{i,k}^{(p)}] \\
    &= \int\int_{F, \Phi} P(F|\M_g)P(\Phi|\M_g) d_{i,k}^{(p)}\ dF d\Phi
\end{align*}

A notion of local preferential attachment for $\M_g$ can then be defined on the basis of this expected local degree.

\begin{definition}[ILFM - local preferential attachment, $\M_g$]
We say that ILFM in $\M_g$ satisfies the local preferential attachment effect iff for any indexing, for any node $i$, $1 \leq i \leq N$, for any step $p$, $1 \leq p < N, \exists \epsilon \in ]0,1[$ s.t:

\begin{equation*}
P(gd_{i,k}^{(N)} \geq x + \epsilon | gd_{i,k}^{(p)} \geq x, \M_g) \textrm{ increases with } x, x \in ]0,N[.
\end{equation*}

\end{definition}

\textcolor{red}{Include a ref about Continous burstiness}

One has $\forall \epsilon \in ]0,1[, x \in ]0,N[$ :
\begin{align*}
P(gd_{i,k}^{(N)} \geq x + \epsilon | gd_{i,k}^{(p)} \geq x, \M_g) &= 1 - P(gd_{i,k}^{(N)} =x | gd_{i,k}^{(p)} = x, \M_g) \\
    &= 1 P(y_{ip+1}=0, ..., y_{iN}=0  | \M_g)
\end{align*}

Where the last equation derives from the definition of $gd_{i,k}^{(p)}$ ($gd_{i,k}^{(N)} = gd_{i,k}^{(p)}$ iff $y_{ip+1}=...=y_{iN}=0$).

\textcolor{red}{But do you implicitly assume here, that $f_{jk}=1$ for $j \in (p+1,.., N)$ ? what if it not the case ?}

The last quantity is independant of $x$, leading to:

\begin{proposition}
ILFM with $\M_g$ is neutral wrt the local preferential attachment effect.

\end{proposition}

\subsubsection{IMMSB}

For IMMSB in $\M_e$, the situation is similar to the one of ILFM in $\M_g$ as we do not have a direct access to classes, encoded in the $Z$ variables.

One can nevertheless define a local random variables $y_{ij,k}$ that are 1 if a link is generated between nodes $i$ and $j$ whithin class $k$ and 0 otherwise. One has:

\begin{align*}
P(y_{ij,k}=1 | \M_e) &= P(y_{ij}=1 | z_{i\rightarrow j} = z_{i\leftarrow j}=k, \Phi) P(z_{i\rightarrow j}|F)P(z_{i\leftarrow j}|F) \\
    &= f_{ik} \phi_{kk} f_{jk}
\end{align*}

The local degree $d_{i,k}^{(p)}$ can then be defined as the expectation over the nodes $1,...,p$ of $y_{ij,k}$:
\begin{align*}
d_{i,k}^{(p)} &= \sum_{j=1}^p P(y_{ij,k}=1 | \M_e)  \\
    &= \sum_{j=1}^p f_{ik} \phi_{kk} f_{jk}
\end{align*}

With this definition of the local degree, we can then used the same definition of the local preferential attachment, in $\M_e$, for IMMSB, than the one for ILFM. 

\begin{proposition}
IMMSB satisfies the local preferential attachment in $\M_e$.
\end{proposition}

\begin{proof}
The positive reinforcement effect of the Dirichlet Process \cite{HDP} at the basis of IMMSB corresponds to a burstiness phenomenon and directly translates, for any $i$ and any $k$, as: $\pr(\hat{f}_{ik} \ge x'+\epsilon' \mid \hat{f}_{ik} \ge x',\mathcal{M}_e)$ increases with $x'$ (for all $\epsilon'$ and $x'$ chosen according to the domain of definition of $\hat{f}_{ik}$). Setting $x=x'(\sum_{j\in V} \hat{\Phi}_{kk} \hat{f}_{jk})$ and $\epsilon = \epsilon'(\sum_{j\in V} \hat{\Phi}_{kk} \hat{f}_{jk})$ and exploiting the fact that $\sum_{j\in V} \hat{\Phi}_{kk} \hat{f}_{jk}$ is positive and independent of $i$ leads to: $\pr(d_{i,k} \ge x+\epsilon \mid d_{i,k} \ge x, \mathcal{M}_e)$ increases with $x$, which proves that IMMSB satisfies the local preferential attachment effect.

\end{proof}


\textcolor{red}{Devellopment for IMMSB in Mg...}

