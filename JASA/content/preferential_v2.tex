\section{Preferential attachment}
\label{sec:burstiness}

As mentioned before, preferential attachement can be global, in which case nodes are connected across communities, and/or local to the network communities. Preferential attachment is reminiscent of a phenomenon called \textit{burstiness}, studied in different contexts (\cite{barabasi_burst}). We introduce here definitions for the local and global preferential attachment effects that are extensions of the definitions for burstiness proposed in \cite{clinchant2010information} for text collections. We will first study global preferential attachment for the models \ifm\ and \imb\ in the two contexts defined by $\mathcal{M}_g$ and $\mathcal{M}_e$. We will then turn our attention to local preferential attachment.
%The preferential attachment effect can be observed directly over the global network, or indirectly on latent classes.

\subsection{Global preferential attachment}

Probabilistic models naturally lead to the following generative process for creating links between nodes in a network\footnote{For simplicity in the notation, we consider that nodes can be linked to themselves. Excluding such links does not raise particular problems.}. This process considers all possible pairs of nodes in turn and generates or not a link between them:

\begin{description}
 \item[1.] \textit{For each node $i \in \{1, \dotsc, N\}$},
 \begin{description}
    \item[2.] \textit{For each node $j \in \{1, \dotsc, N\}$},
       \begin{description}
          \item[3.] \textit{Generate a link between $i$ and $j$ with probability $P(y_{ij}=1 | \mathcal{M})$ where $\mathcal{M}$ is either $\mathcal{M}_e$ or $\mathcal{M}_g$}.
       \end{description}
  \end{description}
\end{description}

%\begin{algorithm}[t!]
%\caption{Deep $k$-Means algorithm}
%\label{algo:train}
%\DontPrintSemicolon
%    \KwIn{data $\mathcal{X}$, number of clusters $K$, balancing parameter $\lambda$, scheme for $\alpha$, number of epochs $T$, number of minibatches $N$, learning rate $\eta$}
%    %$\mathcal{X}, \lambda, \eta, T, m_{\alpha}, M_{\alpha}$}
%    \KwOut{autoencoder parameters $\theta$, cluster representatives $\mathcal{R}$}
%    %$\theta$, $\mathcal{R}$}
%	%\STATE{\textbf{Random initialization of} $\theta$ \textbf{and} $\vect{r}_k, \, 1 \le k \le K$}
%	Initialize $\theta$ and $\vect{r}_k, \, 1 \le k \le K$ (randomly or through pretraining)\;
%	\For(\Comment*[f]{inverse temperature levels}){$\alpha = m_{\alpha}$ to $M_{\alpha}$} {
%		\For(\Comment*[f]{epochs per $\alpha$}){$t=1$ to $T$} {
%			\For(\Comment*[f]{minibatches}){$n=1$ to $N$} {
%				Draw a minibatch $\tilde{\mathcal{X}} \subset \mathcal{X}$\;
%				Update $(\theta, \, \mathcal{R})$ using SGD (Eq.~\ref{eq:update})\;
%			}
%		}
%	}
%\end{algorithm}

As one can note, this process considers all nodes in turn, from node 1 to node $N$. An indexing, \textit{i.e.} a mapping between nodes and integers in $\{1,\cdots,N\}$, is however arbitrary and conclusions drawn from the above process should be independent of the indexing. As we will see, the results we establish below are indeed independent of the indexing.

For a given node $i$ at step $p$ of the above process, $p$ nodes, from node 1 to node $p$, have been considered and links from these nodes to node $i$ generated or not. We will denote by $d_i^{(p)}$ the degree of node $i$, i.e. the number of links of node $i$, at the $p^{th}$ step of this process. By definition:
%
\begin{equation} \label{eq:degree_def}
d_i^{(p)} = \sum_{j=1}^p y_{ij}
\end{equation}
%
As mentioned before, preferential attachment characterizes the propensity of nodes in social networks to connect to nodes that already have a lot of connections and can be stated as \textit{the higher the number of links a node has, the more likely it will get new links}. The following definition directly captures this idea:
%
\begin{definition}[Global preferential attachment]
In the above setting, a probabilistic model satisfies the global preferential attachment effect iff for any indexing, for any node $i, \, 1 \leq i \leq N$, for any $p, \, 1 \leq p < N$, $P(d_i^{(N)} \geq n+1 | d_i^{(p)} = n; \M)$ increases with $n$ ($1 \leq n < p$). If $P(d_i^{(N)} \geq n+1 | d_i^{(p)} = n; \M)$ is independent of $n$, the model is said to be neutral \textit{w.r.t.} the global preferential attachment effect. As before, $\mathcal{M}$ is either $\mathcal{M}_e$ or $\mathcal{M}_g$.
\end{definition}
%
Thus, a model satisfies the global preferential attachment effect if and only if the more links a node $i$ has at some point in the process, the more likely a new link will be created with that node.

For both \ifm\ and \imb, in $\M_e$, the generation of links are independent of each other. The fact that $n$ links have been created after $p$ steps has thus no impact on the future links to a given node. In $\M_g$, as one first needs to generate $F$ and $\Phi$ prior to generate all the links, a similar behavior is likely to be observed. Intuitively thus, both \ifm\ and \imb\ are neutral wrt the global preferential attachment effect. The following property formalizes this intuition.
%
\begin{proposition} \label{th:mg_glob}
Both \ifm\ and \imb, for both $\M_e$ and $\M_g$, are neutral wrt the global preferential attachment effect.
\end{proposition}
%
\begin{proof}
We first consider model $\M_e$. Fix any indexing, a node $i$, $i \leq i \leq N$, and a step $p$, $1 \leq p < N$. One has, $\forall n, 1 \leq n < p$ :
%
\begin{align*}
P(d_i^{(N)} \geq n+1 | d_i^{(p)}=n, \M_e) &= 1 - P(d_i^{(N)}=n | d_i^{(p)}=n, \M_e) \\
        &= 1 -P(y_{i,p+1}=0, \dotsc,y_{iN}=0 | \M_e ) \\
        &= 1 - \prod_{j=p+1}^N P(y_{ij}=0 | \M_e)
\end{align*}
%
where the last equality comes from the fact that, in $\M_e$, links are independently generated. Similarly:
%
\begin{align*}
P(d_i^{(N)} \geq n+2 | d_i^{(p)}=n+1, \M_e) &= 1 - \prod_{j=p+1}^N P(y_{ij}=0 | \M_e) \\
                    &=P(d_i^{(N)} \geq n+1 | d_i^{(p)}=n, \M_e)
\end{align*}
%
which shows that both \ifm\ and \imb\ are neutral wrt to global preferential attachment with $\M_e$.

For $\M_g$, it suffices to observe that the above result holds for all $\mat{F}$ and $\mat{\Phi}$, and not only for $\mat{\hat{F}}$ and $\mat{\hat{\Phi}}$, so that:
%
\begin{equation*}
P(d_i^{(N)} \geq n+1 | d_i^{(p)}=n, \M_g)  = \int_{\mat{F},\mat{\Phi}} P(\mat{F},\mat{\Phi}|\M_g) P(d_i^{(N)} \geq n+1 | d_i^{(p)}=n, \mat{F},\mat{\Phi}) \ d\mat{F} d\mat{\Phi}
\end{equation*}
%
As the models are neutral with $(\mat{F},\mat{\Phi})$, $P(d_i^{(N)} \geq n+1 | d_i^{(p)}=n, \mat{F},\mat{\Phi}) = P(d_i^{(N)} \geq n+2 | d_i^{(p)}=n+1, \mat{F},\mat{\Phi})$ and thus:
%
\begin{equation*}
P(d_i^{(N)} \geq n+2 | d_i^{(p)}=n+1, \M_g) = P(d_i^{(N)} \geq n+1 | d_i^{(p)}=n, \M_g)
\end{equation*}
%
which completes the proof.
\end{proof}

\vspace{0.1cm}
We now turn to local preferential attachment that deals with the fact that preferential attachment can be also observed within classes of nodes, as exemplified in \cite{LeskovecBKT08}. The classes we consider here are the latent classes of the stochastic mixed-membership models.

\subsection{Local preferential attachment}
\label{sec:local_me}

Local preferential attachment is a restriction of global preferential attachment at the community level and aims at capturing the fact that the more links a node has in a given community, the more links it will have in the future within this community. The latent classes used in \ifm\ and \imb\ play the role of latent communities gathering nodes sharing unobserved properties. Local preferential attachement can thus be studied in stochastic mixed membership models by studying how preferential attachement is captured within latent classes. This nevertheless entails that the latent classes be set in one way or another, meaning that the question of whether stochastic mixed membership models comply with the local preferential attachment effect only makes sense in $\me$, and not in $\mg$.

For \textbf{\ifm}, the situation wrt to local preferential attachment is very similar to the one for global preferential attachment. This is due to the fact that, in $\M_e$ (i.e. given $F$ and $\Phi$), a local degree can be defined in the same way as the global degree above.

Considering the same generative process as before, for $\M_e$ and \ifm, the local degree in class $k$ ($0\leq k\leq K-1$), for a node $i$ such that $f_{ik}=1$, is defined by:
%
\begin{equation*}
d_{i,k}^{(p)} = \sum_{j=1, f_{jk}=1}^p y_{ij}
\end{equation*}
%
Note that if $f_{ik}=0$, $d_{i,k}^{(p)} = 0$ for all $p$. This then leads to the following definition for local preferential attachment with \ifm.
%
\begin{definition}[\ifm\ - local preferential attachment, $\M_e$]\label{def:locdeg-discrete}
We say that \ifm, in $\M_e$, satisfies the local preferential attachment effect iff for any indexing, for any node $i, \, 1\leq i \leq N$ such that $f_{ik}=1$, and for any step $p, \, 1\leq p < N$, $P(d_{i,k}^{(N)} \geq n+1 | d_{i,k}^{(p)}=n,\M_e)$ increases with $n$ ($1\leq n < p$). If $P(d_{i,k}^{(N)} \geq n+1 | d_{i,k}^{(p)}=n,\M_e)$ is independent of $n$, the model is said to be neutral wrt to the local preferential attachment effect.
\end{definition}
%
As before, we have the following property.
%
\begin{proposition}
\ifm, with $\M_e$, is neutral \textit{w.r.t.} the local preferential attachment effect.
\end{proposition}

\begin{proof}
The proof is identical to the first part of the proof of Property \ref{th:mg_glob}.
\end{proof}

\vspace{0.1cm}
For \textbf{\imb} in $\M_e$, we do not have a direct access to classes, encoded in the $Z$ variables. One can nevertheless define  local random variables $y_{ij,k}$ that are 1 if a link is generated between nodes $i$ and $j$ within class $k$ and 0 otherwise. One has:
%
\begin{align*}
P(y_{ij,k}=1 | \M_e) &= P(y_{ij}=1 | z_{i\rightarrow j} = z_{i\leftarrow j}=k, \Phi) P(z_{i\rightarrow j}=k|F)P(z_{i\leftarrow j}=k|F) \\
    &= f_{ik} \phi_{kk} f_{jk}
\end{align*}
%
The local degree $d_{i,k}^{(p)}$ can then be defined as the expectation of $y_{ij,k}$ over the nodes $1,\dotsc,p$:
%
\begin{align}\label{def:locdegimb}
d_{i,k}^{(p)} &= \sum_{j=1}^p P(y_{ij,k}=1 | \M_e)  \nonumber \\
    &= \sum_{j=1}^p f_{ik} \phi_{kk} f_{jk}
\end{align}
%
As one can note, such a local degree is not necessarily an integer and the definition of the local preferential attachment has to be adapted accordingly.
%
\begin{definition}[\imb\ - local preferential attachment, $\M_e$]
We say that \imb, in $\M_e$, satisfies the local preferential attachment effect iff for any indexing, for any node $i, \, 1\leq i \leq N$ such that $f_{ik}>0$, for any step $p, \, 1\leq p < N$, and  for all $\epsilon$ compatible with the domain of definition of $d_{i,k}$ and $x$, $P(d_{i,k}^{(N)} \geq x+\epsilon | d_{i,k}^{(p)} \geq x,\M_e)$ increases with $x$. If $P(d_{i,k}^{(N)} \geq x+\epsilon | d_{i,k}^{(p)} \geq x,\M_e)$ is independent of $x$, the model is said to be neutral wrt to the local preferential attachment effect.
\end{definition}

This definition can be seen as the continuous counterpart of Definition~\ref{def:locdeg-discrete}. If $\epsilon$ is too large, the probability is null and is independent on $x$, hence the compatibility requirement with the domain of definition of $x$ and $d_{i,k}$.

Because of the Hierarchical Dirichlet Process underlying the \imb\ model, $\mat{f}_i$ follows a Dirichlet distribution: $\mat{f}_i \sim \Dir((\alpha_0 \beta_k + N_{ik})_{1 \le k \le K})$, with $\mat{\beta}\sim \gem(\gamma)$ and $N_{ik}$ being the number of edges connecting node $i$ through class $k$ (see for example \cite{teh2006hierarchical}) and $K$ the number of latent classes obtained. The marginals $f_{ik}$ are thus distributed according to a Beta distribution: $f_{ik} \sim \Beta(a_{ik}, b_{ik})$ with $a_{ik} = \alpha_0\beta_k + N_{ik}$ and $b_{ik} = \sum_{k'=1, k' \ne k}^{K} \alpha_0\beta_k' + N_{ik'}$.

The following property displays a sufficient condition on $x$, $\epsilon$, $a_{ik}$ and $b_{ik}$ for \imb\ to satisfy the local preferential attachment.

\begin{proposition}\label{prop:IMBlocal}
%Let $F_k^p = \sum_{j=1}^p \hat{f}_{jk}$, $x'=\frac{x}{F_k^p \phi_{kk}}$ and $\epsilon'=\frac{\epsilon}{F_k^N \phi_{kk}}$. In the region where $x$ and $\epsilon$ are such that:
%\[ F_k^N a_{ik} x'^{a_{ik}-1} (1-\epsilon') > F_k^p a_{ik} x'^{a_{ik}} + b_{ik} F_k^p (1-x'^{a_{ik}}), \]
Let $E_k^p = \sum_{j=1}^p \hat\phi_{kk} \hat{f}_{jk}$ . In the region where $x$ and $\epsilon$ are such that
\begin{equation*}
%x^{a_{ik}-1}\frac{f_{ik}}{E_{d_{ik}}}\left(a_{ik}(1-\epsilon) -x(a_{ik}-b_{ik}) \right) \geq b_{ik}\frac{(F_k^p)^{a_{ik}}\phi_{kk}^{a_{ik}-1}}{F_k^N}
x^{a_{ik}-1}\left(a_{ik}(E_k^N-\epsilon) + x(b_{ik} - a_{ik}) \right) \geq b_{ik}(E_k^p)^{a_{ik}}
\end{equation*}
$P(d_{i,k}^{(N)} \geq x+\epsilon | d_{i,k}^{(p)} \geq x,\M_e)$ increases with $x$.
\end{proposition}


\begin{proof}

We consider IMMSB in $\me$. Let $F_k^P=\sum_{j=1}^p \hat f_{jk}$ and $F_k^N=\sum_{j=1}^N \hat f_{jk}$.
Using the change of variables $x'=\frac{x}{F_k^p \phi_{kk}}$ and $\epsilon' = \frac{\epsilon}{F_k^N \phi_{kk}}$, one gets
\begin{align*}
p(d_{ik}^{(N)} > x+\epsilon | d_{ik}^{(p)} > x, \me) &= p(f_{ik} > qx'+\epsilon' | f_{ik} > x') \\
&=\frac{P(\hat f_{ik} > qx'+\epsilon')}{P(\hat f_{ik} > x')} = g(x')
\end{align*}
where $q=\frac{F_{ik}^p}{F_{ik}^N}$. The conditional distribution $g(x')$ is not trivially equal to one in the case where $qx'+\epsilon' > x' \Leftrightarrow \epsilon' > x'(1-q)$. Further, the survival function of $\hat f_{ik}$ is $P(\hat f_{ik} >x') = 1-\int_0^{x'} f(x')$ where $f(x')$ is the density of $\hat f_{ik}$. One can show that the marginal distribution of $\hat f_{ik}$ is a Beta of the form $f(x')=\Beta(a_{ik}, b_{ik})$  where $a_{ik} = \alpha_0\beta_k + N_{ik}$ and $b_{ik} = \sum_{k'\neq k} \alpha_0\beta_{k'} + N_{ik'}$  (this is a consequence of the form of the posterior distribution of the DP). In the following we ommit the references to $i$ and $k$ for the parameters of the Beta, simply noting $\Beta(a, b)$ for short. The derivative of $g$ is
\begin{equation*}
g'(x') = \frac{-q (qx'+\epsilon')^{a-1}(1-qx'-\epsilon')^{b-1}\int_{x'}^1t^{a-1} (1-t)^{b-1} dt + x'^{a-1}(1-x')^{b-1}\int_{qx'+\epsilon'}^1 t^{a-1}(1-t)^{b-1} dt}{\left(\int_{x'}^1t^{a-1}(1-t)^{b-1}dt\right)^2}
\end{equation*}
But one has
\begin{equation*}
\int_{qx'+\epsilon'}^1 t^{a-1}(1-t)^{b-1}dt \geq (qx'+\epsilon')^{a-1}\int_{qx'+\epsilon'}^1 (1-t)^{b-1}dt = (qx'+a)^{a-1}\frac{(1-qx'-\epsilon')^b}{b}
\end{equation*}
and
\begin{equation*}
\int_{x'}^1 t^{a-1}(1-t)^{b-1}dt \leq (1-x')^{b-1}\int_{x'}^1 t^{a-1}dt = (1-x')^{b-1}\frac{(1-x'^a)}{a}
\end{equation*}
Thus one can show that
\begin{align*}
Cg'(x') &\geq x'^{a-1} \frac{1-qx'-\epsilon'}{b} -q\frac{1-x'^a}{a}  \\
        &= \frac{1}{abF_k^N} \left[ ax'^{a-1}(1-\epsilon')F_k^N + F_k^P (b(x'^a-1) - ax'^a) \right]
\end{align*}
where $C=\frac{\left(\int_{x'}^1t^{a-1}(1-t)^{b-1}dt\right)^2}{(qx'+\epsilon')^{a-1}(1-qx'-\epsilon')^{b-1}(1-x')^{b-1}}$, is a positive constant. Thus, A sufficient condition for $g$ to be increasing is that
\begin{align*}
&F_k^N a x'^{a-1} (1-\epsilon') \geq F_k^p a x'^a + b F_k^p (1-x'^a) \\
&\Leftrightarrow x'^{a-1} a(1-\epsilon') \geq \frac{F_k^p}{F_k^N} ( x'^a(a-b) + b) \\
&\Leftrightarrow x'^{a-1} \left(a(1-\epsilon') - x'(a-b)\frac{F_k^p}{F_k^N} \right) \geq b \frac{F_k^p}{F_k^N}
\end{align*}
Finally, let $E_k^l=\sum_{j=1}^l \phi_{kk}f_{jk}=\phi_{kk}F_k^l$, by rolling back the change of variables, one obtains
\begin{equation*}
x^{a-1}\left(a(E_k^N-\epsilon) +x(b-a) \right) \geq b (E_k^p)^a
\end{equation*}
which concludes the proof.


\end{proof}

%\vspace{0.1cm}
% %The power law distribution assumed here for $\hat{f}_{ik}$ directly translates the positive reinforcement effect of the Dirichlet Process \cite{HDP} at the basis of \imb, that corresponds to a burstiness phenomenon.
%The definitions that we have considered for the (local and global) preferential attachment effects are extensions of the definitions for burstiness proposed in \cite{clinchant2010information}. Power law distributions have been often used to model bursty phenomena (\cite{barabasi_burst,clauset2009power} to cite but a few), and our assumption on $\hat{f}_{ik}$ reflects this usage.
% % It is easy to see, using the same development as the one we have used, that power distributions are bursty in the sense of \cite{clinchant2010information}.


%As one can note, when $F_k^p$ is small (typically in the first steps of the process), then the above condition is likely to be met and \imb\ satisfies the preferential attachment effect. However, when $F_k^p$ gets closer to $F_k^N$ the above condition is no longer met.
As one can note, when $x$ is large, $\epsilon$ is small and $b_{ik}-a_{ik}>0$ (which roughly means that the class $k$ concentrates less than half of the capacaity of the network), then the above condition is likely to be met. In this situation, the \imb satisfies the local preferential attachment effect.

We now present an experimental illustration of the above theoretical results.



