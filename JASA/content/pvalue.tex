The measures considered to evaluate the preferential attachment rely on a goodness of fit. Indeed, it has been reported that preferential attachment leads to networks characterized by a degree distribution with heavy tail drawn from a power law \cite{barabasi1999emergence}. A graphical method, most often used to verify that the observations are consistent with this law  consists in constructing the histogram representing the degree distribution and if the plot on doubly logarithmic axes approximately falls on a straight line, then one can assume that the distribution follows a power law. Thus, the comparison of the degree distribution in the log-log scale with a linear function gives us a qualitative measure for the preferential attachment. To obtain a second evaluation of the power law hypothesis for the degree distribution, we follow the statistical framework, introduced by \cite{clauset2009power}, for discerning and quantifying power-law behavior in empirical data. This framework combines maximum-likelihood fitting methods with goodness-of-fit tests based on the Kolmogorov-Smirnov statistic. It includes the following steps:


\begin{itemize}
	\item Estimate the parameters $\alpha$ and $x_\text{min}$ of the power law model. $\alpha$ is the scaling parameter of the law and $x_\text{min}$, the lower bound for the tail. It has been fixed to the smallest value observed in the distributions evaluated, in our experiments to allow their comparisons.
	\item  Using the Kolmogorov-Smirnov (KS) statistic, compute the distance $KS_{obs}$  between the degree distribution obtained on the network with the theoretical distribution corresponding to the power law with the estimated parameters.
	\item Sample $S$ synthetic datasets from the power law with the estimated parameters. For each sample  dataset $s \in S$, compute the distance $KS_{s}$ between the distribution obtained on this synthetic dataset, drawn from the power law, with the corresponding theoretical distribution using the Kolmogorov-Smirnov statistic. 
    \item Decide how many sample dataset $S$ to use, with a rule of thumb, based on a worst-case performance analysis of the test \cite{clauset2009power}. To obtain a precision of the $p$-value about $\epsilon$, one should choose $S = \frac{1}{4}\epsilon^{-2}$.  
    \item  The p-value is defined as the fraction of the resulting statistics $KS_s, s \in \{1,...,S\}$ obtained on the samples larger than the value $KS_{obs}$ computed on the network distribution.  
\end{itemize}

 If  p-value is large (close to 1), then the difference between the data and the model can be attributed to statistical fluctuations alone; if it is small, the model is not a plausible fit for the data and we can not conclude that there is an evidence for the preferential attachment in the network. 
However, as mentioned in \cite{clauset2009power} high value of the $p$-value should be considered with caution for at least two reasons. First, there may be other distribution that match the data equally or better. Second, a small number of samples of the data may lead to high p-value and reflect the fact that is hard to rule out a hypothesis in such a case.


