\section{Inference Context}

In a Bayesian context, the learning process consists in finding the posterior distribution of the random parameters $F$ and $\Phi$, given the observed data $Y$, such that : 

\begin{equation}
    \p(F, \Phi | Y, \mg) = \frac{\p(Y|F,\Phi)\p(F|\mg)\p(\Phi|\mg)}{\p(Y|\mg)}
\end{equation}

Where $\mg$ is the set the hyperparamters of the current model.


For mixed membership models the evidence $\p(Y|\mg)$ has no closed form solution which makes a direct MAP inference procedure infeasible. Thus the learning process rely on an approximate inference. It consists in a iterative procedure that updates the posterior through typically MCMC updates for true posterior recovery.

Typically MCMC update consists in the following updates of parameters :  

\begin{align}
    \hat f_{ik} &\sim \p(f_{ik} | F^{-ik}, Y, \mg) \\
    \hat \phi_{kk'} &\sim \p(\phi_{kk'} | \Phi^{-kk'}, Y, F, \mg)
\end{align}

At the end of the inference process, assuming that the MCMC has reached its  equilibrium, one can reconstruct the posterior parameters such that $\hat F = (\hat \phi_{ik})_{i,k \in V\times[0,.., K-1]}$ and $\hat \Phi = (\hat \phi_{kk'})_{k,k' \in [0,.., K-1]^2}$ and :

\begin{equation}
    \p(F, \Phi|Y, \mg) \approx \p(\hat F) \p(\hat \Phi)
\end{equation}

Finally the prediction task consists of measuring an unobserved variable $y^{new}$ given that the information from the observed data was transferred to the posterior distribution : 

\begin{align*}
    \p(y^{new} | Y, \mg) &= \int_F \int_\Phi \p(y^{new}|F,\Phi) \p(F,\Phi|\mg) dF d\Phi \\
                          &\approx \E_{\hat F, \hat \Phi} [y^{new} | \hat F, \hat \Phi] = \mathrm{Bern}(\mathcal{K}(f_i \Phi f_j^T)) \\
                          &= \p(y^{new} | \me)
\end{align*}

Where $\mathcal{K}$ is an isomorphism used to map the support of the bilinear product to a probability space, wich is a sigmoid and the identity for respectively ILFM and IMMSB. Furthermore we denote the set of estimated parameters $\me = \{\hat F,  \hat \Phi \}$.


In the rest of this paper, we will refers to both context in which we study a Bayesian model, and we recall here the fundemental property of each one :
\begin{itemize}
    \item $\mg$ that represents the set of hyperprameters of $\mathcal{M}$, such that the random graph $G$ is exchangeable, thus for any permutation $\pi$ on integers, one has: \[ P((y_{ij})_{i,j\in \mathcal{R}} | \mg) = P((y_{\pi(i)\pi(j)})_{i,j\in \mathcal{R}} | \mg) \]
    \item $\me$ that represents the set of elements ($\bm{F}$ and $\bm{\Phi}$) of $\mathcal{M}$ such that interactions are conditionally independant:  \[P((y_{ij})_{i,j\in \mathcal{R}} | \me) = \prod_{i,j\in \mathcal{R}} P(y_{ij} | \me)  \]
\end{itemize}

%
%
%  Theritical Extension
%
%

% Links to conjugate prior

%\paragraph{Remark (Diaconis-Ylvisaker characterisation of conjugate priors.}~\\
%In the case of conjugate distribution, the Diaconis-Ylvisaker theorem give insight about the form of the predictive distribution $\p(y^{new} | \me)$ \cite{orbanz2009functional}.
%
%\begin{theorem}[Diaconis-Ylvisaker characterisation of conjugate priors]
%Let $P_x(.|\Theta)$ be a natural exponential family model dominated by Lebesgue measure, with open parameter space $\Omega_\theta \subset \mathbb{R}^d$.
%    Let $P_\theta$ b a prior on $\Theta$ which does not concentrate on a singleton. Then $P_\theta$ is a  conjugate prior of $P_X$  w.r.t Lebesgue measure on $\mathbb{R}^d$ if and only if :
%
%\begin{equation}
%    \E_{P_\Theta(\Theta|X_1=x_1,...,X_n=x_n)} [\E_{P_X(x|\Theta=\theta)}[X]] = \frac{y+n\hat x}{a+n}
%\end{equation}
%
%\end{theorem}
%
%That is, given observation $x_1,...,x_n$, the expected value of a new draw $x$ under unknown value of the parameter is linear in the sample average $\hat x = \frac{1}{n}\sum x_i$.



% Relation of Me/Mg and aldoos-hoover theory representation.
