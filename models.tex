\section{Models}
%\emph{Yet another view} ~\\
\label{sec:models}

As mentioned before, we focus in this study on two major representatives of the latent models used for link prediction in social networks, namely the latent feature model \cite{BMF} and the mixed-membership stochastic block model \cite{MMSB}. To be as general as possible, we consider non-parametric extensions of these models, respectively based on the Indian Buffet Process (IBP) and the Hierarchical Dirichlet Process (HDP). Similar extensions have already been considered in the past, {\it e.g.} through the Infinite Latent Feature model \cite{ILFRM} and through conditional random fields \cite{iMMSB} or a dynamic version of the Hierarchical Dirichlet Process \cite{diMMSB}.
%\textcolor{red}{To be completed - maybe second extension not considered yet}

We now briefly describe the two models retained.
%Our two chosen baseline use prior distributions that fall into the two major classes of discrete nonparametric priors. The Hierarchical Dirichlet Process (HDP) that generalizes the Latent Dirichlet Allocation (LDA) for infinite mixtures models. On the other hand, the Indian Buffet Process (IBP), which is the generalization of the Beta-Bernoulli compound distribution (ie Beta Process), which generates infinite binary matrices. The nonparametric models in their truncated version are equivalent to well-known models such as LDA, widely used for text analysis, and Mixed Membership Stochastic Blockmodel which is an adaptation of the latter for relational learning.~\\

%We adopt the following notation; if a matrix has a negative index superscripted, it indicates that the values corresponding to this index are excluded. A dot $\bm{.}$ in the index means that we marginalize over all possible values.

\subsection{Infinite Latent Feature Model (ILFM)}

In the latent feature model, each node is represented by a vector of binary features. The probability of linking two nodes is then based on a weighted similarity between their feature vectors, the weight matrix being generated according to a normal distribution. In its non-parametric version, the feature vectors are now generated according to an IBP, leading feature vectors of infinite dimensions (even though only a finite number of dimensions are actually active). The following steps summarizes this process:
%
\begin{enumerate}
\item Generate a feature matrix $\mat{F}_{N \times \infty}$ representing the feature vector of each node: $\mat{F} \sim \IBP(\alpha)$
\item Generate a weight matrix for each latent feature:\\
 $\mat{\phi}_{mn} \sim N(0, \sigma_w), \, m,n \in \mathbb{N}^{+*}$
\item Generate or not a link between any node $i$ and any node $j$ according to: 
%
\begin{equation}
y_{ij} \sim \mathrm{Bern}(\sigma(\mat{f}_{i} \mat{\Phi} \mat{f}_{j}^\top))
\label{eq:link-ilfm}
\end{equation}
\end{enumerate}
%
where $\sigma()$ is the sigmoid function, mapping $[-\infty, +\infty]$ values to [0,1], and where $y_{ij}$ is a binary variable indicating that a link has been generated ($y_{ij}=1$) or not ($y_{ij}=0$). We will denote by $\mat{Y}$ the $N \times N$ matrix with elements $y_{ij}$. Finally, $\mat{f}_{i}$ denotes the row vector corresponding to the $i^{th}$ row of $\mat{F}$.

This model makes use of two real hyper-parameters, one for the IBP process ($\alpha$), and one for the variance of the normal distribution underlying the weight matrix ($\sigma_w$). In the case of undirected networks, the matrices $\mat{Y}$ and $\mat{\Phi}$ are symmetric and only their upper (or lower) diagonal parts are generated. Lastly, both $\mat{F}$ and $\mat{\Phi}$ are infinite matrices. In practice however, one always deal with a finite number of latent features. A graphical representation of this model is given in Figure~\ref{fig:ilfrm}.

\begin{figure}[t]
	\centering
	\minipage{0.25\textwidth}\vspace{1cm}
	\scalebox{0.88}{
	\begin{tikzpicture}
  % Define nodes
  \node[obs]                      (y) {$y_{ij}$};
  \node[latent, left=1.2cm of y] (fi) {$\mat{f}_i$};
  \node[latent, right=1.2cm of y] (fj) {$\mat{f}_j$};
  \node[latent, above= of y]    (ibp) {$\mat{F}$};;
  \node[latent, below= of y, yshift=-0.3cm]   (W) {$\mat{\Phi}$};
  \node[const, left=0.7cm of ibp]   (a) {$\alpha$};
  \node[const, right=0.7cm of W]   (sw) {$\sigma_w$};

  % Connect the nodes
  \edge {fi,fj,W} {y} ;
  \edge[dashed] {ibp} {fi,fj} ;
  \edge {sw} {W} ; 
  \edge {a} {ibp} ; 

  % Plates
  \plate {yx} {(fj)(y)} {$N$} ;
  \plate[label={[label distance=-0.6cm]195:$N$}] {} {(fi)(y)(yx.north west)(yx.south west)} {} ;
  %\plate {} {(W)} {$K\times K$};
  %\plate {} {(fi)(y)(yx.north west)(yx.south west)} {$N$} ;
\end{tikzpicture}
}
	\endminipage
	\minipage{0.25\textwidth}
	\scalebox{0.88}{
		/home/dulac/Documents/workInProgress/networkofgraphs/papers/personal/figures/draw/mmsb2.tex}
	\endminipage
	\caption{The two graphical representations of (left) the latent feature model and (right) the latent class model. The difference between the two models lies in the way representations are associated to nodes: a fixed representation is used in the case of the latent feature model, whereas the representation in the latent class model varies according to the link considered.}
	\label{fig:ilfrm}
\end{figure}

Standard Gibbs sampling and Metropolis-Hastings algorithms can be used for inference in this model. We do not detail them here and refer the interested reader to \cite{ILFRM} and \cite{IBP}.

%We here only provide the main updates, useful for the developments presented in the next sections, and refer the reader to \cite{IBP} for a detailed treatment. The Gibbs update for the matrix $\mat{F}$ are given by:
%%
%\begin{align}
%& P(f_{ik} = 1 \mid \mat{F}^{-ik}) = \frac{m_k^{-i}}{N} \nonumber \\
%& P(f_{ik} = 0 \mid \mat{F}^{-ik}) = 1 - \frac{m_k^{-i}}{N} \nonumber
%\end{align}
%%
%where $m_k^{-i}$ represents the number of active features $k$ for all nodes excluding node $i$, hence $m_k^{-i} = \sum_{j=1, j\neq i}^N f_{jk}$. $\mat{F}^{-ik}$ represents the matrix $\mat{F}$ without its element on the $i^{th}$ row and $k^{th}$ column.
%
%The learning of the weight matrix $W$ is computed using a Metropolis-Hasting algorithm in which each weight is sequentially sampled according to (\cite{IBP}): 
%%
%\begin{equation}
%P(\phi_{mn} \mid \mat{Y}, \mat{F}, \mat{\Phi}^{-mn}, \sigma_w) \propto P(\mat{Y} \mid \mat{F}, \mat{\Phi}) P(\phi_{mn} \mid \sigma_w) \nonumber
%\end{equation}
%%
%One can then choose a jumping distribution in the normal family (as for the prior), with a mean based on the previous sample:
%%
%\begin{equation} \label{eq:j_w}
%J(\phi_{mn}^* \mid \phi_{mn}) = \mathcal{N}(\phi_{mn}, \eta) \nonumber
%\end{equation}
%%
%where $\eta$ is a parameter controlling the acceptance ratio, $r_{\phi_{mn}\rightarrow \phi_{mn}^*}$, defined by:
%%
%\begin{equation} \label{eq:r_w}
%r_{\phi_{mn}\rightarrow \phi_{mn}^*} = \frac{ P(\mat{Y} \mid \mat{F}, \mat{\Phi}^*)P(\phi_{mn}^* \mid \sigma_w)J(\phi_{mn} \mid \phi_{mn}^*) }{ P(\mat{Y} \mid \mat{F}, \mat{\Phi})P(\phi_{mn} \mid \sigma_w)J(\phi_{mn}^* \mid \phi_{mn} )} \nonumber
%\end{equation}

\subsection{Infinite Mixed-Membership Stochastic Block Model (IMMSB)}

The MMSB model generates class membership distributions per node on the basis of a Dirichlet distribution. Then, for each connection between two nodes, a particular class for each node is first sampled from the class membership distribution, and the probability of connecting the two nodes is, as in the previous model, based on a Bernoulli distribution integrating the weight of the two classes. 

The non-parametric version parallels this development but considers, in lieu of the Dirichlet distribution, a hierarchical Dirichlet process, leading to the following generative model:
%
\begin{enumerate}
\item Generate the class membership matrix $\mat{F}_{N \times \infty}$:
   \begin{align}
    &\bm{\beta} \sim \gem(\gamma) \nonumber \\
    \mat{f}_i &\sim \DP(\alpha_0, \beta) \quad\text{ for }  i \in \{1, .., N\} \nonumber
   \end{align}
where $\gem$ denotes the Griffiths, ??? distribution over the set of natural numbers and $\DP$ a Dirichlet Process  \cite{HDP}.
\item Generate a weight matrix for each latent class:\\
\[ \phi_{mn} \sim \mathrm{Beta}(\lambda_0,\lambda_1), \, m,n \in \mathbb{N}^{+*} \]
\item For any node $i$ and any node $j$, choose a class from their class membership distribution and generate or not a link according to:
   \begin{align}
    z_{i \rightarrow j} &\sim \mbox{Cat}(\mat{f}_i) \nonumber \\
    z_{i \leftarrow j} &\sim \mbox{Cat}(\mat{f}_j) \nonumber \\
    y_{ij} &\sim \mathrm{Bern}(\phi_{z_{i \rightarrow j}z_{i \leftarrow j}})
    \label{eq:link-immsb}
   \end{align}
\end{enumerate}
%
We have this time four real hyper-parameters, two for the hierarchical Dirichlet process ($\gamma$ and $\alpha_0$) and two for the Beta distribution underlying the weight matrix ($\lambda_0$ and $\lambda_1$). As for the previous model, in the case of undirected networks, the matrices $\mat{Y}$ and $\mat{\Phi}$ are symmetric and only their upper (or lower) diagonal parts are generated; as before again, both $\mat{F}$ and $\mat{\Phi}$ are infinite matrices. A graphical representation of this model is given in Figure~\ref{fig:ilfrm}.

The inference is this model can be performed via collapsed Gibbs sampling updates. Most updates can be found in \cite{HDP} and \cite{diMMSB}. For completeness, we provide them in Appendix~\ref{sec:append}.
%%
%\begin{enumerate}
%\item If the class $k$ has already been observed:
%   \begin{align}
%    \pr(z_{ij} =k \mid \mat{F}^{-ij}) &\propto N_{ik}^{-ij} + \alpha_0 \beta_k
%%    \pr(f_{ji} =k \mid \mat{F}^{-ij}) &\propto N_{jk}^{-ij} + \alpha_0 \beta_k
%    \label{eq:update-immsb}
%   \end{align}
%\item In case of a new class $k_n$:
%   \begin{align}
%    \pr(z_{ij} =k_n \mid \mat{F}^{-ij}) &\propto \alpha_0 \beta_{k_n} \nonumber
%%    \pr(f_{ji} =k_n \mid \mat{F}^{-ij}) &\propto \alpha_0 \beta_{k_n} \nonumber
%   \end{align}
%\end{enumerate}

\subsection{Model Comparison}

Both ILFM and IMMSB are based on a latent representation of each node in the network (matrix $\mat{F}$) and a latent metric (matrix $\mat{\Theta}$). However, in the case of ILFM, this representation takes the form of a binary vector, whereas in IMMSB it is a vector of proportion. Interpreting the latent dimensions as characteristics or classes, one can view ILFM as performing a hard assignment on those classes whereas IMMSB performs a soft assignment. 

Another distinction between the two lies in the use of the sigmoid function in ILFM to obtain the parameter of a Bernoulli distribution from the latent representations and their weights (or correlations). Because of that, the weight matrix $\mat{\Phi}$ can take on a very general form in this model and can easily be generalized to a multivariate distribution. This is not the case in IMMSB where the elements of $\mat{\Phi}$ should lie in the interval $[0;1]$. 

Lastly, a major difference lies in the fact that the complete latent representation of each node is used in ILFM to generate a link, whereas in IMMSB, for each link to be generated, one first selects one component from the latent representations of the nodes involved in the link. This allows one to capture the fact that different classes may explain different links for the same node. At the same time, it has an impact on the homophily effect, as shown in the next section. Note that ILFM is also able to capture the fact that different classes may explain different links for the same node, even though more implicitly, by relying on different dimensions of the latent representations.

ILFM was introduced as performing better than MMSB (parametric version). However, with MCMC inference strategy, ILFM has higher time complexity due to his non-conjugacy that imply a matrix product to compute the likelihood of the model. Furthermore, the complexity of iMMSB is still $O(E^2K^2)$ (against $O(E^3 + EK + K^2)$ for ILFM), which means that this inference scheme don't allow scalability of data. But this complexity can be alleviate to work with millions of nodes by using a inference strategy that rely on variational inference \cite{gopalan2013efficient}.
